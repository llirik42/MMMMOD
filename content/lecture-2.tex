\chapter{Использование булевых переменных}

\section{Импликация для скаляров}

Пусть есть переменные $x$ и $y$. Хочется с помощью алгебраической записи записать условие: <<если $x = 0$, то $y = 0$>>.

Ответ: $x \ge y$.


Пусть теперь хочется: <<если $x = 0$, то $y = 1$>>. Ответ: $x \ge 1 - y$.

Пусть теперь хочется: <<если $x = 1$, то $y = 0$>>. Ответ: $1 - x \ge y$.

Пусть теперь хочется: <<если $x = 1$, то $y = 1$>>. Ответ: $1 - x \ge 1 - y$.


\section{Импликация для векторов}

Пусть теперь $x$ и $y$ --- это вектора, то есть $x = (x_i)$, при этом $i \in I$, $y = (y_i)$, $i \in K$.

Предположим, что есть два множества: 

\[I^0 \subset I, \qquad I^1 \subset I\]
\[I^0 \cap I^1 = \O\]


И хочется следующее: если $x$ имеет следующий вид:

\[x_i = \begin{cases}
	1,& i \in I^1\\
	0,& i \in I^0
\end{cases}\]

то $y = (0, 0, \dots, 0) = 0$.

Делается это так
\[\forall j \sum_{i \in I^1}(1-x_i) + \sum_{i \in I^0}x_i \ge y_j\]

Если бы нам хотелось, чтобы $y$ равнялся 1, то справа нужно было бы написать $1-y$.

Пусть теперь хочется немножко в обратную сторону. Если $x = 0$, то

\[y_i = \begin{cases}
	1,& i \in K^1\\
	0,& i \in K^0
\end{cases}\]

при этом условия на $K^0$ и $K^1$ аналогичны условиям на $I^0$ и $I^1$.

Делается это так:

\[
	\forall j x_j \ge \sum_{i \in K^0} y_i + \sum_{i \in K^1} (1-y_i)
\]

Однако здесь есть нюанс. Нам бы хотелось, чтобы если $x = 0$, то $y$ принимал определённые значения, а вот если $x \neg 0$, то есть $x = 1$, то чтобы $y$ мог принимать любые значения. Однако теоретически справа может стоять довольно большое число (например, если в $K^0$ содержится много индексов, а у вектора $y$ в соответствующих позициях стоят единицы). Тогда получится неравенство $1 \ge \text{<<большое число>>}$, то есть неравенство не выполняется. Хоть исходная посылка ложна, а значит $y$ может принимать любые значения.

Чтобы этого не было необходимо модифицировать неравенство, добавление коэффициента слева:
\[
	\forall j x_j \cdot \abs{K^0 \cup K^1} \ge \sum_{i \in K^0} y_i + \sum_{i \in K^1} (1-y_i)
\]

где $\abs{K^0 \cup K^1}$ --- мощность объединения множеств $K^0$ и $K^1$.


Теперь рассмотрим самый общий случай:

\[
	x_i = \begin{cases}
		1,& i \in I^1 \\
		0,& i \in I^0
	\end{cases}
	\Rightarrow
	y_i = \begin{cases}
		1,& i \in K^1 \\
		0,& i \in K^0
	\end{cases}
\]


Чтобы этого не было необходимо модифицировать неравенство, добавление коэффициента слева:
\[
\abs{K^0 \cup K^1} \cdot \Big(\sum_{i \in I^0} x_i + \sum_{i \in I^1} (1-x_i)\Big) \ge \sum_{i \in K^0} y_i + \sum_{i \in K^1} (1-y_i)
\]

Коэффициент слева был добавлен по аналогии с предыдущим пунктом.

\section{Альтернативные переменные}

Пусть в нашей задаче сформулированы 2 ограничения:

\[f_1(x) \ge b_1,\]
\[f_2(x) \ge b_2,\]

однако нам достаточно, чтобы выполнялось хотя бы одно из них. Тогда эти условия называются \definitionfont{альтернативными}. Как это можно записать?

Предположим, что мы реализуем алгоритм, который решает нашу проблему, при этом он находит решения, которые удовлетворяют всем условиям.  

Введём переменную $y \in \{0, 1\}$, значение которой будет следующим: если $y = 1$, то выполнено первое условие, а если $y = 0$, то --- второе. В идеальном случае наш алгоритм сам выберет значение нашей переменной $y$ (0 или 1), однако какое бы значение он не выбрал, то итоговое решение будет удовлетворять или первому ограничению, или второму.

Как теперь записать это алгебраически? Записывать нужно так
\[
	f_1(x) \ge b_1 - W(1-y),
\]

где $W$ --- это какая-то большая величина. Какое конкретно значение эта величина принимает, зависит от конкретной задачи. Где-то можно положить $W = 10^6$, где-то $W = 50$. Однако полностью избавиться от $W$ и записать условие без него не получится.

Второе ограничение записывается похожим образом
\[
	f_2(x) \ge b_2 - Wy
\]

Если $y =10$, то имеем
\[f_1(x) \ge b_1,\]
\[f_2(x) \ge b_2 - W \approx -\infty\]

Первое неравенство верно, как и второе. Однако важно, что выполнено первое.

Если $y = 0$, то имеем 
\[f_1(x) \ge b_1 - W \approx -\infty,\]
\[f_2(x) \ge b_2\]

Аналогично, оба неравенства выполнено, но важно, что второе неравенство выполнено безоговорочно.

\textbf{Другой способ}

Можно было ввести две переменные $y_1, y_2 \in \{0, 1\}$ и записать следующим образом:

\[
	f_1(x) \ge b_1 - W(1-y_1),
	f_2(x) \ge b_2 - W(1-y_2),
	y_1+y_2 = 1.
\]

Это почти то же самое. Условие на сумму $y_1 + y_2$ нужно, чтобы либо $y_1 = 0$, а $y_2 = 1$, либо $y_1 = 1$, а $y_2 = 0$.

\textbf{Другие ограничения}

Предположим теперь у нас ограничения с другим знаком неравенства

\[f_1(x) \le b_1,\]
\[f_2(x) \le b_2,\]

Как теперь записать то же самое условие алгебраически?

\[
	f_1(x) \le b_1 + W(1+y_1),
\]

\[
	f_2(x) \le b_2 + Wy_2.
\]

Аналогично можно было бы записать через $y$.


\section{Обработка нелинейностей}

Пусть мы встретили какую-то нелинейность $x \cdot y$. Введём новую переменную
\[
	z = x \cdot y.
\]

Однако просто заменить в выражении $x \cdot$ на $z$ нельзя, поскольку нужно изменить ограничения. Нам нужно ввести ограничение на новую переменную $z = 1$ $\Longleftrightarrow$ $x = 1$ и $y = 1$.

Это можно раскрыть через две импликации: 1) если $z = 1$, то $x = 1$ и $y = 1$. 2) Если $x = 1$ и $y = 1$, то $z = 1$.

Распишем каждую импликацию

\section{Нахождение минимума}

Пусть в нашем выражении где-то встретилось $u = \min\{x_1, x_2\}$.
