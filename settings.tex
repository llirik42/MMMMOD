\usepackage[T1, T2A]{fontenc}
\usepackage[fontsize=12pt]{fontsize}
\usepackage{anyfontsize}
\usepackage[english, russian]{babel}
\usepackage[utf8]{inputenc}
\usepackage{hyperref}
\usepackage{sectsty}
\usepackage{amsthm}
\usepackage{amssymb}
\usepackage{amsmath}
\usepackage{amsfonts}
\usepackage{physics}
\usepackage{titlesec}
\usepackage{indentfirst}
\usepackage{pgfplots}
\pgfplotsset{compat=newest}
\usepackage{enumitem}
\usepackage{ulem}
\usepackage{fancyhdr}
\usepackage{float}
\usepackage[most]{tcolorbox}
\usepackage{tabularx}
\usepackage{graphicx}
\usepackage{svg}
\usepackage[nameinlink]{cleveref}

\usetikzlibrary{arrows.meta}
\usetikzlibrary{calc}

\tikzset{
	vertex/.style={circle, draw=none, minimum size=2pt, inner sep=1pt, fill=black},
	edge/.style={->, >=stealth, thin}
}

\definecolor{darkblue}{rgb}{0,0,0.5}

\hypersetup{
	colorlinks=true,
	urlcolor=blue,
	linkcolor=blue,
}

\linespread{1.15}

\renewcommand{\rmdefault}{cmr}
\renewcommand{\sfdefault}{cmss}
\renewcommand{\ttdefault}{cmtt}
\renewcommand{\rmdefault}{cmss}
\renewcommand{\ttdefault}{cmss}

\titleformat{\chapter}{\huge\bfseries}{\thechapter}{0.5em}{\huge}[\vspace{0ex}\titlerule]

\titleformat{\section}
{\normalfont\LARGE\bfseries}
{\thesection}{0.5em}{\LARGE}

\titleformat{\subsection}
{\normalfont\Large\bfseries}
{\thesubsection}{0.5em}{\Large}

\fancyhead{}
\fancyfoot[C]{\thepage}

\renewcommand{\P}{\mathcal{P}}
\newcommand{\Q}{\mathcal{Q}}
\newcommand{\R}{\mathbb{R}}

\newcommand{\maps}{\longmapsto}
\newcommand*\circled[1]{\tikz[baseline=(char.base)]{\node[shape=circle,draw,inner sep=1pt] (char) {$#1$};}}
\newcommand{\const}{\textit{const}}
\renewcommand{\land}{\;\&\;}
\newcommand{\definitionfont}[1]{\textit{#1}}
\newcommand{\complexity}{\mathcal{O}}

\newtcolorbox{note}[1][]{
	colback=cyan!5!white,
	colframe=cyan!75!black,
	fonttitle=\bfseries,
	#1
}

\newcommand{\problemlength}[1]{\abs{#1}}

\newtheoremstyle{mainstyle}
{\topsep}
{\topsep}
{\normalfont}
{0pt}
{\large\bfseries}
{\newline\null}
{\parindent}
{\thmname{#1}\thmnumber{ #2}\textnormal{\thmnote{ (#3)}}}

\theoremstyle{mainstyle}
\newtheorem*{definition}{Определение}
\newtheorem*{unnumberedfact}{Утверждение}
\newtheorem*{fact}{Утверждение}
\newtheorem*{prooof}{Доказательство}
\newtheorem*{example}{Пример}
\newtheorem*{remark}{Замечание}
\newtheorem*{solution}{Решение}
\newtheorem*{returntoproblem}{Возвращение к задаче}
\newtheorem*{problem}{Задача}
\newtheorem*{implication}{Следствие}
\newtheorem*{algorithm}{Алгоритм}
\newtheorem*{mathmodel}{Математическая модель}
\newtheorem*{nstepprocess}{$N$-шаговый процесс}
