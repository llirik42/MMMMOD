\definecolor{darkblue}{rgb}{0,0,0.5}

\hypersetup{
	colorlinks=true,
	urlcolor=cyan,
	linkcolor=darkblue,
}

\linespread{1.15}

\renewcommand{\rmdefault}{cmr}
\renewcommand{\sfdefault}{cmss}
\renewcommand{\ttdefault}{cmtt}
\renewcommand{\rmdefault}{cmss}
\renewcommand{\ttdefault}{cmss}

\newcommand{\definitionfont}[1]{\textit{#1}}

\titleformat{\chapter}[display]
{\normalfont\huge\bfseries}
{\chaptertitlename\ \thechapter\bigskip}{0pt}{\Huge}

\titleformat{\section}
{\normalfont\huge\bfseries}
{\thesection}{1em}{\huge}

\titleformat{\subsection}
{\normalfont\LARGE\bfseries}
{\thesubsection}{1em}{\LARGE}

\fancyhead{}
\fancyfoot[C]{\thepage}

\renewcommand{\P}{\mathcal{P}}
\newcommand{\Q}{\mathcal{Q}}
\renewcommand{\O}{\mathcal{O}}
\newcommand{\maps}{\longmapsto}
\newcommand*\circled[1]{\tikz[baseline=(char.base)]{
		\node[shape=circle,draw,inner sep=1pt] (char) {#1};}}
\newcommand{\const}{\textit{const}}
\renewcommand{\land}{\;\&\;}

\newtheoremstyle{mainstyle}
{\topsep}
{\topsep}
{\normalfont}
{0pt}
{\Large\bfseries}
{\newline\null}
{\parindent}
{\thmname{#1}\thmnumber{ #2}\textnormal{\thmnote{ (#3)}}}

\theoremstyle{mainstyle}
\newtheorem*{definition}{Определение}
\newtheorem*{unnumberedfact}{Утверждение}
\newtheorem{fact}{Утверждение}
\newtheorem*{prooof}{Доказательство}
\newtheorem*{example}{Пример}
\newtheorem*{remark}{Замечание}
\newtheorem*{solution}{Решение}
\newtheorem*{returntoproblem}{Возвращение к задаче}
\newtheorem*{problem}{Задача}
\newtheorem*{implication}{Следствие}

\crefname{fact}{утверждение}{утверждения}
