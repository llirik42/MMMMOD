\chapter{Введение}

\definitionfont{Организованные системы} --- системы, в которых решения принимаются <<сознательно>> (например, люди, промышленные предприятия, магазины).

Раньше (примерно до второй мировой войны) все решения принимались лишь на основе опыта и здравого смысла. Однако позже появились сложные системы, в которых ни опыт, ни здравый смысл не работают.

Идея: рассматриваем числовые характеристики систем для принятия решений. Для этого используем математические модели (упрощённое, но адекватное описание реальной жизни).

\section{Построение математических моделей}

Исходные данные --- конкретная проблема в конкретной ситуации.

Алгоритм:
\begin{enumerate}
	\item Нужно понять, что будем оптимизировать? По каким критериям будет оценивать решения?
	
	\item Какие характеристики существенно влияют на оптимизируемую характеристику?
	
	\item Формулировка задачи с точным указанием характеристик. Переменные --- меняющиеся ($x, y, z, \dots$), параметры --- константы $a, b, c, \dots$ (например, площадь, помещение и т.д.).
	
	\item Выбор обозначений, запись с учётом обозначений, ограничения и требования для переменных.
\end{enumerate}

Пример:

Где нужно причалить к берегу? Хочется: побыстрее попасть в A.

Характеристики рассматриваемой системы:
\begin{itemize}
	\item AB = 15 км
	
	\item $v_{BA} = 5$ км/ч
	
	\item $v_{\text{лодки}} = 4$ км/ч
	
	\item Расстояние до берега = 9 км
	
	\item $O$ --- точка причаливания
	
	\item $x$ --- расстояние от B до точки $O$
	
	\item течения воды нет (то есть скорость течения = 0)
	
	\item цвет лодки не существенен :D
\end{itemize}

Оптимизируем: $t$.

Составим уравнение для $t$:

\[
	t = \underbrace{\frac{\sqrt(x^2+81)}{4}}_{\text{движение по воде}} + \underbrace{\frac{15 - x}{5}}_{\text{движение по суше}} \to \min_x
\]

Ограничения: $0 \le x \le 15$ --- потому чт иначе решение точно не оптимальное.

\[t' = \frac{2x}{8\sqrt{x^2+81}} - \frac{1}{5} = \frac{10x - 8\sqrt{x^2+81}}{40\sqrt{x^2+81}},\]

\[\frac{x}{4\sqrt{x^2+81}} = \frac{1}{5},\]

\[\frac{x^2}{16(x^2 + 81)} = \frac{1}{25},\]

\[	25x^2 = 16x^2 + 16 \cdot 81,\]

\[9x^2 = 16 \cdot 81,\]

\[x^2 = 16 \cdot 9,\]

\[x_1 = -12, x_2 = 12.\]

Решение $x_1 = -12$ не подходит, так как ...

$x = 12$ является ответом (проверить --- упражнение :D). Можно заметить, что при x > 12 $t' > 0$, а при $x < 12$ $t' < 0$, а значит $x = 12$ --- действительно точка минимума.

\chapter{Оптимизационные (экстремальные) задачи}

Формулировка экстремальной задачи:

\begin{enumerate}
	\item Есть $f(x)$
	
	\item Есть набор ограничений $g_1(x) \le b1 \dots$
	
	\item $x \in X$, $X$ --- множество всех решений
\end{enumerate}

Нужно: $\min_x f(x)$ или $\max_x f(x)$.

\definitionfont{Допустимые решения} --- все значения, которые удовлетворяют всем $\{g_i\}$.

Определение: Если $\exists x^* \forall x \in X : f(x^*) \ge f(x)$ и $x^*$ --- допустимое решение, то $x^*$ --- решение задачи.

Другая формулировка: $x^*$ является решением, если $f(x^*) = \max_{x \in X} f(x)$. $g(x)$ - может быть всё что угодно!!

Пример. Пусть наши исходные данные это
\[f(x) = c_1 x_1 + c_2 x_2 \to \max\]

\[g(x) = ax_1 + bx_2 \le d.\]

\[\forall i x_i \ge 0.\]

Нужно найти $x_1$ и $x_2$.

Пусть $p \in \P$ ($\P$ --- общая задача), а $p$ --- конкретная задача, где $c_1, c_2, a, b, d$ --- конкретные числа. 

\definitionfont{Общая задача} --- множество конкретных задач.

\definition

\definitionfont{Длина входа задачи} --- Это количество ячеек в пямяти, которое занимает задача. Допущение: каждое число занимает в памяти одну ячейку.

\section{Алгоритмы}

\definition

\definitionfont{Элементарные операции} --- арифметические операции и операции сравнения.

\definitionfont{$T_A(p)$} --- количество элементарных операций. Это трудоёмкость алгоритма $A$ решения задачи $p \in \P$.

Чем больше $\abs{p}$, тем больше $T_A(p)$, поэтому будем оценивать так:

\[T_A(p) \le \underbrace{f_A(\abs{p})}_{\text{оценка трудоёмкости}}\]

Если $f_A(\abs{p}) = C \cdot \abs{p}^k$, то будем называть алгоритм <<хорошим>> (полиномиальным).

Примерами задач, для которых существуют <<хорошие>> алгоритмы, являются математические задачи (линейное и нелинейное программирование, где $x$ --- множество векторов) и задачи комбинаторики (например, перестановки).

\subsection{Линейное программирование}

\[\sum_{j=1}^{n} c_j x_j \to \max_{(x_j)},\]

\[\sum_{j=1}^{n}a_{ij} \le b_i,\]

\[x_j \ge 0, j=1\dots n\]

или же

\[x_j \in \{0, 1, 2, \dots\}\]

Частный случай: $x_j \in \{0, 1\}$.

\section{Свойства оптимизационных задач}

\fact

\[\max_x f(x) = \max (-1) \cdot (-f(x)) = \underbrace{-\min -f(x)}_{\text{новая задача}}.\]

\fact \definitionfont{Оценка сверху} (релаксированные задачи)

\[\max_{x \in X} f(x) \le \max_{x \in X'} f(x)\]

если $x \subseteq X'$.

\fact \definitionfont{Сведение к другой задаче}

Пусть есть задача $\P$ $\max_x f(x)$. Будем говорить, что она сводится к задаче $\Q$ $\max_{y \in Y} g(y)$, если из оптимального решения $y^*$ задачи $\Q$ можно построить оптимальное решение $x^*$ задачи $\P$ с помощью \underline{эффективного алгоритма}. 

Остаётся вопрос: как понять, что $x^*$ --- оптимальное решение?

\fact Если

\begin{enumerate}
	\item $f(x^0) \ge g(y^*)$,
	
	\item $\forall x \in X \exists y : f(x) \le g(y)$
\end{enumerate}

то $x^0$ --- оптимальное решение задачи $\P$.

\proof

Нам бы хотелось доказать, что $x^0$ --- оптимальное решение. Пусть задача $\P$ имеет оптимальное решение $x^*$. По второму пункту для $x^*$ существует некоторый $y^0$ такой, что $f(x) \le g(y^0)$.

Однако по первому пункту $f(x^0) \ge g(y^*)$, где $y^*$ --- оптимальное решение для $\Q$.

Значит имеем:

\[f(x^0) \ge g(y^*) \ge g(y^0) \ge f(x^*)\]

То есть \[f(x^0) \ge f(x^*)\], где $x^*$ --- оптимальное решение задачи $\P$, а $x^0$ --- предполагаемое оптимальное решение задачи $\P$. Значит $x^0 = x^*$ --- тоже оптимальное решение задачи $\P$.

\example

$(\P)$

\[\sum_{j=1}^{n} c_j x_j \to \max_{(x_j)}\]

\[\sum_{j=1}^{n} a_j x_j \le b\]

\[x_1 + x_2 = d \]

\[x_j \ge 0, j = 1\dots n\]

Заметим, что $x_1$ можно выразить через $x_2$ (и наоборот).

Рассмотрим другую задачу:

$(\Q)$

\[c_1(d - y_2) + \sum_{j=2}^{n} c_j y_j \to \max\].

\[a_1 (d-y_2) + \sum_{j=2}^{n}a_j y_j \le b\].

И плюс какие-то ограничения на $y_j, j=2 \dots n$.


Покажем, что задача $\P$ сводится к задаче $\Q$.

\proof

Пусть $y^* = (y^*_1, \dots y^*_n) = (y^*_j)$ --- это оптимальное решение задачи $\Q$. Будем строить вектор $x^0$ (предполагаемое оптимальное решение задачи $\P$) следующим образом:

\[x^0_1 = d - y_2,\]
\[x^0_j = y_j, j = 2 \dots n\]


