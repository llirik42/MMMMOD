\chapter{Динамическое программирование}

\noindent\fbox{
	\parbox{\linewidth}{
		\qquad Динамическое программирование эффективно, если процесс принятия  решения состоит из многих шагов, то есть когда итоговое решение --- последовательность принимаемых решений (\definitionfont{стратегия}). Теоретически с помощью динамического программирования можно решить любую экстремальную задачу, однако практически это не всегда возможно, так как появляется слишком много состояний... \textbf{Главная идея динамического программирования} заключается в том, чтобы не пытаться решить задачу непосредственно, а поместить её в семейство аналогичных задач, среди которых есть просто решаемые.
	}
}

\section{Задача загрузки судна}

\example

Есть грузовое судно и набор различных контейнеров. Требуется загрузить судно контейнерами таким образом, чтобы при их дальнейшей продаже заработать как можно больше.

\bigskip

\textbf{Формальное описание}:
\begin{itemize}[nosep]
	\item есть $N$ контейнеров;
	
	\item $x_i$ --- сколько контейнеров с номером $i$ нужно взять на судно;
	
	\item $h_i$ --- место на грузовой площадке, занимаемое контейнером с номером $i$;
	
	\item $c_i$ --- ценность контейнера с номером $i$;
	
	\item $P$ --- размер грузовой площадки судна;
\end{itemize}

\[
\sum_{i=1}^{N} c_i x_i \to \max_{(x_i), \; i \in \{1, \dots, N\}}
\]
\[
\sum_{i=1}^{N} h_i x_i \le P
\]
\[
x_i \in \{0, 1, 2, 3, \dots\}, \quad i \in {1, \dots N}
\]

\solution

Для решения задачи рассмотрим семейство аналогичных задач. Задачи данного семейства будут охарактеризовываться парой $(n, p)$, где
\begin{itemize}[nosep]
	\item $p$ --- место на площадке;
	
	\item $n$ --- минимальный номер для контейнеров, которые у нас есть.
\end{itemize}

\[
n = 1 \dots N \qquad p = 0 \dots P.
\]

Вместо рассмотрения общей задачи про все $N$ контейнеров и всю грузовую площадку размером $P$ будем рассматривать задачи, в которых нам <<доступны>> не все контейнеры, а лишь начинающиеся с номера $n$, а также в которых нам <<недоступна>> вся грузовая площадка судна, а лишь её часть размером $p$.

Запишем целевую функцию и ограничения для элемента семейства
\[
\sum_{i=n}^{N} c_i x_i \to \max_{(x_i), \; i \in \{n, \dots N\}}
\]
\[
\sum_{i=n}^{N} h_i x_i \le p
\]
\[
x_i \in \{0, 1, 2, 3, \dots\} \quad i \in \{n, \dots, N\}
\]

Мы формально записали формулировку задачи для некоторого элемента семейства, при этом этот элемент характеризуется парой $(n, p)$.

Зададимся вопросом: а есть ли в этом семействе задачи, которые можно легко решить? Да, например если $n = N$, то есть если у нас в распоряжении есть лишь контейнеры с номером $N$.

\textbf{Введём обозначение}. Пусть $\boxed{f_n(p)}$ --- оптимальное значение целевой функции задачи $(n, p)$. По своей суть $f_n(p)$ ---это максимальный доход, который мы получим, если будем на площадку размера $p$ грузить контейнеры с номерами $n, n+1, n+2, \dots, N$.

Легко заметить, что
\[
f_N(p) = \bigg[\frac{p}{h_N}\bigg],
\]

то есть решать задачу $(N, p)$ для любого $p \le P$ мы умеем. Теперь нужно совершить переход к решению исходной задачи
\[
f_N(p) \longrightarrow f_1(P).
\]

Для этого сформулируем принцип.

\definition[принцип оптимальности для оптимальной стратегии]

\definitionfont{Оптимальная стратегия обладает тем свойством, что каким бы не было первое решение, последующие решения должны образовывать оптимальную стратегию относительно состояния, полученного по итогам первого решения}.

\example

Пусть мы стоим у доски и нам захотелось как можно быстрее выйти из аудитории через дверь. Каким бы ни был наш первый шаг, если мы хотим дойти до двери как можно быстрее, придётся всё время действовать оптимально. То есть даже если первый шаг был оптимальным, но потом мы накосячили и пошли неоптимально, стратегия точно не получится оптимальной.

\bigskip

\textbf{Вернёмся к задаче}. Предположим, что мы умеем считать
\[
f_{n+1}(p), f_{n+2}(p), \dots, f_{N}(p),
\]

однако нам бы хотелось посчитать $f_n(p)$, как это можно сделать? Предположим, что $x_n = x = 1$, тогда груз с номером $n$ даст нам ценность $c_n \cdot x = c_n$ и займёт на площадке место $h_n \cdot x = h_n$. Запишем целевую функцию
\[
\forall p \le P \qquad f_n(p) = \max_{h_n \cdot x \le p, \; x = 0, 1, 2, \dots} \big\{c_n \cdot x + f_{n+1}(p - h_n \cdot x)\big\}.
\]

Мы получили \textbf{рекуррентное соотношение динамического программирования}. На данном этапе $f_n(p)$ можно получить перебором значение $x_n = x$, а $f_{n+1}(\dots)$ нам уже известно.

Какой здесь будет алгоритм при реализации? Алгоритм будет включать в себя $N$ шагов, на каждом из которых мы будем вычислять $f_n(p)$. По сути мы будем заполнять табличку

\begin{table}[h!]
	\centering
	\begin{tabular}{|c|  c | c | c | c | c| c|} 
		\hline
		$p$ & $f_1(p)$ & $f_2(p)$ & $\dots$ & $f_n(p)$ & $f_{N-1}(p)$ & $f_N(p)$\\ [0.5ex] 
		\hline
		0 & $\times$ & $\vdots$ & $\vdots$ & $\vdots$ & $\vdots$ & $\vdots$ \\ 
		1 & $\times$ & $\vdots$ & $\vdots$ & $\vdots$ & $\vdots$ & $\vdots$ \\
		2 & $\times$ & $\vdots$ & $\vdots$ & $\vdots$ & $\vdots$ & $\vdots$ \\
		3 & $\times$ & $\vdots$ & $\vdots$ & $\vdots$ & $\vdots$ & $\vdots$ \\
		$\vdots$ & $\vdots$ & $\vdots$ & $\vdots$ & $\vdots$ & $\vdots$ & $\vdots$ \\
		$P-1$ & $\times$ & $\vdots$ & $\vdots$ & $\vdots$ & $\vdots$ & $\vdots$ \\
		$P$ & $\dots$ & $\vdots$ & $\vdots$ & $\vdots$ & $\vdots$ & $\vdots$ \\ [1ex] 
		\hline
	\end{tabular}
\end{table}

Для решения задачи нам важно лишь $f_1(P)$, считать $f_1(p)$ для всех $p \le P$ не нужно.

Алгоритм имеет две стадии:
\begin{enumerate}[nosep]
	\item \textit{обратный ход} $N \to 1$ --- заполнение таблицы,
	
	\item \textit{прямой ход} $1 \to N$ --- вычисление оптимального решения.
\end{enumerate}

Сначала мы будем идти с конца и для всех значений $p$ считать $f_N(p)$. Потом на основании этого будем для всех значений $p$ считать $f_{N-1}(p)$ и так далее. Когда мы заполним все столбцы, кроме $f_1(p)$, мы посчитаем $f_1(P)$ и на этом заполнение таблички закончим.

Как теперь по табличке получить само оптимальное решение --- отдельный вопрос, который будет рассмотрен позже.

\section{$N$-шаговый процесс принятия решений}

Пусть $N$ --- число шагов в нашем процессе. Исходные данные задачи:
\begin{itemize}
	\item $P_i$ --- множество возможных состояний на $i$-ом шаге, $i = 1 \dots N+1$. Шага с номером $N+1$ не будет, поэтому $p_{N+1}$ можно считать фиктивной переменной, введённой для удобства. В рассмотренной ранее задачи множеством наших состояний были числа $\{0, 1, 2, \dots, P\}$ --- то есть размер на площадке, занимаемым всеми выбранными контейнерами на некотором шаге;
	
	\item $p_1 = p_0$ --- некоторая известная заданная величина;
	
	\item $Q_i$ --- множество возможных решений на $i$-ом шаге, $i = 1 \dots N$.
	
	\item $T_i(p, q)$ --- функции перехода из состояния $p$ в состояние $q$ на $i$-ом шаге, $i = 1 \dots N$;
	
	\item $Q_i(p)$ --- множество допустимых решений на $i$-ом шаге в состоянии $p$, то есть
	\[
	Q_i(p) = \{q \in Q_i \qquad T_i(p, q) \in P_{i+1}\}
	\]	
\end{itemize}

Берём $q \in Q_1(p_0)$. На каждом шаге выбираем состояние из множества допустимых состояний. То есть $\{q_1, q_2, \dots, q_n\}$ --- наша допустимая стратегия. С помощью функции перехода переводят нас в допустимое решение, то есть
\[
p_{i+1} = T_i(p_i, q_i) \in P_{i+1}
\]

Как понять, что стратегия хорошая? Введём $g_i(p, q)$ --- функция дохода на $i$-ом шаге, если мы в состоянии $p$ принимаем решение $q$.

\section{Задача выбора оптимальной стратегии $N$-шагового процесса}

Пусть у нас задача с теми же исходными данными:

\begin{itemize}
	\item $N$ --- число шагов;
	
	\item $g_i(p, q)$ --- функция дохода на $i$-ом шаге, если мы в состоянии $p$ принимаем решение $q$;
	
	\item $P_i$ --- $\dots$;
	
	\dots
\end{itemize}

Нам нужно:
\[
\max_{\{q_1, q_2, \dots, q_N\}} \sum_{i=1}^{N} g_i(p_i, q_i)
\]

с ограничениями
\[
p_1 = p_0 \qquad p_i \in P_i, i \in \{2, \dots, N+1\}
\]
\[
q_i \in Q_i, i \in \{1, \dots, N\}
\]
\[
p_{i+1} = T_i(p_i, q_i), i \in \{1, \dots, N\}
\]

Решим задачу по аналогии с задачей погрузки судна: рассмотрим семейство задач и рассмотрим элемент семейства $(n, p)$. Рассмотрим данный элемент семейства (шаг $n$)
\[
\max_{\{q_n, q_{n+1}, \dots, q_N\}} \sum_{i=n}^{N} g_i(p_i, q_i)
\]
с ограничениями
\[
p_n = p
\]
\[
p_i \in P_i, i \in \{n+1, \dots, N+1\}
\]
\[
q_i \in Q_i, i \in \{n, \dots, N\}
\]
\[
p_{i+1} = T_i(p_i, q_i), i \in \{n, \dots, N\}
\]

В данных обозначениях исходная задача --- это $f_1(p_0)$. Напишем рекуррентные соотношения в общем виде. Пускай мы знаем как решать <<простую задачу>>
\[
\forall p \in P_{N} \qquad f_N(p) = \max_{q \in Q_{N}(p)} g_N(p, q)
\]

Теперь нам нужно осуществить переход от <<простой задачи>> к исходной. Пусть мы находимся на шаге $n$, тогда
\[
f_n(p) = \max_{q \in Q_{n}(p)} \Big[g_n(p, q) + \underbrace{f_{n+1}\big(T_{n}(p, q)\big)}_{\text{уже знаем решение}}\Big]
\]

Но мы не знаем, как выглядят оптимальные решения на каждом шаге, то мы не знаем стратегию. Для этого будем на каждом шаге считать $f_n(p)$ и $q_n(p)$ --- значение $q$, на котором достигается максимум в состоянии $p$. Это значение запоминаем.

\[
p_1^* = p_0, \qquad q_1^* = q_1(p_1^*),
\]
\[
p_2^* = T_1(p_1^*, q_1^*), \qquad q_2^* = q_2(p_2^*),
\]
\[
p_3^* = T_2(p_2^*, q_2^*), \qquad q_3^* = q_3(p_3^*),
\]
\[
\dots
\]
\[
p_N^* = T_{N_1}(p_{N_1}^*, q_{N_1}^*), \qquad q_N^* = q_N(p_N^*),
\]

Совокупность всех значений $\{q_i^*\}_{i=1}^{N}$ и будет нашей оптимальной стратегией. Однако почему это будет именно оптимальной стратегией?

Итак, имеем стратегию $\{q_1^*, q_2^*, \dots, q_N^*\}$ --- стратегия. Докажем, что она оптимальна. Покажем, что
\[
f_1(p_0) = \sum_{i=1}^N g_i(p_i^*, q_i^*).
\]

\prooof

\[
f_1(p_0) \stackrel{def}{=} f_1(p_1^*) = \max_{q \in Q_{1}(p_1^*)} \Big[g_1(p_1^*, q) + f_{2}\big(T_{1}(p_1^*, q)\big)\Big]
\]

По определению максимум происходит при $q = q_1^* = q(p_1^*)$, поэтому
\[
f_1(p_0) = g_1(p_1^*, q_1^*) + f_{2}\big(T_{1}(p_1^*, q_1^*)\big) =  g_1(p_1^*, q_1^*) +  g_2(p_2^*, q_2^*) + \dots = \dots = \text{имеем, что нужно}
\]

\example[о машине на острове]

На некотором острове есть машина; для работы машины нужны детали, которые могут изнашиваться. В скором времени нужно будет вызвать самолёт, который сможет доставить необходимые детали, однако максимальная масса груза ограничена. Сколько деталей какого типа нужно заказать?

\textbf{1. Что будем оптимизировать?} Ответ: машина должна работать максимальное время.

\textbf{2. Что существенно влияет на оптимизируемую характеристику?}

\begin{itemize}
	\item $i = 1 \dots N$ --- вид детали;
	
	\item $h = \{h_i\}_{i=1}^N$ --- масса деталей;
	
	\item $t = \{t_i\}_{i=1}^N$ --- срок службы деталей;
	
	\item $x = \{x_i\}_{i=1}^N$ --- сколько нужно взять деталей в посылку;
	
	\item $d = \{d_i\}_{i=1}^N$ --- количество работающих деталей в машине;
	
	\item $P$ --- максимальная масса посылки;
\end{itemize}

\textbf{3. Математическая формулировка задачи}

Пусть $T$ --- время работы машины, тогда
\[
T = \min_{i = 1 \dots N} \big((x_i + d_i) \cdot t_i\big) \to \max
\]
\[
\sum_{i=1}^{N} h_i x_i \le P
\]

\solution

Решим задачу с помощью динамического программирования следующим образом
\begin{itemize}[nosep]
	\item \underline{количество шагов процесса} = количество видов деталей = $N$;
	
	\item \underline{на $i$-ом шаге} будем определять, сколько деталей $i$-го типа нужно взять в посылку;
	
	\item \underline{текущее состояние} --- свободная масса груза в посылке, то есть сколько ещё массы можно использовать.
\end{itemize}

\bigskip

Будем рассматривать семейства задач, которые определяются парой $(n, p)$. Фактически это означает, что у нас в распоряжении
\begin{itemize}[nosep]
	\item есть детали не всех видов, а лишь от $n$ до $N$, $n \le N$;
	
	\item есть не вся масса посылки $P$, а лишь её часть $p \le P$.
\end{itemize}

Пусть $f_n(p)$ --- максимальная время, которое проработает машина в семействе задаче $(n, p)$. Заметим, что если $n = N$, то задача легко решается
\[
\forall p \le P \qquad f_N(p) = \bigg(d_N + \bigg[\frac{p}{h_n}\bigg]\bigg) \cdot t_N
\]

Тогда решение состоит в том, что нужно заказать деталей вида $N$ на максимум, то есть сколько можем, столько и заказываем.

Теперь когда у нас база для решения задачи, осуществим переход $f_{n+1} \to f_n(p)$ в предположении, что $f_{n+1}(p)$ мы знаем $\forall p \le P$.
\[
\boxed{f_n(p) = \max_{\stackrel{x = x_n}{h_n \cdot x \le p, \; x \ge 0}} \min\Big\{(d_n + x) \cdot t_n \; ; \; f_{n+1}(p - x \cdot h_n)\Big\}}
\]

$(d_n + x) \cdot t_n$ --- это сколько проработают детали $n$-го вида с учётом уже имеющихся в машине и тех, которые будут заказаны ($x = x_n$ --- сколько деталей данного вида нужно заказать), а $f_{n+1}(p - x \cdot h_n$ --- это по предположению уже известное максимальное время работы всех остальных деталей. Для решения задачи нужно для всех $p \le P$ найти значение $f_n(p)$ и соответствующие значения $x$, на которых максимум и достигается.

В исходной формулировке у нас было
\[
f_n(p) = \max_{q \in Q_{n}(p)} \Big[g_n(p, q) + f_{n+1}\big(T_{n}(p, q)\big)\Big],
\]

Отличие в задаче про машину состоит в том, что у нас в рекуррентном соотношении не суммирование, а взятие минимума. Теоретически здесь может быть и умножение, но большой роли при решении это не играет. Важно, что написано рекуррентное соотношение.


\textbf{Конкретная задача}

\begin{itemize}
	\item $N = 4$;
	
	\item $h = \{3, 2, 3, 2\}$ (кг);
	
	\item $t = \{6, 3, 2, 4\}$ (дней);
	
	\item $d = \{1.5, 1.5, 1.5, 1.5\}$.
\end{itemize}

$d_i$ = 1.5 может означать, например, что в машине установлены две детали вида $i$, при этом одна отработала половину своего срока, а вторую только недавно установили.

Помним, что наши состояния --- это оставшаяся масса груза $p$.

\begin{table}[h!]
	\centering
	\begin{tabular}{ | c | c | c | c | c | } 
		\hline
		$p$ & $(f_1, q_1)$ & $(f_2, q_2)$ & $(f_3, q_3)$ & $(f_4, q_4)$ \\ 
		\hline
		0 & $\times$ & $(3, 0)$   & $(3, 0)$ & $(6, 0)$ \\\hline
		1 & $\times$ & $(3, 0)$   & $(3, 0)$ & $(6, 0)$ \\\hline
		2 & $\times$ & $(3, 0/1)$ & $(3, 0)$ & $(10, 1)$ \\\hline
		3 & $\times$ & $(4.5, 0)$ & $(5, 1)$ & $(10, 1)$ \\\hline
		4 & $\times$ & $(4.5, 0)$ & $(5, 1)$ & $(14, 2)$ \\\hline
		5 & $\times$ & $(5, 0)$   & $(5, 1)$ & $(14, 2)$ \\\hline
		6 & $\times$ & $(5, 1)$   & $(6, 2)$ & $(18, 3)$ \\\hline
		7 & $\times$ & $\times$   & $\times$ & $\times$ \\\hline
		8 & $(6, 0)$ & $(6, 1)$   & $(7, 2)$ & $(22, 4)$ \\\hline
	\end{tabular}
\end{table}

Будем заполнять таблицу справа налево, запоминая $q_i$ --- то значение $x$, на котором достигается максимум.

\begin{enumerate}
	\item[$\boxed{n=4}$]
	
	На первом шаге $n = N = 4$, запишем выражение для $f_4(p)$
	
	\[
	f_4(p) = \bigg(d_n + \bigg[\frac{p}{h_4}\bigg]\bigg) \cdot t_4 = \bigg(1.5 + \bigg[\frac{p}{2}\bigg]\bigg) \cdot 4,
	\]
	\[
	q_4 = \bigg[\frac{p}{2}\bigg].
	\]
	
	То есть мы можем посчитать значение $f_4(p)$ для любого $p \le P$, при этом нам известно, на котором достигается максимум ($q_4$).
	
	Посчитаем значение $f_4$ для всех возможных состояний
	\[
	f_4(0) = 6, \quad q_4 = \bigg[\frac{0}{2}\bigg] = 0;
	\]
	\[
	f_4(1) = 6, \quad q_4 = \bigg[\frac{0}{2}\bigg] = 0;
	\]
	\[
	f_4(2) = 10, \quad q_4 = \bigg[\frac{0}{2}\bigg] = 1;
	\]
	\[
	f_4(3) = 10 \quad q_4 = \bigg[\frac{0}{2}\bigg] = 1;
	\]
	\[
	f_4(4) = 14, \quad q_4 = \bigg[\frac{0}{2}\bigg] = 2;
	\]
	\[
	f_4(5) = 14, \quad q_4 = \bigg[\frac{0}{2}\bigg] = 2;
	\]
	\[
	f_4(6) = 18, \quad q_4 = \bigg[\frac{0}{2}\bigg] = 3;
	\]
	\[
	f_4(8) = 22, \quad q_4 = \bigg[\frac{0}{2}\bigg] = 4.
	\]
	
	Занесём все эти данные в последний столбец таблицы.
	
	\item[$\boxed{n=3}$] На втором шаге $n = 3$. Запишем выражение для $f_3(p)$
	
	\begin{align*}
		f_3(p) =& \max_{\stackrel{x = x_3}{h_3 \cdot x \le p, \; x \ge 0}} \min\Big\{(d_3 + x) \cdot t_3 \; ; \; f_{4}(p - x \cdot h_3)\Big\} \\
		=& \max_{\stackrel{x = x_3}{3x \le p, \; x \ge 0}} \min\Big\{3 + 2x \; ; \; f_{4}(p - 3x)\Big\}
	\end{align*}
	
	На данном шаге нам нужно для всех $0 \le p \le P$ вычислить значение $f_3(p)$ и запомнить значение $x$, на котором достигается максимум в каждом конкретном состоянии. Для каждого состояния мы будем перебирать различные значения $x$. Для примера найдём $f_3(8)$. Для этого нам нужно перебрать значения $x \in \{0, 1, 2\}$. При $x > 2$ неравенство $3x \le 8$ уже не выполняется, а $x < 0$ нам не подходят по условию. Вычислим значение $\min$ для каждого из этих трёх значений $x$
	
	\[
	x = 0: \qquad \min\big\{3 + 2 \cdot 0 \; ; \; f_4(8)\big\} = \min\{3, 22\} = 3
	\]
	\[
	x = 1: \qquad \min\big\{3 + 2 \cdot 1 \; ; \; f_4(5)\big\} = \min\{5, 14\} = 5
	\]
	\[
	x = 2: \qquad \min\big\{3 + 2 \cdot 2 \; ; \; f_4(2)\big\} = \min\{7, 10\} = \circled{7}
	\]
	
	То есть мы рассмотрели три разных значения $x$ (0, 1, 2), для каждого из них вычислили значение $\min\Big\{(d_n + x) \cdot t_n \; ; \; f_{n+1}(p - x \cdot h_n)\Big\}$, а после этого выбрали из трёх итоговых значение максимальное --- 7, при этом запомнили, что это значение достигается при $x = 2$. Запишем это в таблицу
	
	
	
	Однако данные вычисления можно записать существенно короче
	
	\[
	f_3(8) = \begin{array}{c|l}
		0 & \{3 \; ; \; f_4(8) = 22\} = 3 \\
		1 & \{5 \; ; \; f_4(8) = 14\} = 5 \\
		2 & \{7 \; ; \; f_4(8) = 10\} = \circled{7}
	\end{array}
	\]
	
	В первом столбце у нас идут перебираемые значения $x$, а далее для каждого из них вычисление $\min$, само слово <<min>> писать здесь излишне. В кружок обведено значение $\max$. Данной нотации и будем придерживаться всюду далее. Посчитаем $f_3$ для всех $p$
	
	\[
	f_3(0) = \begin{array}{c|l}
		0 & \{3 \; ; \; f_4(0) = 6\} = \circled{3}
	\end{array}
	\]
	
	\[
	f_3(1) = \begin{array}{c|l}
		0 & \{3 \; ; \; f_4(1) = 6\} = \circled{3}
	\end{array}
	\]
	
	\[
	f_3(2) = \begin{array}{c|l}
		0 & \{3 \; ; \; f_4(2) = 10\} = \circled{3}
	\end{array}
	\]
	
	\[
	f_3(3) = \begin{array}{c|l}
		0 & \{3 \; ; \; f_4(3) = 10\} = 3 \\
		1 & \{5 \; ; \; f_4(0) = 6\} = \circled{5}
	\end{array}
	\]
	
	\[
	f_3(4) = \begin{array}{c|l}
		0 & \{3 \; ; \; f_4(4) = 14\} = 3 \\
		1 & \{5 \; ; \; f_4(1) = 6\} = \circled{5}
	\end{array}
	\]
	
	\[
	f_3(5) = \begin{array}{c|l}
		0 & \{3 \; ; \; f_4(5) = 14\} = 3 \\
		1 & \{5 \; ; \; f_4(2) = 10\} = \circled{5}
	\end{array}
	\]
	
	\[
	f_3(6) = \begin{array}{c|l}
		0 & \{3 \; ; \; f_4(6) = 18\} = 3 \\
		1 & \{5 \; ; \; f_4(3) = 10\} = 5 \\
		2 & \{7 \; ; \; f_4(0) = 6\} = \circled{6}
	\end{array}
	\]
	
	\[
	f_3(8) = \begin{array}{c|l}
		0 & \{3 \; ; \; f_4(8) = 22\} = 3 \\
		1 & \{5 \; ; \; f_4(3) = 10\} = 5 \\
		2 & \{7 \; ; \; f_4(2) = 10\} = \circled{7}
	\end{array}
	\]
	
	Занесём все данные в таблицу. Ещё раз повторение: В столбец $(f_3, q_3)$ возле максимального значения, обведённого в кружочек, для каждого состояния $p$ мы ещё записываем значение $x$, в котором достигается максимум. Так, для $f_3(8)$ это $x = 2$, для $f_3(5)$ это $x = 1$ и т.д.
	
	\item[$\boxed{n = 2}$] На третьем шаге $n = 2$
	
	\begin{align*}
		f_2(p) =& \max_{\stackrel{x = x_2}{h_2 \cdot x \le p, \; x \ge 0}} \min\Big\{(d_2 + x) \cdot t_2 \; ; \; f_{3}(p - x \cdot h_2)\Big\} \\
		=& \max_{\stackrel{x = x_2}{2x \le p, \; x \ge 0}} \min\Big\{4.5 + 3x \; ; \; f_{3}(p - 2x)\Big\}
	\end{align*}
	
	\[
	f_2(0) = \begin{array}{c|l}
		0 & \{4.5 \; ; \; f_3(0) = 3\} = \circled{3}
	\end{array}
	\]
	
	\[
	f_2(1) = \begin{array}{c|l}
		0 & \{4.5 \; ; \; f_3(1) = 3\} = \circled{3}
	\end{array}
	\]
	
	\[
	f_2(2) = \begin{array}{c|l}
		0 & \{4.5 \; ; \; f_3(2) = 3\} = \circled{3} \\
		1 & \{7.5 \; ; \; f_3(0) = 3\} = \circled{3}
	\end{array}
	\]
	
	\[
	f_2(3) = \begin{array}{c|l}
		0 & \{4.5 \; ; \; f_3(3) = 5\} = \circled{4.5} \\
		1 & \{7.5 \; ; \; f_3(1) = 3\} = 3
	\end{array}
	\]
	
	\[
	f_2(4) = \begin{array}{c|l}
		0 & \{4.5 \; ; \; f_3(4) = 5\} = \circled{4.5} \\
		1 & \{7.5 \; ; \; f_3(2) = 3\} = 3 \\
		2 & \{10.5 \; ; \; f_3(0) = 3\} = 3
	\end{array}
	\]
	
	\[
	f_2(5) = \begin{array}{c|l}
		0 & \{4.5 \; ; \; f_3(5) = 5\} = 4.5 \\
		1 & \{7.5 \; ; \; f_3(3) = 5\} = \circled{5} \\
		2 & \{10.5 \; ; \; f_3(1) = 3\} = 3
	\end{array}
	\]
	
	\[
	f_2(6) = \begin{array}{c|l}
		0 & \{4.5 \; ; \; f_3(6) = 6\} = 4.5 \\
		1 & \{7.5 \; ; \; f_3(4) = 5\} = \circled{5} \\
		2 & \{10.5 \; ; \; f_3(2) = 3\} = 3 \\
		3 & \{13.5 \; ; \; f_3(0) = 3\} = 3
	\end{array}
	\]
	
	\[
	f_2(8) = \begin{array}{c|l}
		0 & \{4.5 \; ; \; f_3(8) = 7\} = 4.5 \\
		1 & \{7.5 \; ; \; f_3(6) = 6\} = \circled{6} \\
		2 & \{10.5 \; ; \; f_3(4) = 5\} = 5 \\
		3 & \{13.5 \; ; \; f_3(2) = 3\} = 3 \\
		4 & \{16.5 \; ; \; f_3(0) = 3\} = 3
	\end{array}
	\]
	
	\item[$\boxed{n = 1}$] На четвёртом шаге $n = 1$
	
	Наша исходная задача --- это $f_1(8)$. Для всех остальных значений $p \neq 8$ $f_1(p)$ не нужно считать.
	
	\begin{align*}
		1_3(8) =& \max_{\stackrel{x = x_1}{h_1 \cdot x \le 8, \; x \ge 0}} \min\Big\{(d_1 + x) \cdot t_1 \; ; \; f_{2}(8 - x \cdot h_1)\Big\} \\
		=& \max_{\stackrel{x = x_1}{3x \le 8, \; x \ge 0}} \min\Big\{9 + 6x \; ; \; f_{2}(8 - 3x)\Big\}
	\end{align*}
	
	\[
	f_1(8) = \begin{array}{c|l}
		0 & \{9 \; ; \; f_2(8) = 6\} = \circled{6} \\
		1 & \{15 \; ; \; f_2(5) = 5\} = 5 \\
		2 & \{21 \; ; \; f_2(2) = 3\} = 5
	\end{array}
	\]
	
	Вспомним, что $f_1(8)$ --- максимальное время работы машины для исходной задачи. Поскольку мы получили, что $f_1(8) = 6$, то в исходной задаче после заказа груза машина проработает ещё 6 дней.
	
	\textbf{Как же теперь загрузить самолёт?}
	
	\begin{enumerate}
		\item $p_1^* = 8$, $q_1^* = 0$
		
		\item $p_2^* = 8$, $q_2^* = 1$
		
		\item $p_3^* = 8 - h_2 \cdot q_2^* = 6$, $q_3^* = 2$
		
		\item $p_4^* = p_3^* - h_3 \cdot q_3^* = 0$, $q_4^* = 0$
	\end{enumerate}
	
	То есть оптимальное решение --- это
	\[
	x_1 = q_1^* = 0,
	\]
	\[
	x_2 = q_2^* = 1,
	\]
	\[
	x_3 = q_3^* = 2,
	\]
	\[
	x_4 = q_4^* = 0.
	\]
	
	То есть наша стратегия --- это $\{0, 1, 2, 0\}$; при данной стратегии машина проработает ещё 6 дней. При любых других стратегиях время работы будет меньше.
\end{enumerate}

\remark

Чтобы не заполнять всю табличку, можно было в начале прибегнуть к оптимизации, посчитав множество возможных состояний для каждого шага. Тогда бы мы считали на шаге $i$ значения $f_i(p)$ не для всех $p \le P$, а лишь для этих самых возможных состояний (значений).

Например, можно заметить, что для подсчёта $f_1(8)$ нам нужно было знать лишь $f_2(8)$, $f_2(5)$ и $f_2(2)$. Значение $f_2(p)$ для остальных $p$ нам в итоге вообще не пригодились.
