\chapter{Введение}

\definition

\definitionfont{Организованные системы} --- системы, в которых решения принимаются <<сознательно>>. Примеры таких систем: люди, промышленные предприятия, магазины.

Примеры задач:
\begin{itemize}[nosep]
	\item где построить магазин, чтобы получать наибольшую прибыль?
	
	\item сколько производить деталей на заводе, чтобы отношение между доходом и выручкой было наибольшим?
\end{itemize}

\bigskip

\begin{note}
Примерно до второй мировой войны все сложные решения принимались лишь на основе опыта и здравого смысла. Однако позже появились сложные системы, в которых опыта и здравого смысла оказалось недостаточно. Тогда же появилась \textbf{идея} рассматривать числовые характеристики систем для принятия решения, при этом должны использоваться \definitionfont{математические модели} --- упрощённые, но адекватные описания реальной жизни.
\end{note}

\section{Построение математических моделей}

\begin{note}
	 Математические модели строятся на основании \definitionfont{исходных данных} --- конкретных проблем в конкретных жизненных ситуациях.
\end{note}

\algorithm[построения математических моделей]

\begin{enumerate}[nosep]
	\item[]
	
	\item Нужно понять, \textbf{что будем оптимизировать?} По каким критериям будем оценивать решения? Например, если нас спрашивают, где построить магазин, то нам нужно понять, как выбрать для этого место. Нужно, чтобы была максимальная прибыль или чтобы была наибольшая удалённость от конкурентов? Если мы закупаем детали для предприятия, то нужно узнать, хочется ли нам наибольшую прибыль или же минимальные издержки. А может нам важно количество произведённой продукции?
	
	\item Нужно понять, \textbf{какие характеристики существенно влияют на оптимизируемую характеристику}. Например, если наша оптимизируемая характеристика --- это прибыль предприятия, то нужно определить, из чего складываются выручка и затраты.
	
	\item \textbf{Формулировка задачи} с точным указанием всех характеристик. Каждая из характеристик должна относиться либо к \definitionfont{переменным}, либо к \definitionfont{параметрам}. Первые могут меняться (обозначаются они через $x, y, z, \dots$), а вторые --- это константы (обозначаются они через $a, b, c, \dots$). Например, переменные --- это выручка и затраты предприятия, а параметры --- это площадь помещения, количество станков, количество работников.
	
	\item Выбор всех обозначений и \textbf{математическая запись} с учётом ограничений и требований для переменных.
	
	\item \sout{Понимание, что изначальная проблема состоит в другом, не учтены такие-то параметры, а значит нужно вернуться к самому началу.}
\end{enumerate}

\remark

Алгоритм примерный, поэтому на практике этапы и их содержание могут отличаться от того, что написано в алгоритме.

\problem[о лодке]

Для примера рассмотрим следующую задачу: ваш знакомый плывёт на лодке, и ему нужно попасть в определённую точку на берегу. Он спрашивает вас, куда ему нужно причалить.

\begin{center}
	\begin{tikzpicture}
		\draw[black, very thick] (-1, -1) -- (4.5, 2.3);
		\draw[gray, dashed] (4.5, -1.1) -- (3, 1.4);
		\filldraw[black] (0.5, -0.1) circle (2pt) node[anchor=south]{A};
		\filldraw[black] (3, 1.4) circle (2pt) node[anchor=south]{B};
		\filldraw[black] (2, 0.8) circle (2pt) node[anchor=south]{O};
		\filldraw[black] (4.5, -1.1) circle (2pt) node[anchor=west]{C};
		\draw [-{Stealth[length=2.5mm]}] (2, -0.5) -- node[above=1mm] {\text{Лодка\quad\;}} (4.2, -1.1);
		\draw [-{Stealth[length=2.5mm]}] (-2, -3.5) -- node[above=1mm] {\text{Берег\quad\;}} (0, -0.7);
		\draw [|<->|] (1.60294, 1.46176) -- node[above=1mm] {$x$} (2.60294, 2.06176);
		\draw [|<->|] (2.16176, 2.79706) -- node[above=1mm] {$15$} (-0.33824, 1.29706);
		\draw [|<->|] (5.42647, -0.54412) -- node[above=1mm] {$9$} (3.92647, 1.95588);
	\end{tikzpicture}
\end{center}

\textbf{Характеристики рассматриваемой системы}

\begin{itemize}[nosep]
	\item $C$ --- текущее местоположение лодки;
	
	\item $A$ --- точка, в которую нужно попасть;
	
	\item $B$ --- ближайшая к лодке точка берега;
	
	\item $O$ --- точка причаливания;
	
	\item $AB$ = 15 км;
	
	\item $BC$ = 9 км (расстояние от лодки до берега);
	
	\item $v_{\text{по суше}} = 5$ км/ч;
	
	\item $v_{\text{лодки}} = 4$ км/ч;
	
	\item $OB = x$;

	\item $0 \le x \le 15$, потому что иначе решение точно не будет оптимальным;
	
	\item течения воды нет, то есть скорость течения равна нулю;
	

	\item \sout{цвет лодки не существенен.}
\end{itemize}

\bigskip

\solution

Будем оптимизировать время причаливания $t$s
\begin{align*}
	t &= \underbrace{\frac{OC}{v_{\text{по воде}}}}_{\text{движение по воде}} + \underbrace{\frac{OA}{v_{\text{по суше}}}}_{\text{движение по суше}} \\
	&= \boxed{\frac{\sqrt{x^2+81}}{4} + \frac{15 - x}{5} \to \min_x}
\end{align*}

Для решения задачи найдём нули производной
\[\derivative{t}{x} = \frac{2x}{8\sqrt{x^2+81}} - \frac{1}{5} = 0,\]
\[\frac{x}{4\sqrt{x^2+81}} = \frac{1}{5},\]
\[\frac{x^2}{16(x^2 + 81)} = \frac{1}{25},\]
\[25x^2 = 16x^2 + 16 \cdot 81,\]
\[9x^2 = 16 \cdot 81,\]
\[x^2 = 16 \cdot 9,\]
\[x_1 = -12, \qquad x_2 = 12.\]

Решение $x = -12$ не подходит ввиду ограничения выше, а вот $\boxed{x = 12}$ является ответом.

\section{Экстремальные задачи}

\definition

\definitionfont{Экстремальная задача} формулируется следующим образом

\begin{enumerate}[nosep]
	\item Есть функция $f(x)$, значение которой нужно оптимизировать;
	
	\item Есть набор ограничений $g_1(x) \le b_1, g_2(x) \le b_2, \dots$;
	
	\item $x \in X$, $X$ --- \definitionfont{множество всех решений},
\end{enumerate}

при этом нужно найти
\[\min_{x \in X} f(x) \quad \text{или} \quad \max_{x \in X} f(x).\]

\definition

\definitionfont{Допустимые решения} --- множество всех значений $x \in X$, которые удовлетворяют всем ограничения $\{g_i\}$.

\definition

Допустимое решение $x^*$ называется \definitionfont{оптимальным решением задачи}, если
\[\forall x \in X \qquad f(x^*) \ge f(x).\]

или то же самое
\[f(x^*) = \max_{x \in X} f(x).\]

Вторая запись означает, что мы ищем максимальное значение $f(x)$, перебирая все $x$ из множества $X$.

\remark

Ограничения $\{g_i(x)\}$ могут быть какими угодно.

\example

Пусть наши исходные данные это
\[f(x) = c_1 x_1 + c_2 x_2 \to \max_x,\]
\[g(x) = ax_1 + bx_2 \le d,\]
\[x_1 \ge 0, \quad x_2 \ge 0,\]

а задача состоит в поиске оптимальных значений $x_1$ и $x_2$.

\definition

Будем говорить, что есть \definitionfont{общая задача} $\P$, а $p \in \P$ --- конкретная задача, в которой у всех параметров есть конкретные значения, например, если бы в задаче выше $c_1, c_2, a, b, d$ были бы конкретными числами. То есть \definitionfont{общая задача} --- это множество конкретных задач.

\definition

\definitionfont{Длина входа задачи} --- это количество ячеек в памяти, которое занимает задача с допущением, что каждое число занимает в памяти ровно одну ячейку. Будем обозначать это $\problemlength{p}$.

\section{Алгоритмы и их трудоёмкость}

\definition

\definitionfont{Элементарные операции} --- арифметические операции и операции сравнения.

\definition

\definitionfont{Трудоёмкость алгоритма} $A$ решения задачи $p \in \P$ --- это количество элементарных операций, используемых в этом алгоритме. Будем обозначать это $T_A(p)$.

\remark

Чем больше $\problemlength{p}$, тем больше $T_A(p)$, поэтому целесообразно оценивать трудоёмкость так
\[T_A(p) \le f_A\big(\problemlength{p}\big).\]

\definition

Алгоритм $A$ будем называть \definitionfont{<<эффективным>>} (\definitionfont{полиномиальным}), если
\[
f_A\big(\problemlength{p}\big) = \underbrace{C}_{\const} \cdot \problemlength{p}^k.
\]

\begin{note}
Примерами задач, для которых существуют <<хорошие>> алгоритмы, являются, например математические задачи, в которых $x$ --- множество векторов (линейное и нелинейное программирование), и задачи комбинаторики (например, перестановки).
\end{note}

\definition

\definitionfont{Задача линейного программирования} определяется следующим образом

\[\sum_{j=1}^{n} c_j x_j \to \max_{(x_j)},\]
\[\sum_{j=1}^{n}a_{ij} x_j \le b_i, \quad i = 1\dots m,\]
\[x_j \in \{0, 1, 2, \dots\}, \quad j = 1\dots n.\]

Распространённый частный случай: $x_j \in \{0, 1\}$.

\section{Примеры задач}

\subsection{Задача о машине}

\problem\label{pr:car_on_island}

На некотором острове есть машина; для работы машины нужны детали, которые могут изнашиваться. В скором времени нужно будет вызвать самолёт, который сможет доставить необходимые детали, однако максимальная масса груза ограничена. Сколько деталей каждого типа нужно заказать?

\mathmodel

\textbf{1. Что будем оптимизировать?} Ответ: машина должна работать максимальное время.

\bigskip

\textbf{2. Что существенно влияет на оптимизируемую характеристику?}

Пусть $i = 1 \dots n$ --- вид детали.

\bigskip

\textit{Параметры} (то есть фиксированные значения)

\begin{itemize}[nosep]
	\item $m = \{m_i\}_{i=1}^n$ --- масса деталей;
	
	\item $t = \{t_i\}_{i=1}^n$ --- срок службы деталей (часов, дней, месяцев --- неважно);

	\item $a = \{a_i\}_{i=1}^n$ --- количество работающих деталей в машине;
	
	\item $M$ --- максимальная масса посылки.
\end{itemize}

\bigskip

\textit{Переменные} (то что может меняться)

\begin{itemize}[nosep]	
	\item $x = \{x_i\}_{i=1}^n$ --- сколько нужно взять деталей в посылку.
\end{itemize}

\bigskip

\textbf{3. Математическая формулировка задачи}

Пусть $T$ --- время работы машины, тогда
\[T = \min_{i = 1 \dots n} \big((x_i + a_i) \cdot t_i\big) \to \max_{(x_i)},\]
\[\sum_{i=1}^{n} m_i x_i \le M,\]
\[x_i \ge 0.\]

Первое выражение говорит о том, что мы максимизируем время работы машины $T$ при различных $(x_i)$, то есть нам нужно подобрать такие значения $x_1, x_2, \dots, x_n$, при которых время работы будет максимальным. Выражение $(x_i + a_i)$ --- это то, сколько деталей вида $i$ будет после прилёта самолёта с посылкой. Если учесть, что каждая деталь типа $i$ имеет срок службы $t_i$, то все детали данного типа проработают $(x_i + a_i) \cdot t_i$. Ясно, что машина перестанет работать, когда какой-то вид деталей в ней отработает свой срок службы, значит время работы машины --- это минимум из времён работы всех деталей.

Второе выражение отражает ограничение задачи, которое состоит в том, что мы не можем заказать груз большей массы, чем установленная максимальная масса посылки.

Третье выражение говорит о том, что количество заказываемых деталей не может быть отрицательным --- оно и понятно.

\subsection{Задача о салфетках}

\problem\label{pr:napkins}

Пусть есть некоторое кафе, в которое каждый день ходят люди, при этом известно, сколько человек посещает кафе каждый день недели. Каждому гостю на день выдают салфетку, которую вечером стирают. Салфетки можно стирать с помощью \underline{быстрой} и \underline{медленной} стирок. Первая --- дорогая, но работает условно моментально, вторая --- дешёвая, но выдача постиранных салфеток происходит лишь через день. Как сэкономить деньги на стирке так, чтобы всем посетителям всегда хватало салфеток?

\mathmodel

\textbf{1. Что будем оптимизировать?} Ответ: нужно минимизировать затраты на стирку.

\bigskip

\textbf{2. Что существенно влияет на оптимизируемую характеристику?}

Пусть $i = 1 \dots 7$ --- день недели.

\bigskip

\textit{Параметры}

\begin{itemize}[nosep]	
	\item $c_1$ --- цена быстрой стирки;
	
	\item $c_2$ --- цена медленной стирки;
	
	\item $T$ --- общее количество салфеток в кафе;
	
	\item $p_i$ --- количество гостей в $i$-ый день недели;
\end{itemize}

\bigskip

\textit{Переменные}

\begin{itemize}[nosep]	
	\item $x_i$ --- количество салфеток, отданных в \underline{быструю} стирку в $i$-ый день недели;
	
	\item $y_i$ --- количество салфеток, отданных в \underline{медленную} стирку в $i$-ый день недели.
\end{itemize}

\bigskip

\textbf{3. Математическая формулировка задачи}

Пусть $C$ --- затраты на стирку за неделю, тогда

\[C = \sum_{i=1}^7 (c_1 x_i + c_2 y_i) \to \min_{(x_i), (y_i)},\]
\[x_i + y_i = p_i,\]
\[x_i \ge 0, \quad y_i \ge 0,\]
\[\begin{cases}
	T - y_7 \ge p_1, \\
	T - y_1 \ge p_2, \\
	T - y_2 \ge p_3, \\
	\dots \\
	T - y_5 \ge p_6, \\
	T - y_6 \ge p_7.
\end{cases}\]

Первое выражение означает, что мы минимизируем затраты на стирку в неделю при различных $x_1, x_2, \dots, x_7$ и $y_1, y_2, \dots, y_7$. Выражение $(c_1 x_i + c_2 y_i)$ означает затраты на стирку в $i$-ый день недели.

Второе выражение означает, что каждый день все грязные салфетки отправляются в стирку, то есть то, что не остаётся не постиранных.

Последние семь неравенств означают, что всем посетителям всегда хватает салфеток.

\subsection{Задача раскроя}

\problem[раскроя]\label{pr:cutting_stock}

Вашему знакомому сантехнику нужно определённое количество коротких труб, однако в магазине можно купить только длинные. Сколько нужно купить длинных труб, чтобы их можно было раскроить на короткие? Пусть нужны
\begin{itemize}[nosep]
	\item 10 труб по 6 м,
	
	\item 15 труб по 5 м,
	
	\item 26 труб по 4 м,
\end{itemize}

а в магазине продаются лишь трубы по 13 м.

\mathmodel

\textbf{1. Что будем оптимизировать?} Ответ: нужно минимизировать число покупаемых труб по 13 м.

\bigskip

\textbf{2. Что существенно влияет на оптимизируемую характеристику?}

Рассмотрим все варианты, как можно раскроить длинную трубу на короткие

\begin{table}[h!]
	\centering
	\begin{tabular}{| c | c | c | c | } 
		\hline
		№ & 6 м & 5 м & 4 м \\ 
		\hline
		1 & 2 & 0 & 0 \\\hline
		2 & 1 & 1 & 0 \\\hline
		3 & 1 & 0 & 1 \\\hline
		4 & 0 & 2 & 0 \\\hline
		5 & 0 & 1 & 2 \\\hline
		6 & 0 & 0 & 3 \\\hline
	\end{tabular}
\end{table}

То есть раскроить длинную трубу на короткие можно 6 разными способами. Например, на две трубы длиной 6 метров (первая строка таблицы), на одну трубу длиной 6 метров и на одну трубу длиной 5 метров (вторая строка таблицы) и так далее.

\bigskip

\textit{Параметры}

\begin{itemize}[nosep]
	\item $x_i$ --- количество длинных труб, раскроенных $i$-ым способом;
\end{itemize}

\bigskip

\textbf{3. Математическая формулировка задачи}

Пусть $T$ --- количество изрезанных длинных труб, тогда

\[
T = \sum_{i=1}^6 x_i \to \min_{(x_i)}
\]
\[
\begin{cases}
	2x_1 + x_2 + x_3 \ge 10 \\
	x_2 + 2x_4 + x_5 \ge 15 \\
	x_3 + 2x_5 + 3x_6 \ge 26
\end{cases}
\]

Первое выражение означает, что мы минимизируем количество раскроенных длинных труб, таким образом минимизируя затраты на их покупку.

Последние три неравенства означают, что после раскроя длинных труб мы получили достаточное количество коротких: первое неравенство для труб длиной 6 метров, второе --- 5 метров, третье --- 4 метра.

\remark

В общем случае задача раскроя является NP-полной.

\section{Свойства оптимизационных задач}

\fact[от максимума к минимуму]

\underline{Пусть} есть задача, в которой нужно максимизировать значение функции $f(x)$ при $x \in X$, \underline{тогда}

\[\max_{x \in X} f(x) = \max_{x \in X} \Big(\big(-1\big) \cdot \big(-f(x)\big)\Big) = \underbrace{-\min_x \big(-f(x)\big)}_{\text{новая задача}}.\]

То есть задача нахождения максимума функции $f(x)$ эквивалентна задаче нахождения минимума функции $\big(-f(x)\big)$.

\fact[оценка сверху, релаксированные задачи]

\underline{Если} $X \subseteq X'$, \underline{то}
\[
\max_{x \in X} f(x) \le \max_{x \in X'} f(x).
\]

\definition

Пусть есть две общие задачи: $(\P) \max{x \in X} f(x)$ и $(\Q) \max_{y \in Y} g(y)$. Будем говорить, что задача $\P$ \definitionfont{сводится к задаче} $\Q$, если $\forall p \in \P \; \forall q \in \Q$

\begin{enumerate}[nosep]
	\item существует полиномиальный алгоритм $A_1$, который переводит входные данные задачи $p$ во входные данные задачи $q$;
	
	\item существует полиномиальный алгоритм $A_2$, с помощью которого можно из оптимального решения $y^*$ задачи $q$ построить оптимальное решение $x^0$ задачи $p$,
\end{enumerate}

то есть
\[
p \quad \stackrel{A_1}{\longrightarrow} \quad q,
\]
\[
x^0 \quad \stackrel{A_2}{\longleftarrow} \quad y^*.
\]

Остаётся вопрос: как понять, что построенное решение $x^0$ --- оптимальное?

\remark

Везде далее всегда будем оптимизировать именно $\max$, а не $\min$.

\fact[сведение к другой задаче]\label{fact:reduction_to_other_problem}

\underline{Пусть} есть две задачи: $\P$ и $\Q$, при этом
\begin{enumerate}[nosep]
	\item $x^0$ --- допустимое решение задачи $\P$,
	
	\item $y^*$ --- оптимальное решение задачи $\Q$,
	
	\item $f(x^0) \ge g(y^*)$,
	
	\item $\forall x \in X \; \exists y \in Y \quad f(x) \le g(y)$,
\end{enumerate}

\underline{тогда} $x^0$ --- оптимальное решение задачи $\P$.

\prooof

Пусть задача $\P$ имеет оптимальное решение $x^*$. По четвёртому условию для $x^*$ существует некоторый $y^0$ такой, что 
\[
f(x^*) \le g(y^0). \tag{*}
\]

По второму условию $y^*$ является оптимальным решением задачи $\Q$, в то время как $y^0$ --- допустимое (необязательно оптимальное) решение задачи $\Q$, значит верно следующее
\[
g(y^*) \ge g(y^0). \tag{**}
\]

Составим цепочку неравенств
\[
f(x^0) \stackrel{(3)}{\ge} g(y^*) \stackrel{(**)}{\ge} g(y^0) \stackrel{(*)}{\ge} f(x^*).
\]

Таким образом имеем неравенство
\[
f(x^0) \ge f(x^*),
\]

хотя $x^*$ --- оптимальное решение задачи $\P$. Значит $f(x^0) = f(x^*)$, то есть $x^0$ тоже является оптимальным решением.

\example\label{ex:reduction_to_other_problem}

$(\P)$
\[\sum_{j=1}^{n} c_j x_j \to \max_{(x_j)},\]
\[\sum_{j=1}^{n} a_j x_j \le b \tag{1},\]
\[x_1 + x_2 = d \tag{2},\]
\[x_j \ge 0, \quad j = 1\dots n. \tag{3}\]

Заметим, что $x_1$ можно выразить через $x_2$ (и наоборот). Рассмотрим другую задачу

$(\Q)$
\[c_1(d - y_2) + \sum_{j=2}^{n} c_j y_j \to \max_{(y_j)},\].
\[a_1 (d-y_2) + \sum_{j=2}^{n}a_j y_j \le b,\].
\[y_j \ge 0, \quad j = 1\dots n,\]
\[y_2 \le d.\]

Покажем, что задача $\P$ сводится к задаче $\Q$.

\prooof

Пусть $y^* = \begin{pmatrix} y^*_1 & \dots & y^*_n \end{pmatrix}$ --- это оптимальное решение задачи $\Q$. Будем строить решение $x^0$ задачи $\P$ следующим образом
\[
x^0_j = \begin{cases}
	d - y_2^*,& j = 1 \\
	y_j^*,& j > 1
\end{cases}
\]

то есть $x^0 = \begin{pmatrix} d - y_2^* & y_2^* & y_3^* & \dots & y_n^* \end{pmatrix}$.

\bigskip

\textbf{Допустимость решения}

Покажем, что $x^0$ --- допустимое решение задачи $\P$. Для этого нужно показать, что оно удовлетворяет всем ограничениям. Заметим, что без условия $y_2 \le d$ значение $x_1$ может быть меньше нуля, а значит решение $x^0$ точно было бы не допустимым (3).

\begin{enumerate}[nosep]
	\item Проверим (2)
	\[
	x^0_1 + x^0_2 = d - y^*_2 + y^*_2 = d.
	\]
	
	\item Проверим (1)
	\begin{align*}
		\sum_{j=1}^{n} a_j x^0_j =& \; a_1 x^0_1 + a_2 x^0_2 + \dots + a_n x^0_n \\
		=& \; a_1 (d - y^*_2) + a_2 y^*_2 + \dots + a_n y^*_n \\
		=& \; a_1 (d-y^*_2) + \sum_{j=2}^{n}a_j y^*_j \le b.
	\end{align*}
	
	То есть $x^0$ удовлетворяет всем ограничениям, значит $x^0$ --- допустимое решение задачи $\P$.
\end{enumerate}

\bigskip

\textbf{Оптимальность решения}

Мы показали, что $x^0$ --- допустимое решение задачи $\P$. Осталось показать, что оно оптимальное. Для этого будем использовать \hyperref[fact:reduction_to_other_problem]{сведение к другой задаче}. Для этого нужно проверить, что выполняются все условия для его использования.

\begin{enumerate}[nosep]
	\item Выполнимость условия $f(x^0) \ge g(y^*)$ следует из того, что при подстановке, использованной при проверке (1), получится равенство
	\[
	f(x^0) = g(y^*).
	\]
	
	\item Выполнимость условия
	\[
	\forall x \in X \; \exists y \in Y \quad f(x) \le g(y)
	\]
	
	следует из того, что по любому $x = (x_j)$ можно построить $y = (y_j)$ следующим образом
	\[
	y_j = \begin{cases}
		d - x_2,& j = 1 \\
		x_j,& j > 1
	\end{cases}
	\]
	
	И вновь, если всё аккуратно подставить, то получится
	\[
	f(x) = g(y) \quad \rightarrow \quad f(x) \le g(y).
	\]
\end{enumerate}

Таким образом мы доказали, что можно применить \hyperref[fact:reduction_to_other_problem]{сведение к другой задаче}. Следствием этого является то, что задача $\P$ действительно сводится к задаче $\Q$.

Мы взяли задачу с $n$ переменными и перешли от неё к задаче с $n-1$ переменными. Разве не круто?!

\section{Решение задачи о салфетках}

\solution[задачи о салфетках]

Решим \hyperref[pr:napkins]{задачу о салфетках} с помощью сведения к другой задаче. Запишем математическую формулировку нашей исходной задачи

$(\P) $
\[
\qquad\qquad\qquad\qquad C = \sum_{i=1}^7 (c_1 x_i + c_2 y_i) \to \min_{(x_i), (y_i)}
\]
\[
\qquad\qquad x_i + y_i = p_i,
\]
\[
\qquad\qquad x_i \ge 0, \quad y_i \ge 0,
\]
\[
\qquad\qquad\begin{cases}
	T - y_7 \ge p_1, \\
	T - y_1 \ge p_2, \\
	T - y_2 \ge p_3, \\
	\dots \\
	T - y_5 \ge p_6, \\
	T - y_6 \ge p_7.
\end{cases}
\]

\bigskip

\textbf{Новая задача}

Новая задача $\Q$ будет эквивалентна исходной задаче $\P$, однако в ней все $x_i$ будут выражены через $y_i$ и $p_i$ следующим образом
\[
x_i = p_i - y_i.
\]

Запишем оптимизируемую характеристику в новой задаче $\Q$
\begin{align*}
	C =& \sum_{i=1}^7 (c_1 x_i + c_2 y_i) \to \min_{(x_i), (y_i)} \\
	=& \sum_{i=1}^7 \big(c_1 (p_i - y_i) + c_2 y_i\big) \to \min_{(y_i)} \\
	=&  \sum_{i=1}^7 \big(y_i (c_2 - c_1) + c_1 p_i\big) \to \min_{(y_i)} \\
	=& \underbrace{(c_2 - c_1)}_{\const, <0} \sum_{i=1}^7 y_i + \underbrace{c_1 \sum_{i=1}^{7} p_i}_{\const} \to \min_{(y_i)} \\
	\sim& \sum_{i=1}^7 y_i \to \max_{(y_i)}.
\end{align*}

То есть фактически наша новая задача $\Q$ выглядит следующим образом
\[
\sum_{i=1}^7 y_i \to \max_{(y_i)},
\]

с учётом ограничений
\[
\begin{cases}
	T - y_7 \ge p_1, \\
	T - y_1 \ge p_2, \\
	T - y_2 \ge p_3, \\
	\dots \\
	T - y_5 \ge p_6, \\
	T - y_6 \ge p_7.
\end{cases}
\]

Стоит заметить, что в задаче $\P$ все $x_i \ge 0$, а значит нужно ввести ограничения на $y_i$
\[
	y_i \le p_i.
\]

Итого, новая задача формулируется следующим образом

$(\Q)$ 
\[
\sum_{i=1}^7 y_i \to \max_{(y_i)},
\]

\[
\begin{cases}
	T - y_7 \ge p_1, \\
	T - y_1 \ge p_2, \\
	T - y_2 \ge p_3, \\
	\dots \\
	T - y_5 \ge p_6, \\
	T - y_6 \ge p_7;
\end{cases}
\]

\[
y_i \le p_i, \quad i = 1 \dots 7.
\]

\textbf{Решение новой задачи}

Нам нужно максимизировать сумму $y_i$, при этом на каждый $y_i$ есть два ограничения сверху. Так, например для $y_2$ есть ограничения
\[
\begin{cases}
	T - y_2 \ge p_3, \\
	y_2 \le p_2;
\end{cases}
\]
\[
\Updownarrow
\]
\[
\begin{cases}
	y_2 \le T - p_3, \\
	y_2 \le p_2.
\end{cases}
\]

Ясно, что если все $y_i$ выбрать максимально возможными, то и их сумма будет максимально возможной. Максимально возможное значение $y_i$, которое удовлетворяет условие --- это минимум из двух верхних границ. Например, для $y_2$ это будет
\[
	\min\{T - p_3, p_2\}.
\]

Таким образом, мы уже можем записать оптимальное решение $y^*$ задачи $\Q$
\[
y_1^* = \min\{T - p_2, p_1\},
\]
\[
y_2^* = \min\{T - p_3, p_2\},
\]
\[
\dots
\]
\[
y_6^* = \min\{T - p_7, p_6\},
\]
\[
y_7^* = \min\{T - p_1, p_7\}.
\]

\textbf{Возвращение к исходной задаче}

После того, как мы нашли оптимальное решение новой задачи $\Q$, нужно вернуться к исходной задаче $\P$. Её оптимальное решение $x^0$ выражается следующим образом
\[
x_i^0 = p_i - y_i^*.
\]

Таким образом, мы свели исходную задачу $\P$ к новой задаче $\Q$ с меньшим числом переменных, решили новую задачу и по её оптимальному решению построили оптимальное решение исходной задачи.

Почему $x^0$ --- оптимальное решение задачи $\P$? Для доказательства этого можно использовать \hyperref[fact:reduction_to_other_problem]{сведение к другой задаче} по аналогии с тем, как это было сделано в \hyperref[ex:reduction_to_other_problem]{примере}.

\section{Использование булевых переменных}

\begin{note}
Очень часто в реальных задачах встречаются самые разные логические условия. Например: <<если верно ..., то должно быть верно ...>>. Данные условия можно записать на языке формул математической логики, однако в рамках данного курса будет удобнее, если они будут записаны с использованием алгебраических выражений. Таким образом мы сможем записать любые ограничения любых задач на языке алгебры. Для записи логических условий хорошо подходят \definitionfont{булевы переменные} (переменные, которые могут принимать лишь значения $0, 1$).
\end{note}

\fact[простые условия]\label{fact:simple_conditions}

\underline{Если} $x$ и $y$ --- булевы переменные, \underline{то} имеют место следующие соответствия логических условий и алгебраической записи

\begin{table}[H]
	\centering
	\begin{tabular}{ | c | >{$}r<{$} @{\,$\ge$\,} >{$}l<{$} |}
		\hline	
		\textbf{Условие} & \multicolumn{2}{ c |}{\textbf{Запись}} \\\hline
		если $x = 0$, то $y = 0$ & x & y \\\hline
		если $x = 0$, то $y = 1$ & x & 1 - y \\\hline
		если $x = 1$, то $y = 0$ & 1 - x & y \\\hline
		если $x = 1$ ,то $y = 1$ & 1 - x & 1 - y \\\hline
	\end{tabular}
\end{table}

\prooof

Для доказательства рассмотрим лишь первые два логических условия, поскольку доказательства для остальных аналогичны.

\begin{enumerate}[nosep]	
	\item Если $x = 0$, то $y = 0$
	
	\begin{itemize}[nosep]
		\item если $x = 0$, то $y$ не может быть равен 1, потому что $0 \ngeq 1$, значит $y$ может равняться лишь 0;
		
		\item если $x = 1$, то $y$ может равняться как 0, так и 1, поскольку $1 \ge 0$ и $1 \ge 1$.
	\end{itemize}
	
	\item Если $x = 0$, то $y = 1$
	
	\begin{itemize}[nosep]
		\item если $x = 0$, то $1-y$ не может быть равен 1, а значит $1-y=0$, как следствие $y = 1$;
		
		\item если $x = 1$, то $1-y$ может равняться как 0, так и 1, значит $y$ может принимать любое значение из $\{0, 1\}$.
	\end{itemize}
\end{enumerate}

\example

Используя предыдущее утверждение, можно алгебраически записывать различные условия задач. Например, пусть в некоторой рассматриваемой задаче есть два логических условия $A$ и $B$, а нам нужно записать на языке алгебры логическое выражение <<если верно $A$, то верно $B$>>. Для этого введём две булевы переменные $x$ и $y$
\[
x = \begin{cases}
	1, & \text{$A$ --- истина} \\
	0, & \text{$A$ --- ложь}
\end{cases}
\]

\[
y = \begin{cases}
	1, & \text{$B$ --- истина} \\
	0, & \text{$B$ --- ложь}
\end{cases}
\]

Тогда <<если верно $A$, то верно $B$>> эквивалентно утверждению <<если $x = 1$, то $y = 1$>>, а по \hyperref[fact:simple_conditions]{предыдущему утверждению} алгебраически это можно записать как
\[
\boxed{1 - x \ge 1 - y}.
\]

Данное неравенство будет добавлено в ограничения рассматриваемой задачи, таким образом исходное условие про $A$ и $B$ будет учтено в математической модели.

\fact[сложные условия]\label{fact:complex_conditions}

\underline{Пусть}
\begin{itemize}[nosep]	
	\item $x = (x_i)$ и $y = (y_k)$ --- булевы векторы;
	
	\item $i \in I$, $k \in K$;
	
	\item $I^0 \subseteq I$ и $I^1 \subseteq I$ --- непересекающиеся множества;
	
	\item $K^0 \subseteq K$ и $K^1 \subseteq K$ --- непересекающиеся множества.
\end{itemize}

Векторы $x$ и $y$ будем называть <<хорошими>>, если верно следующее
\[
x_i = \begin{cases}
	0,& i \in I^0\\
	1,& i \in I^1\\
\end{cases} \qquad y_k = \begin{cases}
0,& k \in K^0\\
1,& k \in K^1\\
\end{cases}
\]

\underline{Тогда} имеют место следующие соответствия логических условий и алгебраической записи

\begin{table}[H]
	\centering
	\begin{tabular}{ | c | >{$}r<{$} @{\,$\ge$\,} >{$}l<{$} |}
		\hline
		\textbf{Условие} & \multicolumn{2}{ c |}{\textbf{Запись}} \\\hline
		$x$ --- <<хороший>> $\Rightarrow$ $y = 0$ & \sum\limits_{i \in I^1}(1-x_i) + \sum\limits_{i \in I^0}x_i & y \\\hline
		$x = 0 \Rightarrow y$ --- <<хороший>> & \norm{K^0 \cup K^1} \cdot x & \sum\limits_{k \in K^1}(1-y_k) + \sum\limits_{k \in K^0}y_k \\\hline
		$x$ --- <<хороший>> $\Rightarrow$ $y$ --- <<хороший>> & \norm{K^0 \cup K^1} \cdot \Big(\sum\limits_{i \in I^1}(1-x_i) + \sum\limits_{i \in I^0}x_i\Big) & \sum\limits_{k \in K^1}(1-y_k) + \sum\limits_{k \in K^0}y_k \\\hline
	\end{tabular}
\end{table}

\begin{note}
	В первое условии $y$ --- вектор из одного элемента, то есть фактически булева переменная.
	
	Во втором условии $x$ --- вектор из одного элемента, то есть фактически булева переменная.
\end{note}

\prooof

\begin{enumerate}[nosep]
	\item[]
	
	\item Если $x$ --- <<хороший>>, то $y = 0$
	
	\begin{itemize}[nosep]
		\item если вектор $x$ является <<хорошим>>, то обе суммы равняются нулю, значит $y$ ничего не остаётся кроме как быть равным нулю;
		
		\item если вектор $x$ не является <<хорошим>>, то слева будет число $\ge 1$, а значит $y$ может быть как $0$, так и $1$.
	\end{itemize}
	
	\item Если $x = 0$, то $y$ --- <<хороший>>
	
	\begin{itemize}[nosep]
		\item если $x = 0$, то обе суммы должны равняться нулю, значит $y_k = 1 \Leftrightarrow k \in K^1$ и $y_k = 0 \Leftrightarrow k \in K^0$, значит $y$ --- <<хороший>>;
		
		\item если $x \neq 0$, то $x = 1$, значит справа сумма сумм может быть какой угодно, поэтому $y$ может иметь любой вид.
	\end{itemize}
	
	\begin{note}
		Откуда взялся коэффициент $\norm{K^0 \cup K^1}$? Если $x \neq 0$, то $y$ должен иметь возможность принимать любые значения (посылка ложна), однако алгебраически это не так. Теоретически правая часть может быть сколь угодно большой, а левая часть без коэффициента может быть лишь не больше 1. Это означает, что наше алгебраическое выражение по смыслу не совпадает с изначальным логическим условием. Чтобы оно совпадало, нужно разрешить $y$ принимать любые значения при $x \neq 0$. Для этого как раз и добавлен коэффициент в левой части неравенства, чтобы неравенство оставалось верным при $x = 1$ и сколь угодно большой правой части.
	\end{note}
	
	\item Аналогично предыдущим пунктам.
\end{enumerate}

\example

Данное утверждение, в отличие от \hyperref[fact:simple_conditions]{простых условий}, позволяет алгебраически записывать логически условия, в которых истинными или ложными должны быть сразу несколько условий.

Например, пусть при решении некоторой задачи нам нужно записать логическое выражение <<если условия $A_1$ и $A_2$ истины, то условие $B_1$ ложно, а $B_2$ и $B_3$ истинны>>. Определим два булевых вектора $x = (x_1, x_2)$ и $y = (y_1, y_2, y_3)$ следующим образом
\[
x_i = \begin{cases}
	1,& \text{$A_i$ --- истина}\\
	0,& \text{$A_i$ --- ложь}\\
\end{cases} \qquad y_k = \begin{cases}
1,& \text{$B_k$ --- истина}\\
0,& \text{$B_k$ --- ложь}\\
\end{cases}
\]

Исходное логическое выражение эквивалентно <<если $x = (1, 1)$, то $y = (0, 1, 1)$>>. По \hyperref[fact:complex_conditions]{предыдущему} утверждению алгебраически это можно записать так (третий случай)
\[
\boxed{3 \cdot (1 - x_1 + 1 - x_2) \ge (1 - y_2) + (1 - y_3) + y_1}\tag{*}.
\]

Коэффициент $3$ --- это $\norm{K^0 \cup K^1}$ из \hyperref[fact:complex_conditions]{утверждения}. Объясним его необходимость на данном примере.

Исходное логическое выражение звучит как <<если $x = (1, 1)$, то $y = (0, 1, 1)$>>. Это означает, что если $x \neq (1, 1)$, то на $y$ нет никаких ограничений, то есть он может принимать любые значения. Проверим, верно ли это, определим $\hat{x} = (0, 0)$ и подставим его в $(^*)$ \underline{без коэффициента}
\[
1 - 0 + 1 - 0 \ge (1 - y_2) + (1 - y_3) + y_1,
\]
\[
2 \ge (1 - y_2) + (1 - y_3) + y_1.
\]

Неравенству выше не удовлетворяет вектор $\hat{y} = (1, 0, 0)$, поскольку $2 \ngeq 3$. То есть несмотря на то, что $\hat{x}$ не удовлетворяет условию, на $y$ накладываются какие-то ограничения. Значит если из $(*)$ убрать коэффициент $3$, то неравенство не будет эквивалентно логическому выражению <<если $x = (1, 1)$, то $y = (0, 1, 1)$>>.

\fact[альтернативные условия]\label{fact:alternative_conditions}

\underline{Пусть} в некоторой задаче сформулированы два ограничения
\[f_1(x) \ge b_1, \tag{1}\]
\[f_2(x) \ge b_2. \tag{2}\]

Введём булеву переменную $y$ следующим образом
\[
y = \begin{cases}
	0,& \text{выполняется (1)}\\
	1,&\text{выполняется (2)}
\end{cases}
\]

и пусть $W$ --- <<большая величина>>, то есть
\[
W \gg b_1, \quad W \gg b_2.
\]

\underline{Тогда} логическое выражение <<выполняется либо (1), либо (2)>> записывается алгебраически следующим образом
\[f_1(x) \ge b_1 - W(1-y),\]
\[f_2(x) \ge b_2 - Wy,\]
\[\]

\prooof

\begin{itemize}[nosep]
	\item[]	
	\item Если $y = 1$, то
	\[f_1(x) \ge b_1,\]
	\[f_2(x) \ge b_2 - W \to -\infty.\]
	
	Первое неравенство верно, как и второе. Однако если со вторым есть вопросы, а точно ли оно верно, то вот первое выполняется гарантировано.
	
	\item Если $y = 0$, то имеем
	\[f_1(x) \ge b_1 - W \to -\infty,\]
	\[f_2(x) \ge b_2.\]
	
	Аналогично, оба неравенства выполнены, но важно, что второе неравенство выполняется гарантировано.
\end{itemize}

Значит при любых значениях $y$ выполняется либо (1), либо (2).

\remark

То же самое можно записать с использованием двух булевых переменных $y_1$ и $y_2$
\[
y_1 = \begin{cases}
	1,& \text{(1) выполняется}\\
	0,&\text{(1) не выполняется}
\end{cases} \quad\quad y_2 = \begin{cases}
1,& \text{(2) выполняется}\\
0,&\text{(2) не выполняется}
\end{cases}
\]
\[f_1(x) \ge b_1 - W(1-y_1),\]
\[f_2(x) \ge b_2 - W(1-y_2),\]
\[y_1+y_2 = 1.\]

\begin{note}
	Без условия $y_1 + y_2 = 1$ нет гарантий, что одно из неравенств будет выполняться.
\end{note}

\remark

$W$ --- это некоторая <<большая величина>>, какое конкретно значение эта она принимает, зависит от конкретной задачи. Где-то можно положить $W = 10^6$, где-то $W = 50$, однако полностью избавиться от $W$ и записать условие без него не получится.

\remark

\underline{Если} ограничения записаны в другую сторону, то есть
\[f_1(x) \le b_1,\]
\[f_2(x) \le b_2,\]

\underline{то} записать алгебраически их можно следующим образом
\[f_1(x) \le b_1 + W(1-y_1),\]
\[f_2(x) \le b_2 + Wy_2,\]
\[y_1 + y_2 = 1.\]

Аналогично можно записать через $y$.

\remark

Смысл булевых переменных $y$, $y_1$ и $y_2$ следующий.  Предположим, что мы реализуем алгоритм, который решает нашу задачу, при этом он находит решения, которые удовлетворяют всем условиям. Наш алгоритм  такой, что он сам выберет <<наилучшие>> значения этих переменных и на их основании построит оптимальное решение.

\fact[замена нелинейностей]\label{fact:substituion_of_nonlinear}

\underline{Пусть} при решение задачи в некотором выражении нам встретилась нелинейность $\hat{x} \cdot \hat{y}$ ($\hat{x}$ и $\hat{y}$ --- булевы переменные), \underline{тогда} после замены $\hat{z} = \hat{x} \cdot \hat{y}$ необходимо ввести ограничения
\[
\begin{cases}
	2\hat{z} \le \hat{x} + \hat{y}, \\
	\hat{z} + 1 \ge \hat{x} + \hat{y}.
\end{cases}
\]

\begin{note}
	Работать с нелинейностями неудобно, поэтому всегда хочется избавить от нелинейностей, однако просто заменить в выражении $\hat{x} \cdot \hat{y}$ на $\hat{z}$ нельзя, поскольку нужно изменить ограничения задачи. 
\end{note}

\prooof

Поскольку $\hat{z} = \hat{x} \cdot \hat{y}$, верно следующее
\[
\hat{z} = 1 \quad \Longleftrightarrow \quad \hat{x} = 1 \land \hat{y} = 1
\]

Данное логические условие можно расписать через две импликации
\begin{enumerate}[nosep]
	\item Если $\hat{z} = 1$, то $\hat{x} = 1$ и $\hat{y} = 1$. Это можно записать алгебраически, используя \hyperref[fact:complex_conditions]{сложные условия}
	\[
	x = \begin{pmatrix}\hat{z}\end{pmatrix}, \qquad y = \begin{pmatrix} \hat{x} & \hat{y} \end{pmatrix},
	\]
	\[
	I^0 = \O, \quad I^1 = \{1\}, \qquad K^0 = \O, \quad K^1 = \{1, 2\}
	\]
	\[
	\abs{K^0 \cup K^1} \cdot \Big(\sum_{i \in I^1}(1-x_i) + \sum_{i \in I^0}x_i\Big) \ge \sum_{k \in K^1}(1-y_k) + \sum_{k \in K^0}y_k,
	\]	
	\[\Updownarrow\]	
	\[
	2 \cdot (1 - \hat{z}) \ge (1 - \hat{x}) + (1 - \hat{y}),
	\]
	\[
	2\hat{z} \le \hat{x} + \hat{y}.
	\]
	
	\item Если $\hat{x} = 1$ и $\hat{y} = 1$, то $\hat{z} = 1$. Данное логическое условие можно записать алгебраически, используя \cref{fact:complex_conditions}. Будем использовать самый общий случай
	\[
	x = \begin{pmatrix}\hat{x} & \hat{y}\end{pmatrix}, \qquad y = \begin{pmatrix} \hat{z} \end{pmatrix},
	\]
	\[
	I^0 = \O, \quad I^1 = \{1, 2\}, \qquad K^0 = \O, \quad K^1 = \{1\}
	\]
	\[
	\abs{K^0 \cup K^1} \cdot \Big(\sum_{i \in I^1}(1-x_i) + \sum_{i \in I^0}x_i\Big) \ge \sum_{k \in K^1}(1-y_k) + \sum_{k \in K^0}y_k,
	\]
	\[\Updownarrow\]
	\[
	(1 - \hat{x}) + (1 - \hat{y}) \ge 1 - \hat{z},
	\]
	\[
	\hat{z} + 1 \ge \hat{x} + \hat{y}.
	\]
\end{enumerate}

\section{Задача о проектах}

\problem[о проектах]

Пусть есть 5 проектов, в которые можно вложиться. Для вложения в каждый проект нужно внести определённую сумму денег. Все проекты после вложения в них принесут определённый доход через какое-то время. Есть определённые условия, на которых можно вкладываться в проекты. Как получить наибольшую прибыль, имея ограниченные ресурсы?

\mathmodel

\textbf{1. Что будем оптимизировать?} Ответ: нужно максимизировать получаемую прибыль.

\bigskip

\textbf{2. Что существенно влияет на оптимизируемую характеристику?}

\begin{table}[h!]
	\centering
	\begin{tabular}{| c | c | c | c | c | c |} 
		\hline
		\textbf{Проект}             & $A$ & $B$ & $C$ & $D$ & $E$ \\\hline
		\textbf{Доход}              & 3   & 2   & 1   & 4   & 2 \\\hline
		\textbf{Начальные вложения} & 1.5 & 0   & 0.5 & 4   & 1 \\\hline
	\end{tabular}
\end{table}

\textit{Условия вложения в проекты}
\begin{enumerate}[nosep]
	\item нужно вложиться хотя бы в один проект;
	
	\item если вложились в $A$, то необходимо вложиться в $D$;
	
	\item если вложились в $B$ и $C$, то необходимо вложиться в $A$;
	
	\item если вложились в $B$, то необходимо вложиться в $C$ и $E$.
\end{enumerate}

\bigskip

\textit{Параметры}
\begin{itemize}[nosep]
	\item $c_i$ --- доход с проекта $i$;
	
	\item $d_i$ --- начальные вложения в проект $i$;
	
	\item $Q$ --- стартовый капитал;
	
	\item $C$ --- общий доход.
\end{itemize}

\bigskip

\textit{Переменные}
\begin{itemize}[nosep]
	\item $x_1, x_2, x_3, x_4, x_5$ --- булевы переменные, которые означают, будет ли вложение в соответствующий проект
	\[
	x_i = \begin{cases}
		1, & \text{будем вкладываться в $i$-ый проект}, \\
		0, & \text{иначе}.
	\end{cases}
	\]
\end{itemize}

\bigskip

\textbf{3. Математическая формулировка задачи}

Пусть $D$ --- сумма всех расходов, $C$ --- сумма всех доходов, $F$ --- прибыль, тогда
\[
D = \sum_{i=1}^5 x_i d_i, \quad C = \sum_{i=1}^5 x_i c_i,
\]
\[
F = C - D = \sum_{i=1}^5 x_i c_i - \sum_{i=1}^5 x_i d_i = \sum_{i=1}^5 x_i (c_i - d_i) \to \max_{(x_i)},
\]
\[
D \le Q \quad \Longleftrightarrow \quad \sum_{i=1}^5 x_i d_i \le Q.
\]

Последнее ограничение означает, что мы не можем вложить в проекты больше стартового капитала.

\bigskip

\textbf{Запись условий вложения в проекты}

\begin{enumerate}[nosep]
	\item Первое условие можно записать алгебраически через
	
	\[
	\sum_{i=1}^{5} x_i \ge 1
	\]
	
	\item Второе условие эквивалентно <<если $x_1 = 1$, то $x_4 = 1$>>, алгебраически это записывается как (\hyperref[fact:simple_conditions]{простые условия})
	\[
	1 - x_1 \ge 1 - x_4 \quad \Longleftrightarrow \quad x_1 \le x_4.
	\]
	
	\item Третье условие эквивалентно <<если $x_2 = 1$ и $x_3 = 1$, то $x_1 = 1$>>, алгебраически это записывается так (\hyperref[fact:complex_conditions]{сложные условия})
	\[
	(1 - x_2) + (1 - x_3) \ge 1 - x_1 \quad \Longleftrightarrow \quad 1 + x_1 \ge x_2 + x_3.
	\]
	
	\item Четвёртое условие эквивалентно <<если $x_2 = 1$, то $x_3 = 1$ и $x_5 = 1$>>, алгебраически это записывается так (\hyperref[fact:complex_conditions]{сложные условия})
	\[
	2 \cdot (1 - x_2) \ge 1 - x_3 + 1 - x_5 \quad \Longleftrightarrow \quad 2x_2 \le x_3 + x_5.
	\]
\end{enumerate}

\bigskip

\textbf{Итоговая модель}

\[
\sum_{i=1}^5 x_i (c_i - d_i) \to \max_{(x_i)}
\]

\[
\sum_{i=1}^5 x_i d_i \le Q
\]

\[
\begin{cases}
	\sum\limits_{i=1}^{5} x_i \ge 1, \\
	x_1 \le x_4, \\
	1 + x_1 \ge x_2 + x_3, \\
	2x_2 \le x_3 + x_5;
\end{cases}
\]

\[
x_i \in \{0, 1\}.
\]

\section{Задача о предприятии}

\problem[о предприятии]

Пусть есть предприятие, которое производит определённые виды продукции, затрачивая некоторые свои ресурсы. Для простоты будем считать, что у нас есть лишь один вид ресурсов, который можно использовать для производства продукции. Как получить наибольший доход?

\mathmodel

\textbf{1. Что будем оптимизировать?} Ответ: нужно максимизировать получаемый доход.

\bigskip

\textbf{2. Что существенно влияет на оптимизируемую характеристику?}

Пусть $i = 1 \dots n$ --- вид продукции, $I = \{1, 2, \dots, n\}$ --- список всех производимых товаров.

\bigskip

\textit{Параметры}

\begin{itemize}[nosep]
	\item $c_i$ --- доход продукции;
		
	\item $b_i$ --- расход ресурса на производство одной единицы продукции;
	
	\item $B$ --- запасы ресурса.
\end{itemize}

\bigskip

\textit{Переменные}
\begin{itemize}[nosep]
	\item $x_i$ --- количество единиц производимой продукции.
\end{itemize}

\bigskip

\textbf{3. Математическая формулировка задачи}

\[
C = \sum_{i \in I}c_i x_i \to \max_{(x_i),}
\]
\[
\sum_{i \in I}b_i x_i \le B,
\]
\[
\forall i \in I \quad x_i \ge 0.
\]

\textbf{Изменения в задаче}

Казалось бы, математическая модель составлена, однако тут приходит директор предприятия и говорит, что правительство определило два списка социально значимых товаров $I_1$, $I_2$, при этом нужно либо из первого списка производить не менее $a_1$ единиц продукции, либо из второго не менее $a_2$ единиц продукции. Оба условия можно записать следующим образом
\[
\sum_{i \in i_1} x_i \ge a_1, \quad \sum_{i \in I_2} x_i \ge a_2,
\]

Нам нужно, чтобы выполнялось хотя бы одно из них. Для этого введём булеву переменную
\[
y = \begin{cases}
	0, & \text{производим не менее $a_1$ из $I_1$}, \\
	1, & \text{производим не менее $a_2$ из $I_2$};
\end{cases}
\]

и запишем с её помощью требуемое условие (\hyperref[fact:alternative_conditions]{альтернативные условия})
\[
\sum_{i \in I_1} x_i \ge a_1 - Wy,
\]
\[
\sum_{i \in I_2} x_i \ge a_2 - W(1 - y).
\]

\textbf{Новые изменения в задаче}

Казалось бы, сейчас математическая модель предприятия окончательно составлена, но... К вам вновь приходит директор предприятия и говорит, что правительство издало приказ, по которому если производится достаточно продукции из обоих списков, то предприятие получает надбавку к финансированию.

Пусть $C_0$ --- надбавка за достаточное производство продукции из обоих списков социально значимых товаров. Введём две новые булевы переменные
\[
y_1 = \begin{cases}
	1, & \text{производим не менее $a_1$ из $I_1$}, \\
	0, & \text{иначе};
\end{cases}
\]
\[
y_2 = \begin{cases}
	1, & \text{производим не менее $a_2$ из $I_2$}, \\
	0, & \text{иначе}.
\end{cases}
\]

Изменим выражения для оптимизируемой характеристики и логических условий
\[
C = \sum_{i \in I}c_i x_i + C_0 y_1 y_2 \to \max_{(x_i), y_1, y_2}
\]
\[
\sum_{i \in I_1} x_i \ge a_1 - W(1 - y_1),
\]
\[
\sum_{i \in I_2} x_i \ge a_2 - W(1 - y_2).
\]

\textbf{Замена нелинейности}

В общем и целом теперь нас всё устраивает, кроме нелинейности в виде $y_1 y_2$. Произведём замену нелинейности с помощью \hyperref[fact:substituion_of_nonlinear]{замены нелинейностей} и запишем ограничения на новую булеву переменную
\[
z = y_1 y_2,
\]
\[
C = \sum_{i \in I}c_i x_i + C_0 z \to \max_{(x_i), z},
\]
\[
2z \le y_1 + y_2,
\]
\[
z + 1 \ge y_1 + y_2.
\]

\textbf{Итоговая модель}

\textit{Параметры}: $\{c_i\}$, $\{b_i\}$, $B$, $a_1$, $a_2$, $C_0$.

\textit{Переменные}: $\{x_i\}$, $y_1$, $y_2$, $z$.

\[
C = \sum_{i \in I}c_i x_i + C_0 z \to \max_{(x_i), z},
\]
\[
\sum_{i \in I}b_i x_i \le B,
\]
\[
\sum_{i \in I_1} x_i \ge a_1 - W(1 - y_1),
\]
\[
\sum_{i \in I_2} x_i \ge a_2 - W(1 - y_2),
\]
\[
2z \le y_1 + y_2, \quad z + 1 \ge y_1 + y_2,
\]
\[
y_1 \in \{0, 1\} \quad y_2 \in \{0, 1\}, \quad z \in \{0, 1\},
\]
\[
\forall i \in I \quad x_i \ge 0.
\]
