\chapter{Сетевые модели}

\section{Теория графов}

\definition

\definitionfont{Графом} называется пара множеств $G = (V, E)$, где

\begin{itemize}[nosep]
	\item $V$ --- \definitionfont{множество вершин},
	
	\item $E \subseteq V \times V$ --- \definitionfont{множество рёбер}. 
\end{itemize}

\begin{figure}[H]
	\centering	
	\begin{tikzpicture}[scale=1.5]
		\node[vertex, label=above:$v_1$] (v1) at (0,0) {};
		\node[vertex, label=left:$v_2$] (v2) at (-1,-1) {};
		\node[vertex, label=right:$v_3$] (v3) at (-1,-3) {};
		\node[vertex, label=above:$v_4$] (v4) at (0.5,-1.5) {};
		\node[vertex, label=right:$v_5$] (v5) at (1.5,-2.5) {};
	
		\draw (v1) -- node[left] {$e_1$} (v2);
		\draw (v2) -- node[left] {$e_3$} (v3);
		\draw (v2) -- node[above] {$e_2$} (v4);
		\draw (v4) -- node[above] {$e_4$} (v5);
	\end{tikzpicture}
	\caption{Пример графа}
\end{figure}

\definition

Вершины называются \definitionfont{смежными}, если они соединены ребром.

\definition

\definitionfont{Ориентированным графом} называется пара множеств $G = (V, E)$, где

\begin{itemize}[nosep]
	\item $V$ --- \definitionfont{множество вершин},
	
	\item $E \subseteq V \times V$ --- \definitionfont{множество дуг}.
\end{itemize}

Будем обозначать дугу от вершины $x$ до вершины $y$ как $\vec{xy}$.

\remark

В неориентированном графе если множество рёбер $E$ содержит ребро $(v_i, v_j)$, то оно также содержит ребро $(v_j, v_i)$. Гарантировать то же самое в ориентированном графе нельзя.

\remark

В дальнейшем для простоты будем считать, что

\begin{itemize}[nosep]
	\item $V = \{1, 2, \dots, n\}$,
	
	\item $E = \{1, 2, \dots, m\}$.
\end{itemize}

\definition

Пусть каждая дуга $j$ ориентированного графа имеет некоторый вес $c(j)$, тогда \definitionfont{табличным видом ориентированного графа} будем называть следующую таблицу

\begin{table}[H]
	\centering
	\begin{tabular}{ | c | c | c | c | } 
		\hline
		$j$ & $i(j)$ & $k(j)$ & $c(j)$ \\ \hline
		$1$ &&& \\ \hline
		$1$ &&& \\ \hline
		$\dots$ &&&\\\hline
		$m$ &&& \\ \hline
	\end{tabular}
\end{table}

\begin{itemize}[nosep]
	\item $j$ --- дуга,
	
	\item $i(j)$ --- начальная точка дуги,
	
	\item $k(j)$ --- конечная точка дуги,
	
	\item $c(j)$ --- вес дуги.
\end{itemize}

\begin{figure}[H]
	\centering	
	\begin{tikzpicture}[scale=1.5]
		\node[vertex, label=above:$i(j)$] (i) at (-1,-1) {};
		\node[vertex, label=left:$k(j)$] (k) at (1,1) {};
		
		\draw[->, edge] (i) -- node[left] {$j$} (k);
	\end{tikzpicture}
	\caption{Пример дуги $j$}
\end{figure}

\definition

\definitionfont{Путь} в графе от вершины $i_1$ до $i_{L+1}$ --- последовательность неповторяющихся вершин. Будем обозначать это как $P(i_1, \dots, i_{L+1})$ или $\{i_1, \dots, i_{L+1}\}$, где $(i_l, i_{l+1}) \in E$.

\remark

В рамках данного курса не будут рассматриваться пути, являющиеся циклами, то есть всегда $i_1 \neq i_k$.

\definition

\definitionfont{Сеть} --- ориентированных граф $G = (I, J)$, в котором

\begin{itemize}[nosep]
	\item есть 2 выделенные вершины $s$ (источник/вход) и $t$ (сток/выход);
	
	\item на множествах вершин и рёбер определён строгий порядок.
\end{itemize}

Вершины $s$ и $t$ определяется так: $s$ --- наименьшая вершина, в которую не входит дуга; а $t$ --- наибольшая вершина, из которой не выходит дуга.

\begin{figure}[H]
	\centering	
	\begin{tikzpicture}[scale=1.5]
		\node[vertex, label=left:\textcolor{blue}{$s$}, fill=blue] (s) at (-3,0) {};
		\node[vertex, label=right:\textcolor{blue}{$t$}, fill=blue] (t) at (1,0) {};
		\node[vertex] (v1) at (-1.6, -0.2) {};
		\node[vertex] (v2) at (-0.5, -0.6) {};
		\node[vertex] (v3) at (-0.9, 0.5) {};
		\node[vertex] (v4) at (-0.5, 1) {};
		
		\draw[->, edge] (s) -- node[left] {} (v1);
		\draw[->, edge] (v1) -- node[left] {} (v2);
		\draw[->, edge] (v1) -- node[left] {} (v3);
		\draw[->, edge] (v3) -- node[left] {} (t);
		\draw[->, edge] (v2) -- node[left] {} (t);
		\draw[->, edge] (v3) -- node[left] {} (v4);
		\draw[->, edge] (s) to[bend left=60] (t);
		\draw[->, edge] (s) to[bend right=60] (t);
	\end{tikzpicture}
	\caption{Пример сети}
\end{figure}

\begin{definition}
	Сеть называется \definitionfont{взвешенной}, если у каждой дуги $j$ есть некоторой вес $\tau_j$. Тогда такую сеть можно задать с помощью следующей таблицы
	
	\begin{table}[H]
		\centering
		\begin{tabular}{ | c | c | c | c | c |} 
			\hline
			$j$ & $i(j)$ & $k(j)$ & $c(j)$ & $\tau_j$ \\ \hline
			$1$ &&&& \\ \hline
			$1$ &&&& \\ \hline
			$\dots$ &&&& \\\hline
			$m$ &&&& \\ \hline
		\end{tabular}
	\end{table}
\end{definition}

\section{Сетевая модель проекта}

\definition

\definitionfont{Проект} --- совокупность работ для достижения определённой цели. \definitionfont{Модель проекта} определяется следующим образом:

\begin{itemize}[nosep]
	\item $J = \{1, \dots, n\}$ --- множество работ проекта,
	
	\item $\tau_{j} \ge 0$ --- длительность работы $j$;
	
	\item $C = \big\{(i, j) \; \big| \; i,j \in J\big\}$ --- частичный порядок, работа $j$ не может начаться раньше окончанию работы $i$.
\end{itemize}

Модель проекта можно записать с помощью следующей таблицы

\begin{table}[H]
	\centering
	\begin{tabular}{ | c | c | c | } 
		\hline
		$j$ & \text{следующие работы} & $\tau_j$ \\ \hline
		$j_0$ & $j_1$, $j_2$, $\dots$ & $\tau_{j_0}$ \\ \hline
		$\dots$ & $\dots$ & $\dots$ \\ \hline
	\end{tabular}
\end{table}

При этом, <<следующие работы>> для работы $j$ --- это работы, которые зависят от $j$, то есть которые не могут начаться раньше, чем завершится работа $j$.

\example\label{ex:office_project}

Рассмотрим следующую модель проекта:

\begin{itemize}[nosep]
	\item $J = \{1, 2, \dots, 9, 10\}$ --- множество работ проекта,
	
	\item $1$ --- выбрать место для офиса,
	
	\item $2$ --- создать финансовый и организационный план,
	
	\item $3$ --- определить обязанности персонала,
	
	\item $4$ --- разработать план офиса,
	
	\item $5$ --- ремонт помещений,
	
	\item $6$ --- отобрать кандидатов на увольнение,
	
	\item $7$ --- нанять новых служащих,
	
	\item $8$ --- назначить ключевых руководителей,
	
	\item $9$ --- распределить обязанности руководителей,
	
	\item $10$ --- обучить персонал;
\end{itemize}

\begin{table}[H]
	\centering
	\begin{tabular}{ | c | c | c | } 
		\hline
		$j$ & \text{следующие работы} & $\tau_j$ \\ \hline
		$1$ & $\{4\}$ & $3$ \\ \hline
		$2$ & $\{3, 9\}$ & $5$ \\ \hline
		$3$ & $\{4, 6\}$ & $3$ \\ \hline
		$4$ & $\{5\}$ & $5$ \\ \hline
		$5$ & $\{10\}$ & $10$ \\ \hline
		$6$ & $\{7, 8\}$ & $2$ \\ \hline
		$7$ & $\{10\}$ & $5$ \\ \hline
		$8$ & $\{10\}$ & $2$ \\ \hline
		$9$ & $\O$ & $5$ \\ \hline
		$10$ & $\O$ & $3$ \\ \hline
	\end{tabular}
\end{table}

\definition

\definitionfont{Сетевая модель проекта} --- наглядное представление работ в виде сети, в которой для каждой работы есть дуга, а отношение следования работ в проекте имеет место и в сети.

\definitionfont{Сетевая модель проекта} --- ориентированный взвешенный граф без циклов $G = (V, E)$ с выделенными вершинами $s$ и $t$, при этом каждой дуге $j = \vec{ik}$ приписан вес $\tau_j \ge 0$. Вершинами сети будут \definitionfont{события} (результаты выполнения работ), а дугами сами работы.

\textbf{Обозначение}: \circled{1} --- вершина с номером $1$, \squared{1} --- дуга с номером $1$.

\example

Изобразим сетевую модель проекта из \hyperref[ex:office_project]{примера}.

\begin{figure}[H]
	\centering	
	\begin{tikzpicture}[scale=2.2]
		\node[vertex, label=left:\textcolor{blue}{\circled{1}}, fill=blue] (1) at (-3,0) {};
		\node[vertex, label=right:\textcolor{blue}{\circled{9}}, fill=blue] (9) at (4,-1) {};
		\node[vertex, label={\circled{4}}] (4) at (-2.3,0.7) {};
		\node[vertex, label={\circled{2}}] (2) at (-2.3,-0.7) {};
		\node[vertex, label={\circled{5}}] (5) at (0,0.7) {};
		\node[vertex, label={\circled{8}}] (8) at (2.2,0.7) {};
		\node[vertex, label={\circled{3}}] (3) at (-0.8,-0.7) {};
		\node[vertex, label={\circled{6}}] (6) at (0.7,-0.7) {};
		\node[vertex] (7) at (2.2,-0.7) {};
		
		\node at (1.3,0.2) {\squared{7}};
		\node at (3.5, 0) {\squared{10}};
		\node at (2, -1.4) {\squared{9}};
		\node at (-1.4, 0.2) {\squared{11}};
		\node at (2.4, -0.6) {\circled{7}};
		\node at (2.45, 0.1) {\squared{12}};
		
		\draw[->, edge] (1) -- node[above left] {\squared{1}} (4);
		\draw[->, edge] (1) -- node[below left] {\squared{2}} (2);
		\draw[->, edge] (4) -- node[above] {\squared{4}} (5);
		\draw[->, edge] (5) -- node[above] {\squared{5}} (8);
		\draw[->, edge] (6) -- node[above] {} (8);
		\draw[->, edge] (6) -- node[above] {\squared{8}} (7);
		\draw[->, edge] (2) -- node[above] {\squared{3}} (3);
		\draw[->, edge] (3) -- node[above] {\squared{6}} (6);
		\draw[->, edge, dashed] (3) -- node {} (4);
		\draw[->, edge] (2) to[bend right=10] (9);
		\draw[->, edge] (8) to[bend left=10] (9);
		\draw[->, edge, dashed] (7) -- node[above] {} (8);
	\end{tikzpicture}
	\caption{Сеть для проекта из примера. Дуги $11$ и $12$ является фиктивными, то есть не соответствуют работам из модели проекта, они нужны для целостности сети.}
\end{figure}

\algorithm[построения сетевой модели]

\begin{enumerate}[nosep]
	\item[]
	
	\item Для каждой работы строим вершины и дуги.
	
	\item Обозначаем подчинённость, рисуя фиктивные дуги из конца текущей работы в следующую в соответствии с проектом.
	
	\item Добавляем начальную и конечную вершины $s$ и $t$.
	
	\item Убираем фиктивные дуги так, чтобы граф не перестал быть сетью, а также чтобы не появились параллельные дуги. Фиктивную дугу можно убрать, если она единственная, которая входит в вершину или выходит из вершины.
	
	\item Пронумеровываем вершины сети так, чтобы для каждой дуги $j$ выполнялось $i(j) < k(j)$.
\end{enumerate}

\subsection{Алгоритм Форда}

\definition

\definitionfont{Рангом} $r(x)$ вершины $x \in V$ называется число дуг в максимальном (по числу дуг) пути из начальной вершины $s$ в вершину $x$. \definitionfont{Рангом проекта} называется $r(t)$ (ранг стока/выхода/конечной вершины).

\algorithm[Форда]\label{alg:ford}

Алгоритм позволяет устанавливать ранги в сети так, чтобы нумерация вершин оказалось корректной с точки зрения условия $i(j) < k(j)$.

\textbf{Суть алгоритма} в следующем:

\begin{itemize}[nosep]
	\item в каждый момент времени будем для каждой вершины $k$ хранить её текущий ранг $r_k$ (изначально все $r_k = 0$);
	
	\item на каждой итерации будем последовательно просматривать все работы из множества $J$ от наименьшей к наибольшей и менять ранг вершин следующим образом: пусть текущая работа --- это $j = (i, k)$, тогда
	\[
	r_k = \max\{r_k, r_i+1\};
	\]
	
	\item алгоритм будет повторяться до тех пор, пока ранги вершин не перестанут меняться.
\end{itemize}

Для удобства ранги вершин можно записывать в таблицу

\begin{table}[H]
	\centering
	\begin{tabular}{ | c | c | c | c | c | } 
		\hline
		$k$ & $r_k^0$ & $r_k^1$ & $r_k^2$ & $\dots$ \\\hline
		$1$ & $0$ & $\dots$ & $\dots$ & $\dots$ \\\hline
		$2$ & $0$ & $\dots$ & $\dots$ & $\dots$ \\\hline
		$\dots$ & $\dots$ & $\dots$ & $\dots$ & $\dots$ \\\hline
		$n$ & $0$ & $\dots$ & $\dots$ & $\dots$ \\\hline
	\end{tabular}
\end{table}

где $r_k^i$ --- ранг вершины $k$ на шаге $i$.

\remark

Значение $r_k$ для одной и той же вершины $k$ может меняться несколько раз в течение одного шага.

\remark

Если сеть относительно небольшая, то быстрее будет <<вручную>> пронумеровать вершины, а не пользоваться алгоритмом Форда.

\fact

Алгоритм Форда создаёт корректную нумерацию вершин в сети.

\section{Практика построения сетей и нумерации вершин}

\subsection{Проект постройки дома}\label{proj:house_building_project}

\problem[Проект постройки дома]

Имеется проект постройки дома состоящий из восьми работ:

\bigskip

\begin{enumerate}[nosep]
	\item Очистка участка
	\item Земляные работы под фундамент
	\item Прокладка наружных коммуникаций
	\item Заливка фундамента и возведение стен
	\item Устройство крыши
	\item Прокладка внутренних коммуникаций
	\item Внутренние отделочные работы
	\item Обустройство территории
\end{enumerate}

\bigskip

Необходимо построить сетевую модель проекта и пронумеровать его вершины.

\bigskip

\textbf{Таблица работ}

\begin{table}[H]
	\centering
	\begin{tabular}{ | c | c | } 
		\hline
		j & следующие работы \\\hline
		$1$ & $\{2,3\}$ \\\hline
		$2$ & $\{4\}$ \\\hline
		$3$ & $\{8,6\}$ \\\hline
		$4$ & $\{5,6\}$ \\\hline
		$5$ & $\O$ \\\hline
		$6$ & $\{7\}$ \\\hline
		$7$ & $\O$ \\\hline
		$8$ & $\O$ \\\hline
	\end{tabular}
\end{table}

\bigskip

\textbf{Построение сетевой модели}

\fbox{Шаг 1} $\,$ Построение вершин и дуг для каждой работы

\begin{figure}[H]
	\centering	
	\begin{tikzpicture}[scale=2.2]
		\node[vertex] (1) at (-3,0) {};
		\node[vertex] (2) at (-2,0) {};
		\node[vertex] (3) at (-3,-1) {};
		\node[vertex] (4) at (-2,-1) {};
		\node[vertex] (5) at (-1,0) {};
		\node[vertex] (6) at (0,0) {};
		\node[vertex] (7) at (-2,-2) {};
		\node[vertex] (8) at (-1, -2) {};
		\node[vertex] (9) at (0,-2) {};
		\node[vertex] (10) at (1, -2) {};
		\node[vertex] (11) at (0,-1) {};
		\node[vertex] (12) at (1, -1) {};
		\node[vertex] (13) at (2,-1) {};
		\node[vertex] (14) at (3, -1) {};
		\node[vertex] (15) at (1,0) {};
		\node[vertex] (16) at (2,0) {};
		
		\draw[->, edge] (1) -- node[above] {\squared{1}} (2);
		\draw[->, edge] (3) -- node[above] {\squared{2}} (4);
		\draw[->, edge] (5) -- node[above] {\squared{3}} (6);
		\draw[->, edge] (7) -- node[above] {\squared{4}} (8);
		\draw[->, edge] (9) -- node[above] {\squared{5}} (10);
		\draw[->, edge] (11) -- node[above] {\squared{6}} (12);
		\draw[->, edge] (13) -- node[above] {\squared{7}} (14);
		\draw[->, edge] (15) -- node[above] {\squared{8}} (16);
		
	\end{tikzpicture}
\end{figure}

\fbox{Шаг 2} $\,$ Обозначение подчиненностей, добавление начальной и конечной вершин

\begin{figure}[H]
	\centering	
	\begin{tikzpicture}[scale=2.2]
		\node[vertex, label=left:\textcolor{blue}{$s$}, fill=blue] (1) at (-4,0) {};
		\node[vertex] (2) at (-3,0) {};
		\node[vertex] (3) at (-3,-1) {};
		\node[vertex] (4) at (-2,-1) {};
		\node[vertex] (5) at (-1,0) {};
		\node[vertex] (6) at (0,0) {};
		\node[vertex] (7) at (-2,-2) {};
		\node[vertex] (8) at (-1, -2) {};
		\node[vertex] (9) at (0,-2) {};
		\node[vertex] (10) at (1, -2) {};
		\node[vertex] (11) at (0,-1) {};
		\node[vertex] (12) at (1, -1) {};
		\node[vertex] (13) at (2,-1) {};
		\node[vertex] (14) at (3, -1) {};
		\node[vertex] (15) at (1,0) {};
		\node[vertex] (16) at (2,0) {};
		\node[vertex, label=right:\textcolor{blue}{$t$}, fill=blue] (17) at (3.5,-1) {};
		
		\draw[->, edge] (1) -- node[above] {\squared{1}} (2);
		\draw[->, edge] (3) -- node[above] {\squared{2}} (4);
		\draw[->, edge] (5) -- node[above] {\squared{3}} (6);
		\draw[->, edge] (7) -- node[above] {\squared{4}} (8);
		\draw[->, edge] (9) -- node[above] {\squared{5}} (10);
		\draw[->, edge] (11) -- node[above] {\squared{6}} (12);
		\draw[->, edge] (13) -- node[above] {\squared{7}} (14);
		\draw[->, edge] (15) -- node[above] {\squared{8}} (16);
		
		\draw[->, edge, dashed] (2) -- node {} (3);
		\draw[->, edge, dashed] (2) -- node {} (5);
		\draw[->, edge, dashed] (4) -- node {} (7);
		\draw[->, edge, dashed] (8) -- node {} (9);
		\draw[->, edge, dashed] (6) -- node {} (11);
		\draw[->, edge, dashed] (6) -- node {} (15);
		\draw[->, edge, dashed] (12) -- node {} (13);
		\draw[->, edge, dashed] (8) -- node {} (11);
		\draw[->, edge, dashed] (10) -- node {} (17);
		\draw[->, edge, dashed] (14) -- node {} (17);
		\draw[->, edge, dashed] (16) -- node {} (17);
		
	\end{tikzpicture}
\end{figure}

\fbox{Шаг 3} $\,$ Убираем фиктивные дуги

\begin{figure}[H]
	\centering	
	\begin{tikzpicture}[scale=2.2]
		\node[vertex, label=left:\textcolor{blue}{$s$}, fill=blue] (1) at (-4,0) {};
		\node[vertex] (2) at (-3,0) {};
		\node[vertex] (3) at (-2,-1) {};
		\node[vertex] (4) at (0,0) {};
		\node[vertex] (5) at (-1, -2) {};
		\node[vertex] (6) at (0,-1) {};
		\node[vertex] (7) at (2,-1) {};
		\node[vertex, label=right:\textcolor{blue}{$t$}, fill=blue] (8) at (3.5,-1) {};
		
		\draw[->, edge] (1) -- node[above] {\squared{1}} (2);
		\draw[->, edge] (2) -- node {} (3);
		\draw[->, edge] (3) -- node {} (5);
		\draw[->, edge] (2) -- node[above] {\squared{3}} (4);
		\draw[->, edge] (6) -- node[above] {\squared{6}} (7);
		\draw[->, edge] (7) -- node {} (8);
		\draw[->, edge] (4) -- node[above] {\squared{8}} (8);
		\draw[->, edge] (5) -- node[above] {\squared{5}} (8);
		
		\draw[->, edge, dashed] (4) -- node {} (6);
		\draw[->, edge, dashed] (5) -- node {} (6);
		
		\node at (2.2, -0.8) {\squared{7}};
		\node at (-2.25, -0.5) {\squared{2}};
		\node at (-1.5, -1.25) {\squared{4}};
		
	\end{tikzpicture}
\end{figure}

\fbox{Шаг 4} $\,$ Нумерация вершин

Пронумеруем вершины, стараясь сделать так, чтобы $i(j) < k(j)$

\begin{figure}[H]
	\centering	
	\begin{tikzpicture}[scale=2.2]
		\node[vertex, label={\circled{1}}, fill=blue] (1) at (-4,0) {};
		\node[vertex, label={\circled{2}}] (2) at (-3,0) {};
		\node[vertex, label={\circled{6}}] (3) at (-2,-1) {};
		\node[vertex, label={\circled{5}}] (4) at (0,0) {};
		\node[vertex, label={\circled{4}}] (5) at (-1, -2) {};
		\node[vertex, label=above right:\circled{3}] (6) at (0,-1) {};
		\node[vertex, label=\circled{8}] (7) at (2,-1) {};
		\node[vertex, label={\circled{7}}, fill=blue, fill=blue] (8) at (3.5,-1) {};
		
		\draw[->, edge] (1) -- node[above] {\squared{1}} (2);
		\draw[->, edge] (2) -- node {} (3);
		\draw[->, edge] (3) -- node {} (5);
		\draw[->, edge] (2) -- node[above] {\squared{3}} (4);
		\draw[->, edge] (6) -- node[above] {\squared{6}} (7);
		\draw[->, edge] (7) -- node {} (8);
		\draw[->, edge] (4) -- node[above] {\squared{8}} (8);
		\draw[->, edge] (5) -- node[above] {\squared{5}} (8);
		
		\draw[->, edge, dashed] (4) -- node {} (6);
		\draw[->, edge, dashed] (5) -- node {} (6);
		
		\node at (2.4, -0.85) {\squared{7}};
		\node at (-2.35, -0.4) {\squared{2}};
		\node at (-1.5, -1.25) {\squared{4}};
		\node at (0.2, -0.4) {\squared{10}};
		\node at (-0.6, -1.3) {\squared{9}};
		
	\end{tikzpicture}
\end{figure}

Составляем табличный вид данной сети:

\begin{table}[H]
	\centering
	\begin{tabular}{ | c | c | c | } 
		\hline
		$j$ & $i(j)$ & $k(j)$ \\ \hline
		
		$1$ & $1$ & $2$ \\ \hline
		$2$ & $2$ & $6$ \\ \hline
		$3$ & $2$ & $5$ \\ \hline
		$4$ & $6$ & $4$ \\ \hline
		$5$ & $4$ & $7$ \\ \hline
		$6$ & $3$ & $8$ \\ \hline
		$7$ & $8$ & $7$ \\ \hline
		$8$ & $5$ & $7$ \\ \hline
		$9$ & $4$ & $3$ \\ \hline
		$10$ & $5$ & $3$ \\ \hline
	\end{tabular}
\end{table}

Как видно для $j = \{4, 7, 9, 10\}$ не удалось корректно пронумеровать вершины. Воспользуемся \hyperref[alg:ford]{алгоритмом Форда} чтобы определить ранги вершин и корректно их пронумеровать.

Будем шаг за шагом заполнять таблицу, для итерации 0: все $r_k^0 = 0$

\begin{table}[H]
	\centering
	\begin{tabular}{ | c | c | } 
		\hline
		$k$ & $r_k^0$ \\ \hline
		$1$ & $0$ \\ \hline
		$2$ & $0$ \\ \hline
		$3$ & $0$ \\ \hline
		$4$ & $0$ \\ \hline
		$5$ & $0$ \\ \hline
		$6$ & $0$ \\ \hline
		$7$ & $0$ \\ \hline
		$8$ & $0$ \\ \hline
	\end{tabular}
\end{table}

\fbox{Итерация 1}

\bigskip

Следуя алгоритму вычисляем $r^1_k$ используя табличный вид сети:

\begin{enumerate}[nosep]
	\item $r_2^1 = max\{r_2^1, r_1^1 + 1\} = max\{0,0+1\} = 1$
	\item $r_6^1 = max\{r_6^1, r_2^1 + 1\} = max\{0,1+1\} = 2$
	\item $r_5^1 = max\{r_5^1, r_2^1 + 1\} = max\{0,1+1\} = 2$
	\item $r_4^1 = max\{r_4^1, r_6^1 + 1\} = max\{0,2+1\} = 3$
	\item $r_7^1 = max\{r_7^1, r_4^1 + 1\} = max\{0,3+1\} = 4$
	\item $r_8^1 = max\{r_8^1, r_3^1 + 1\} = max\{0,0+1\} = 1$
	\item $r_7^1 = max\{r_7^1, r_8^1 + 1\} = max\{4,1+1\} = 4$
	\item $r_7^1 = max\{r_7^1, r_5^1 + 1\} = max\{4,2+1\} = 4$
	\item $r_3^1 = max\{r_3^1, r_4^1 + 1\} = max\{0,3+1\} = 4$
	\item $r_3^1 = max\{r_3^1, r_5^1 + 1\} = max\{4,2+1\} = 4$
\end{enumerate}

Заносим результаты в таблицу:


\begin{table}[H]
	\centering
	\begin{tabular}{ | c | c | c | } 
		\hline
		$k$ & $r_k^0$ & $r_k^1$ \\ \hline
		$1$ & $0$ & $0$ \\ \hline
		$2$ & $0$ & $1$ \\ \hline
		$3$ & $0$ & $4$ \\ \hline
		$4$ & $0$ & $3$ \\ \hline
		$5$ & $0$ & $2$ \\ \hline
		$6$ & $0$ & $2$ \\ \hline
		$7$ & $0$ & $4$ \\ \hline
		$8$ & $0$ & $1$ \\ \hline
	\end{tabular}
\end{table}

\bigskip

\fbox{Итерация 2}

\bigskip

Следуя алгоритму вычисляем $r^2_k$ используя табличный вид сети:

\begin{enumerate}[nosep]
	\item $r_2^2 = max\{r_2^2, r_1^2 + 1\} = max\{0,0+1\} = 1$
	\item $r_6^2 = max\{r_6^2, r_2^2 + 1\} = max\{2,1+1\} = 2$
	\item $r_5^2 = max\{r_5^2, r_2^2 + 1\} = max\{2,1+1\} = 2$
	\item $r_4^2 = max\{r_4^2, r_6^2 + 1\} = max\{3,2+1\} = 3$
	\item $r_7^2 = max\{r_7^2, r_4^2 + 1\} = max\{4,3+1\} = 4$
	\item $r_8^2 = max\{r_8^2, r_3^2 + 1\} = max\{1,4+1\} = 5$
	\item $r_7^2 = max\{r_7^2, r_8^2 + 1\} = max\{4,5+1\} = 6$
	\item $r_7^2 = max\{r_7^2, r_5^2 + 1\} = max\{6,3+1\} = 6$
	\item $r_3^2 = max\{r_3^2, r_4^2 + 1\} = max\{4,3+1\} = 4$
	\item $r_3^2 = max\{r_3^2, r_5^2 + 1\} = max\{4,2+1\} = 4$
\end{enumerate}

Заносим результаты в таблицу:

\begin{table}[H]
	\centering
	\begin{tabular}{ | c | c | c | c |} 
		\hline
		$k$ & $r_k^0$ & $r_k^1$ & $r_k^2$ \\ \hline
		$1$ & $0$ & $0$ & $0$ \\ \hline
		$2$ & $0$ & $1$ & $1$ \\ \hline
		$3$ & $0$ & $4$ & $4$ \\ \hline
		$4$ & $0$ & $3$ & $3$ \\ \hline
		$5$ & $0$ & $2$ & $2$ \\ \hline
		$6$ & $0$ & $2$ & $2$ \\ \hline
		$7$ & $0$ & $4$ & $6$ \\ \hline
		$8$ & $0$ & $1$ & $5$ \\ \hline
	\end{tabular}
\end{table}

\fbox{Итерация 3}

\bigskip

Следуя алгоритму вычисляем $r^3_k$ используя табличный вид сети:

\begin{enumerate}[nosep]
	\item $r_2^3 = max\{r_2^3, r_1^3 + 1\} = max\{0,0+1\} = 1$
	\item $r_6^3 = max\{r_6^3, r_2^3 + 1\} = max\{2,1+1\} = 2$
	\item $r_5^3 = max\{r_5^3, r_2^3 + 1\} = max\{2,1+1\} = 2$
	\item $r_4^3 = max\{r_4^3, r_6^3 + 1\} = max\{3,2+1\} = 3$
	\item $r_7^3 = max\{r_7^3, r_4^3 + 1\} = max\{6,3+1\} = 6$
	\item $r_8^3 = max\{r_8^3, r_3^3 + 1\} = max\{5,4+1\} = 5$
	\item $r_7^3 = max\{r_7^3, r_8^3 + 1\} = max\{6,5+1\} = 6$
	\item $r_7^3 = max\{r_7^3, r_5^3 + 1\} = max\{6,2+1\} = 6$
	\item $r_3^3 = max\{r_3^3, r_4^3 + 1\} = max\{4,3+1\} = 4$
	\item $r_3^3 = max\{r_3^3, r_5^3 + 1\} = max\{4,2+1\} = 4$
\end{enumerate}

Заносим результаты в таблицу:

\begin{table}[H]
	\centering
	\begin{tabular}{ | c | c | c | c | c | } 
		\hline
		$k$ & $r_k^0$ & $r_k^1$ & $r_k^2$ & $r_k^3$ \\ \hline
		$1$ & $0$ & $0$ & $0$ & $0$ \\ \hline
		$2$ & $0$ & $1$ & $1$ & $1$ \\ \hline
		$3$ & $0$ & $4$ & $4$ & $4$ \\ \hline
		$4$ & $0$ & $3$ & $3$ & $3$ \\ \hline
		$5$ & $0$ & $2$ & $2$ & $2$ \\ \hline
		$6$ & $0$ & $2$ & $2$ & $2$ \\ \hline
		$7$ & $0$ & $4$ & $6$ & $6$ \\ \hline
		$8$ & $0$ & $1$ & $5$ & $5$ \\ \hline
	\end{tabular}
\end{table}

Как видно $r_k^2 = r_k^3 \quad \forall k$, значит алгоритм окончен: в $r^3_k$ содержатся ранги вершин. Теперь нужно согласно им пронумеровать вершины графа.

Для этого пользуемся следующим алгоритмом:

\begin{enumerate}[nosep]
	\item Присваиваем вершине с рангом 0 значение 1
	\item Смотрим имеются ли еще вершины с рангом последней пронумерованной вершины. Если есть - даем ей следующий номер вершины и повторяем этот шаг. Если нет - приступаем к шагу 3. Если же все вершины пронумерованы: завершаем алгоритм.
	\item Берем следующий ранг по возрастанию. Приступаем к шагу 2
\end{enumerate}

\bigskip

Удобно делать этот алгорим используя таблицу, которую мы получили: дописываем к ней колонку $H_k$ где будут номера вершин. Выполняя алгоритм нумерации получаем следующую таблицу.

\begin{table}[H]
	\centering
	\begin{tabular}{ | c | c | c | c | c | c |} 
		\hline
		$k$ & $r_k^0$ & $r_k^1$ & $r_k^2$ & $r_k^3$ & $H_k$ \\ \hline
		$1$ & $0$ & $0$ & $0$ & $0$ & $1$ \\ \hline
		$2$ & $0$ & $1$ & $1$ & $1$ & $2$ \\ \hline
		$3$ & $0$ & $4$ & $4$ & $4$ & $6$ \\ \hline
		$4$ & $0$ & $3$ & $3$ & $3$ & $5$ \\ \hline
		$5$ & $0$ & $2$ & $2$ & $2$ & $3$ \\ \hline
		$6$ & $0$ & $2$ & $2$ & $2$ & $4$ \\ \hline
		$7$ & $0$ & $4$ & $6$ & $6$ & $8$ \\ \hline
		$8$ & $0$ & $1$ & $5$ & $5$ & $7$ \\ \hline
	\end{tabular}
\end{table}

Используя эти данные перенумеруем вершины сети (по факту мы заменяем старый номер $k$ на $H_k$).

\begin{figure}[H]
	\centering	
	\begin{tikzpicture}[scale=2.2]
		\node[vertex, label={\circled{1}}, fill=blue] (1) at (-4,0) {};
		\node[vertex, label={\circled{2}}] (2) at (-3,0) {};
		\node[vertex, label={\circled{4}}] (3) at (-2,-1) {};
		\node[vertex, label={\circled{3}}] (4) at (0,0) {};
		\node[vertex, label={\circled{5}}] (5) at (-1, -2) {};
		\node[vertex, label=above right:\circled{6}] (6) at (0,-1) {};
		\node[vertex, label=\circled{7}] (7) at (2,-1) {};
		\node[vertex, label={\circled{8}}, fill=blue, fill=blue] (8) at (3.5,-1) {};
		
		\draw[->, edge] (1) -- node[above] {\squared{1}} (2);
		\draw[->, edge] (2) -- node {} (3);
		\draw[->, edge] (3) -- node {} (5);
		\draw[->, edge] (2) -- node[above] {\squared{3}} (4);
		\draw[->, edge] (6) -- node[above] {\squared{6}} (7);
		\draw[->, edge] (7) -- node {} (8);
		\draw[->, edge] (4) -- node[above] {\squared{8}} (8);
		\draw[->, edge] (5) -- node[above] {\squared{5}} (8);
		
		\draw[->, edge, dashed] (4) -- node {} (6);
		\draw[->, edge, dashed] (5) -- node {} (6);
		
		\node at (2.4, -0.85) {\squared{7}};
		\node at (-2.35, -0.4) {\squared{2}};
		\node at (-1.5, -1.25) {\squared{4}};
		\node at (0.2, -0.4) {\squared{10}};
		\node at (-0.6, -1.3) {\squared{9}};
		
	\end{tikzpicture}
\end{figure}

\subsection{Проект вычисления математического выражения}\label{proj:math_expr_calc_project}

\problem[Проект вычисления математического выражения]

Построить сетевую модель проекта вычисления значения выражения

\[\boxed{\frac{\sqrt{ab}(a + b)^2}{\sqrt{a + b}(a + ab)}}\]

для заданных величин $a, b$

\bigskip

\textbf{Таблица работ}

\begin{table}[H]
	\centering
	\begin{tabular}{ | c | c | c | } 
		\hline
		j & работа & следующие работы \\\hline
		$1$ & $ab$ & $\{4,5\}$ \\\hline
		$2$ & $a+b$ & $\{3, 6\}$ \\\hline
		$3$ & $(a+b)^2$ & $\{7\}$ \\\hline
		$4$ & $\sqrt{ab}$ & $\{7\}$ \\\hline
		$5$ & $(a+ab)$ & $\{8\}$ \\\hline
		$6$ & $\sqrt{a+b}$ & $\{8\}$ \\\hline
		$7$ & $\sqrt{ab}(a+b)^2$ & $\{9\}$ \\\hline
		$8$ & $\sqrt{a+b}(a+ab)$ & $\{9\}$ \\\hline
		$9$ & $\frac{\sqrt{ab}(a + b)^2}{\sqrt{a + b}(a + ab)}$ & $\O$ \\\hline
	\end{tabular}
\end{table}

\bigskip

\textbf{Построение сетевой модели}

\fbox{Шаги 1,2} $\,$ Построение вершин и дуг для каждой работы, обозначение подчиненностей, добавление начальной и конечной вершин.

\begin{figure}[H]
	\centering	
	\begin{tikzpicture}[scale=2.2]
		\node[vertex, label=left:\textcolor{blue}{$s$}, fill=blue] (s) at (-3.5,-2.5) {};
		\node[vertex] (1) at (-3,-1) {};
		\node[vertex] (2) at (-2,-1) {};
		\node[vertex] (3) at (-3,-4) {};
		\node[vertex] (4) at (-2,-4) {};
		\node[vertex] (5) at (-1,-3) {};
		\node[vertex] (6) at (0,-3) {};
		\node[vertex] (7) at (-1,0) {};
		\node[vertex] (8) at (0, 0) {};
		\node[vertex] (9) at (-1,-2) {};
		\node[vertex] (10) at (0, -2) {};
		\node[vertex] (11) at (-1,-5) {};
		\node[vertex] (12) at (0, -5) {};
		\node[vertex] (13) at (1,-1) {};
		\node[vertex] (14) at (2,-1) {};
		\node[vertex] (15) at (1,-4) {};
		\node[vertex] (16) at (2,-4) {};
		\node[vertex] (17) at (3,-2.5) {};
		\node[vertex, label=right:\textcolor{blue}{$t$}, fill=blue] (18) at (4,-2.5) {};
		
		
		\draw[->, edge] (1) -- node[above] {\squared{1}} (2);
		\draw[->, edge] (3) -- node[above] {\squared{2}} (4);
		\draw[->, edge] (5) -- node[above] {\squared{3}} (6);
		\draw[->, edge] (7) -- node[above] {\squared{4}} (8);
		\draw[->, edge] (9) -- node[above] {\squared{5}} (10);
		\draw[->, edge] (11) -- node[above] {\squared{6}} (12);
		\draw[->, edge] (13) -- node[above] {\squared{7}} (14);
		\draw[->, edge] (15) -- node[above] {\squared{8}} (16);
		\draw[->, edge] (17) -- node[above] {\squared{9}} (18);
		\draw[->, edge] (s) -- node {} (1);
		\draw[->, edge] (s) -- node {} (3);
		
		\draw[->, edge, dashed] (2) -- node {} (7);
		\draw[->, edge, dashed] (2) -- node {} (9);
		\draw[->, edge, dashed] (4) -- node {} (5);
		\draw[->, edge, dashed] (4) -- node {} (11);
		\draw[->, edge, dashed] (12) -- node {} (15);
		\draw[->, edge, dashed] (6) -- node {} (13);
		\draw[->, edge, dashed] (8) -- node {} (13);
		\draw[->, edge, dashed] (10) -- node {} (15);
		\draw[->, edge, dashed] (14) -- node {} (17);
		\draw[->, edge, dashed] (16) -- node {} (17);
		
	\end{tikzpicture}
\end{figure}

\fbox{Шаги 3,4} $\,$ Убираем фиктивные дуги и нумеруем вершины

\begin{figure}[H]
	\centering	
	\begin{tikzpicture}[scale=2.2]
		\node[vertex, label={\circled{1}}, fill=blue] (s) at (-3.5,-2.5) {};
		\node[vertex, label={\circled{2}}] (2) at (-2,-1) {};
		\node[vertex, label={\circled{3}}] (4) at (-2,-4) {};
		\node[vertex, label={\circled{5}}] (13) at (1,-1) {};
		\node[vertex, label={\circled{4}}] (15) at (1,-4) {};
		\node[vertex, label={\circled{6}}] (17) at (3,-2.5) {};
		\node[vertex, label={\circled{7}}, fill=blue] (18) at (4,-2.5) {};
		
		
		\draw[->, edge] (s) -- node[above=2mm] {\squared{1}} (2);
		\draw[->, edge] (s) -- node[above=2mm] {\squared{2}} (4);
		\draw[->, edge] (4) -- node[above=20mm, right=10mm] {\squared{3}} (13);
		\draw[->, edge] (2) -- node[above] {\squared{4}} (13);
		\draw[->, edge] (2) -- node[below=7mm, right=10mm] {\squared{5}} (15);
		\draw[->, edge] (4) -- node[above] {\squared{6}} (15);
		\draw[->, edge] (13) -- node[above=2mm] {\squared{7}} (17);
		\draw[->, edge] (15) -- node[above=2mm] {\squared{8}} (17);
		\draw[->, edge] (17) -- node[above] {\squared{9}} (18);
		
	\end{tikzpicture}
\end{figure}

Данная сеть строится и нумеруется просто, поэтому здесь нет нужды использовать алгоритм Форда.

\section{Поиск максимальных путей}\label{alg:max_length_path_searching}

\definition

\definitionfont{Длиной пути} $P(i_1, \dots, i_{L+1})$ называется
\[
\abs{P(i_1, \dots, i_{L+1})} = \sum_{l=1}^{L} \tau_{(i_l, i_{l+1})}
\]

\remark

Вместо суммы может стоять произведение, взятие максимума/минимума и так далее (всё зависит от конкретной задачи и сети).

\problem[поиска максимальных путей]

Есть некоторая сеть $G = (I, J)$, нужно найти все пути максимальной длины от первой до последней вершины ($1 \to n$).

\solution

\begin{figure}[H]
	\centering	
	\begin{tikzpicture}[scale=2]
		\node[vertex, fill=blue] (1) at (-3,0) {};
		\node[vertex, fill=blue] (11) at (2.2,0) {};
		\node[vertex] (2) at (-2, 0) {};
		\node[vertex, label=right:$\dots$] (3) at (-1.6, 0.6) {};
		\node[vertex, label=right:$\dots$] (4) at (-1.6, -0.6) {};
		\node[vertex] (5) at (-1, 0) {};
		\node[vertex, label=right:$\dots$] (6) at (-0.6, 0.6) {};
		\node[vertex] (7) at (-0.6, -0.6) {};
		\node[vertex] (8) at (0, 0) {};
		\node[vertex, label=right:$\dots$] (9) at (0.6, 0) {};
		\node[vertex, label=left:$\dots$] (10) at (1.6, 0) {};
		\draw[->, edge] (1) -- node[left] {} (2);
		\draw[->, edge] (2) -- node[left] {} (3);
		\draw[->, edge] (2) -- node[left] {} (4);
		\draw[->, edge] (2) -- node[left] {} (5);
		\draw[->, edge] (5) -- node[left] {} (6);
		\draw[->, edge] (5) -- node[left] {} (7);
		\draw[->, edge] (7) -- node[left] {} (8);
		\draw[->, edge] (8) -- node[left] {} (9);
		\draw[->, edge] (10) -- node[left] {} (11);
		\draw[->, edge] (1) to[bend left=60] (11);
		\draw[->, edge] (1) to[bend right=60] (11);
	\end{tikzpicture}
	\caption{Задача поиска максимальных путей в сети}
\end{figure}

Для решения задачи будем использовать динамическое программирование. Поскольку нам не задано количество шагов (длины путей зависят от конкретной сети), то будем использовать \hyperref[alg:unknown_step_process]{многошаговый процесс принятия решений}. Вспомним, что по определению сети на множестве вершин задан строгий частичный порядок --- это соответствует требования многошагового процесса (<<согласованность>> функции перехода).

\bigskip

\textbf{Исходные данные}

Формализуем задачу поиска максимальных путей в терминах динамического программирования:

\begin{itemize}[nosep]
	\item \underline{состояние}: $i \in I$ (текущая вершина);
	
	\item \underline{решение}: $j \in J$ (дуга, по которой мы пойдём);
	
	\item \underline{множество возможных состояний}: $P = I$;
	
	\item \underline{множество конечных состояний}: $\bar{P} = \{n\}$;
	
	\item \underline{начальное состояние}: $p_0 = 1$;
	
	\item \underline{множество возможных решений}: $Q = J$;
	
	\item \underline{множество допустимых решений}: $Q(i) = \{j \; \big| \; i(j) = i\}$ (все дуги, по которым можно пойти из текущей вершины);
	
	\item \underline{функция перехода}: $T(i, j) = k(j)$ (находимся в вершине $i$ и двигаемся по дуге $j$);
	
	\item \underline{функция дохода}: $g(i, j) = \tau_j$.
\end{itemize}

\bigskip

\textbf{База процесса}

Пусть $f(i)$ --- длина максимального пути из вершины $i$ в вершину $n$, $f(1)$ --- ответ на вопрос исходной задачи. Так как длину пути можно представить как сумму длин входящих в него путей, по \hyperref[alg:unknown_step_process]{теории} $\boxed{f(n) = 0}$.

\bigskip

\textbf{Переход}

Пусть мы находимся в вершине $i$, тогда нам нужно рассмотреть все дуги, начинающиеся в данной вершине.

\begin{figure}[H]
	\centering	
	\begin{tikzpicture}[scale=2]
		\node[vertex, label=left:$i$] (1) at (-3,0) {};
		\node[vertex] (2) at (-2.5,0) {};
		\node[vertex] (3) at (-2.6,0.4) {};
		\node[vertex] (4) at (-2.6,-0.4) {};
		
		\draw[->, edge] (1) -- node[left] {} (2);
		\draw[->, edge] (1) -- node[left] {} (3);
		\draw[->, edge] (1) -- node[left] {} (4);
	\end{tikzpicture}
	\caption{Пример дуг, начинающихся в вершине $i$}
\end{figure}

Таким образом
\[
\boxed{f(i) = \max_{j: \; i(j) = i} \Big\{\tau_j + f\big(k(j)\big)\Big\}}\tag{*}.
\]

Так как $k(j) > 1$, то значение $f\big(k(j)\big)$ нам известно.

\bigskip

\textbf{Выбор оптимальной стратегии}

Будем записывать все данные в следующую таблицу

\begin{table}[H]
	\centering
	\begin{tabular}{ | c | c | c |} 
		\hline
		$i$ & $f(i)$ & $j(i)$\\ 
		\hline
		$1$ &&\\\hline
		$2$ && \\\hline
		$\dots$ && \\\hline
		$n-1$ && \\\hline
		$n$ && \\\hline
	\end{tabular}
\end{table}

$j(i)$ --- дуга, на которой достигается максимум в (*).

\begin{figure}[H]
	\centering	
	\begin{tikzpicture}[scale=2]
		\node[vertex, fill=blue, label=left:{$i=1$}] (1) at (-3,0) {};
		\node[vertex, fill=blue, label=right:{$i=n$}] (11) at (2.2,0) {};
		\node[vertex] (2) at (-2, 0) {};
		\node[vertex, label=right:$\dots$] (3) at (-1.6, 0.6) {};
		\node[vertex, label=right:$\dots$] (4) at (-1.6, -0.6) {};
		\node[vertex] (5) at (-1, 0) {};
		\node[vertex, label=right:$\dots$] (6) at (-0.6, 0.6) {};
		\node[vertex] (7) at (-0.6, -0.6) {};
		\node[vertex] (8) at (0, 0) {};
		\node[vertex, label=right:$\dots$] (9) at (0.6, 0) {};
		\node[vertex, label=left:$\dots$] (10) at (1.6, 0) {};
		\draw[->, edge, thick, red] (1) -- node[above] {$j(1)$} (2);
		\draw[->, edge] (2) -- node[left] {} (3);
		\draw[->, edge] (2) -- node[left] {} (4);
		\draw[->, edge, thick, red] (2) -- node[above] {$j(2)$} (5);
		\draw[->, edge] (5) -- node[left] {} (6);
		\draw[->, edge, thick, red] (5) -- node[left] {$j(3)$} (7);
		\draw[->, edge, thick, red] (7) -- node[below right] {$j(4)$} (8);
		\draw[->, edge, thick, red] (8) -- node[above]{$j(5)$} (9);
		\draw[->, edge, thick, red] (10) -- node[above] {$j(n-1)$} (11);
		\draw[->, edge] (1) to[bend left=60] (11);
		\draw[->, edge] (1) to[bend right=60] (11);
	\end{tikzpicture}
	\caption{Пример оптимальной стратегии}
\end{figure}

\remark

По ходу процесса, мы узнаем максимальные пути из каждой вершины в $n$-ую. Однако на будущее хотелось бы знать про максимальные пути из $1$-ой вершины в любую. Для этого можно проделать тот же самый процесс, но идти не $1 \to n$, а $n \to 1$.

\bigskip

\textbf{База процесса}

Пусть $g(i)$ --- длина максимального пути из вершины $i$ в вершину $1$. По \hyperref[alg:unknown_step_process]{теории} $\boxed{g(1) = 0}$.

\bigskip

\textbf{Переход}

\[
\boxed{g(i) = \max_{j': \; k(j') = i} \Big\{\tau_{j'} + g\big(i(j')\big)\Big\}}.
\]

Так как $i(j') < i$, то значение $g\big(i(j')\big)$ нам известно.

\bigskip

\textbf{Выбор оптимальной стратегии}

Итоговая таблица имеет следующий вид

\begin{table}[H]
	\centering
	\begin{tabular}{ | c | c | c | c | c| } 
		\hline
		$i$ & $f(i)$ & $j(i)$ & $g(i)$ & $j'(i)$ \\ 
		\hline
		$1$ &&&&\\\hline
		$2$ &&&& \\\hline
		$\dots$ &&&& \\\hline
		$n-1$ &&&& \\\hline
		$n$ &&&& \\\hline
	\end{tabular}
\end{table}

\remark

$f(1) = g(n)$.

\remark

Пусть исходная сеть соответствовала некоторому проекту с набором работ $J$ и продолжительностями работ $\{\tau_j\}$. Проект будет реализован, если будут реализованы самые длинны пути $1 \to n$.

\notation

\begin{itemize}
	\item[]
	
	\item $T_i^{\text{р}}$ (\definitionfont{самое ранее время наступления события} $i$). Это равняется длине максимального пути из $1$-ой вершины в $i$-ую.
	
	\item $T = T_n^{\text{р}} = g(n) = f(1)$ (\definitionfont{критическое время проекта}).
	
	\item $T_i^{\text{П}} = T - f(i)$ (\definitionfont{наиболее позднее время наступления события} $i$).
\end{itemize}

\definition

Путь будем называть \definitionfont{критическим}, если его длина равняется $T$.

\definition

Работу будем называться \definitionfont{критической}, если дуга, соответствующая этой работе, входит в состав критического пути.

\remark

Если увеличить время критической работы, то время проекта увеличится.

\remark

Мы уже говорили, что если работа критическая, то увеличивать её время нежелательно, так как увеличится время всего проекта. А если работа не критическая, то можно ли увеличить её время? И если можно, то насколько?

\fact

Пусть $\Delta_j$ --- увеличение времени работы $j$, тогда максимальное значение $\Delta_j$, которое не приведёт к увеличению времени проекта, вычисляется по формуле
\[
\Delta_j^{\max} = T_{k(j)}^{\text{П}} - T_{i(j)}^{\text{р}} - \tau_j.
\]

\prooof

Заметим, что
\[
g\big(i(j)\big) + \tau_j + f\big(k(j)\big) \le T\tag{*}.
\]

Действительно, $g\big(i(j)\big)$ --- максимальное время выполнения всех работ до работы $j$, а $f\big(k(j)\big)$ --- после работы $j$, поэтому выражение слева уж никак не может быть больше времени всего проекта $T$.

Из (*) следует, что к левой части можно добавить $\Delta_j \ge 0$ так, чтобы неравенство всё ещё выполнялось. Максимальное $\Delta_j$, которое можно прибавить, это значение, при котором будет достигаться равенство, то есть

\[
g\big(i(j)\big) + \tau_j + \Delta_j^{\max} + f\big(k(j)\big) = T,
\]
\[
\Downarrow
\]
\begin{align*}
	\Delta_j^{\max} =& \; T - 	f\big(k(j)\big) - g\big(i(j)\big) - \tau_j \\
	=& \; T_{k(j)}^{\text{П}} - T_{i(j)}^{\text{р}} - \tau_j.
\end{align*}

\begin{figure}[H]
	\centering	
	\begin{tikzpicture}[scale=2]
		\node[vertex, fill=blue, label=left:{$i=1$}] (1) at (-3,0) {};
		\node[vertex] (2) at (-2.55,0.45) {};
		\node[vertex, label=left:{$\dots$}] (3) at (-2.45,0.55) {};
		\node[vertex, label=above:{$i(j)$}] (4) at (-2,1) {};
		\node[vertex, label=above:{$k(j)$}] (5) at (0,1) {};
		\node[vertex, label=right:{$\dots$}] (6) at (0.45,0.55) {};
		\node[vertex] (7) at (0.55,0.45) {};
		\node[vertex, fill=blue, label=right:{$i=n$}] (8) at (1,0) {};
		\node at (-1,1.3) {$\tau_j + \Delta_j$};
		\draw[->, edge] (1) to[bend left=15] (2);
		\draw[->, edge] (3) to[bend left=15] (4);
		\draw[->, edge] (4) to[bend left=15] (5);
		\draw[->, edge] (5) to[bend left=15] (6);
		\draw[->, edge] (7) to[bend left=15] (8);
		\draw [<->] (-3.5, 0.5) -- node[above left = -1mm] {$g\big(i(j)\big)$} (-2.5, 1.5);
		
		\draw [<->] (1.5, 0.5) -- node[above right = -1mm] {$f\big(k(j)\big)$} (0.5, 1.5);
	\end{tikzpicture}
	\caption{Увеличение времени работы в сети}
\end{figure}

\implication

\begin{itemize}[nosep]
	\item[]
	
	\item Если $\Delta_j > \Delta_j^{\max}$, то время проекта увеличится;
	
	\item Если $\Delta_j \le \Delta_j^{\max}$, то время проекта не изменится.
\end{itemize}

\section{Задача о максимальном потоке}

\definition

Пусть
\begin{itemize}[nosep]
	\item $G = (I, J)$ --- взвешенная сеть;
	
	\item $i = 1$ --- вход, $i = n$ --- выход;
	
	\item $b_{ik}$ --- вес дуги $\vec{ik}$;
\end{itemize}

тогда $x = (x_{ik})$ будем называть \definitionfont{потоком в сети} с величиной входного потока $v$, если выполняется следующие ограничения
\[
\forall l \in I \quad \sum_{k:\;(l, k) \in J} x_{lk} - \sum_{i: \; (i, l) \in J} x_{il} = \begin{cases}
	v, & l = 1, \\
	0, & l = 2 \dots n-1, \\
	-v, & l = n;
\end{cases}
\]

(закон сохранения энергии)

\[
\forall j = (i, k) \quad x_{ik} \le b_{ik}.
\]

\definition

\definitionfont{Величину потока $x$ в сети} будем обозначать $v(x)$.

\definition

Пару $(S, \bar{S})$ будем называть \definitionfont{разрезом сети}, если $S$ и $\bar{S}$ разбивают множество вершин сети, при этом обязательно $1 \in S$, $n \in \bar{S}$.

\definition

\definitionfont{Пропускной способностью разреза} будем называть
\[
B(S, \bar{S}) = \sum_{\underset{i \in S, k \in \bar{S}}{(i, k) \in J}} b_{ik}
\]

По сути, это сумма пропускным способностей всех дуг, идущих из $S$ в $\bar{S}$.

\fact[свойство разреза]

Какой бы ни был поток $x = (x_{ik})$ и разрез $(S, \bar{S})$
\[
v(x) \le B(S, \bar{S}).
\]

\prooof

Напишем следующие выражение
\[
Z = \sum_{l \in S} \bigg(\sum_{k:\;(l, k) \in J} x_{lk} - \sum_{i: \; (i, l) \in J} x_{il}\bigg).
\]

Вспомним, что по определению разреза $1 \in S$, а значит $Z = v + \dots$. Также вспомним определение потока и условия оттуда, из них следует, что внутреннее выражение для всех $l \neq 1$ равняется $0$. Из всего этого следует, что
\[
Z = \underbrace{0}_{v = 1} + \underbrace{0 + \dots + 0}_{1 \neq l \in S} = v
\]

Посмотрим на $Z$ с другой стороны. Рассмотрим возможные случаи взаимного расположения вершин $i$ и $k$ для вершины $l$

\begin{itemize}[nosep]
	\item $i \in S, k \in S$;
	
	\item $i \in S, k \in \bar{S}$;
	
	\item $k \in S, k \in S$.
\end{itemize}

\begin{align*}
	Z =& \; \sum_{l \in S} \bigg(\sum_{k:\;(l, k) \in J} x_{lk} - \sum_{i: \; (i, l) \in J} x_{il}\bigg) \\
	=& \;  \sum_{\underset{i \in S, k \in \bar{S}}{(i, k) \in J}} x_{ik} - \sum_{\underset{i \in \bar{S}, k \in S}{(i, k) \in J}} x_{ik} \\
	\le& \; \sum_{\underset{i \in S, k \in \bar{S}}{(i, k) \in J}} x_{ik} \\
	\le& \sum_{\underset{i \in S, k \in \bar{S}}{(i, k) \in J}} b_{ik} \\
	=& \; B(S, \bar{S}).
\end{align*}

Для перехода $\sum_{\dots}x_{ik} \le \sum_{\dots} b_{ik}$ использовалось условие из определения потока в сети.

Таким образом $Z = v$ и $Z \le B(S, \bar{S})$, значит $v \le B(S, \bar{S})$.

TODO: есть путаница, где-то просто $v$, а где-то $v(x)$.

\fact[критерий максимальности потока]

\bigskip
\section{Практика поиска максимального пути в сети}
\subsection{Проект вычисления математического выражения}

Добавим в \hyperref[proj:math_expr_calc_project]{проект вычисления математического выражения} данные о единицах времени, необходимых на вычисление математических операций:

\begin{table}[H]
	\centering
	\begin{tabular}{ | c | c | } 
		\hline
		Операция & Количество единиц времени \\ \hline
		Сложение & $1$ \\ \hline
		Умножение & $2$ \\ \hline
		Деление & $4$ \\ \hline
		Извлечение корня & $6$ \\ \hline
	\end{tabular}
\end{table}

Необходимо найти за какое наименьшее время может быть вычислено выражение, если число вычислителей не ограничено.

\bigskip

\textbf{Постройка взвешенной сети}

Мы уже построили сеть данного проекта, можно её дополнить, добавив веса ребрам исходя из таблицы работ и таблицы с данными о затрачиваемых единиц времени.

\begin{figure}[H]
	\centering	
	\begin{tikzpicture}[scale=2.2]
		\node[vertex, label={\circled{1}}, fill=blue] (s) at (-3.5,-2.5) {};
		\node[vertex, label={\circled{2}}] (2) at (-2,-1) {};
		\node[vertex, label={\circled{3}}] (4) at (-2,-4) {};
		\node[vertex, label={\circled{5}}] (13) at (1,-1) {};
		\node[vertex, label={\circled{4}}] (15) at (1,-4) {};
		\node[vertex, label={\circled{6}}] (17) at (3,-2.5) {};
		\node[vertex, label={\circled{7}}, fill=blue] (18) at (4,-2.5) {};
		
		
		\draw[->, edge] (s) -- node[above=2mm] {\squared{1}} node[below] {2} (2);
		\draw[->, edge] (s) -- node[above=2mm] {\squared{2}} node[below] {1} (4);
		\draw[->, edge] (4) -- node[above=20mm, right=10mm] {\squared{3}} node[above=10mm, right=12mm] {2} (13);
		\draw[->, edge] (2) -- node[above] {\squared{4}} node[below] {6} (13);
		\draw[->, edge] (2) -- node[below=7mm, right=10mm] {\squared{5}} node[below=16mm, right=10mm] {1} (15);
		\draw[->, edge] (4) -- node[above] {\squared{6}} node[below] {6} (15);
		\draw[->, edge] (13) -- node[above=2mm] {\squared{7}} node[below] {2} (17);
		\draw[->, edge] (15) -- node[above=2mm] {\squared{8}} node[below] {2} (17);
		\draw[->, edge] (17) -- node[above] {\squared{9}} node[below] {4} (18);
		
	\end{tikzpicture}
\end{figure}

\bigskip

\textbf{Решение задачи}

По факту решение данной задачи это длина пути наибольшей длины из вершины 1 в вершину 7: $\textbf{критическое время}$. Решение подобных задач было разобрано в \hyperref[alg:max_length_path_searching]{главе 3.4: Поиск максимальных путей}.

$f(i)$ --- длина максимального пути из вершины $i$ в вершину $n$

\[\boxed{f(i) = \max_{j: \; i(j) = i} \Big\{\tau_j + f\big(k(j)\big)\Big\}}\]

Во время решения будем заполнять следующую таблицу:

\begin{table}[H]
	\centering
	\begin{tabular}{ | c | c | c | } 
		\hline
		$i$ & $f(i)$ & $j(i)$ \\ \hline
		$1$ & & \\ \hline
		$2$ & & \\ \hline
		$3$ & & \\ \hline
		$4$ & & \\ \hline
		$5$ & & \\ \hline
		$6$ & & \\ \hline
		$7$ & & \\ \hline
	\end{tabular}
\end{table}

где $j(i)$ --- номер дуги, с которой начинается наибольший путь из $i$ в $n$

Будем заполнять таблицу начиная с $f(7)$ и проходясь по индексам вниз вплоть до $f(1)$, где находится наше решение.
Для этого будем вычислять $f(i)$ выписывая в столбец разные $j$, считая для них $\tau_j + f(k(j))$ и обводя в кружок то значение, на котором достигается максимум.

\bigskip

\begin{enumerate}[nosep]
	\item[\fbox{Шаг 1}] На первом шаге $n = 7$
	
	$f(7)$ --- длина наибольшего пути из вершины 7 в вершину 7, очевидно что здесь нет никакого пути, значит $f(7) = 0$, а $j(7)$ не определено.
	
	Запишем результат в таблицу:
	
	\begin{table}[H]
		\centering
		\begin{tabular}{ | c | c | c | } 
			\hline
			$i$ & $f(i)$ & $j(i)$ \\ \hline
			$1$ & & \\ \hline
			$2$ & & \\ \hline
			$3$ & & \\ \hline
			$4$ & & \\ \hline
			$5$ & & \\ \hline
			$6$ & & \\ \hline
			$7$ & $0$ & $-$ \\ \hline
		\end{tabular}
	\end{table}
	
	\item[\fbox{Шаг 2}] $n = 6$
	
	\[
	f(6) = \begin{array}{c|l}
		9 & \tau_9 + f(7) = 4 + 0 = \circled{4} \\
	\end{array}
	\]
	
	Запишем результат в таблицу:
	
	\begin{table}[H]
		\centering
		\begin{tabular}{ | c | c | c | } 
			\hline
			$i$ & $f(i)$ & $j(i)$ \\ \hline
			$1$ & & \\ \hline
			$2$ & & \\ \hline
			$3$ & & \\ \hline
			$4$ & & \\ \hline
			$5$ & & \\ \hline
			$6$ & $4$ & $9$ \\ \hline
			$7$ & $0$ & $-$ \\ \hline
		\end{tabular}
	\end{table}
	
	\item[\fbox{Шаг 3}] $n = 5$
	
	\[
	f(5) = \begin{array}{c|l}
		7 & \tau_9 + f(6) = 2 + 4 = \circled{6} \\
	\end{array}
	\]
	
	Запишем результат в таблицу:
	
	\begin{table}[H]
		\centering
		\begin{tabular}{ | c | c | c | } 
			\hline
			$i$ & $f(i)$ & $j(i)$ \\ \hline
			$1$ & & \\ \hline
			$2$ & & \\ \hline
			$3$ & & \\ \hline
			$4$ & & \\ \hline
			$5$ & $6$ & $7$ \\ \hline
			$6$ & $4$ & $9$ \\ \hline
			$7$ & $0$ & $-$ \\ \hline
		\end{tabular}
	\end{table}
	
	\item[\fbox{Шаг 4}] $n = 4$
	
	\[
	f(4) = \begin{array}{c|l}
		8 & \tau_8 + f(6) = 2 + 4 = \circled{6} \\
	\end{array}
	\]
	
	Запишем результат в таблицу:
	
	\begin{table}[H]
		\centering
		\begin{tabular}{ | c | c | c | } 
			\hline
			$i$ & $f(i)$ & $j(i)$ \\ \hline
			$1$ & & \\ \hline
			$2$ & & \\ \hline
			$3$ & & \\ \hline
			$5$ & $6$ & $8$ \\ \hline
			$5$ & $6$ & $7$ \\ \hline
			$6$ & $4$ & $9$ \\ \hline
			$7$ & $0$ & $-$ \\ \hline
		\end{tabular}
	\end{table}
	
	\item[\fbox{Шаг 5}] $n = 3$
	
	\[
	f(3) = \begin{array}{c|l}
		3 & \tau_3 + f(5) = 2 + 6 = 8 \\
		6 & \tau_6 + f(4) = 6 + 6 = \circled{12} \\
	\end{array}
	\]
	
	Запишем результат в таблицу:
	
	\begin{table}[H]
		\centering
		\begin{tabular}{ | c | c | c | } 
			\hline
			$i$ & $f(i)$ & $j(i)$ \\ \hline
			$1$ & & \\ \hline
			$2$ & & \\ \hline
			$3$ & $12$ & $6$ \\ \hline
			$5$ & $6$ & $8$ \\ \hline
			$5$ & $6$ & $7$ \\ \hline
			$6$ & $4$ & $9$ \\ \hline
			$7$ & $0$ & $-$ \\ \hline
		\end{tabular}
	\end{table}
	
	\item[\fbox{Шаг 6}] $n = 2$
	
	\[
	f(2) = \begin{array}{c|l}
		4 & \tau_4 + f(5) = 6 + 6 = \circled{12} \\
		5 & \tau_5 + f(4) = 1 + 6 = 7 \\
	\end{array}
	\]
	
	Запишем результат в таблицу:
	
	\begin{table}[H]
		\centering
		\begin{tabular}{ | c | c | c | } 
			\hline
			$i$ & $f(i)$ & $j(i)$ \\ \hline
			$1$ & & \\ \hline
			$2$ & $12$ & $4$ \\ \hline
			$3$ & $12$ & $6$ \\ \hline
			$5$ & $6$ & $8$ \\ \hline
			$5$ & $6$ & $7$ \\ \hline
			$6$ & $4$ & $9$ \\ \hline
			$7$ & $0$ & $-$ \\ \hline
		\end{tabular}
	\end{table}
	
	\item[\fbox{Шаг 7}] $n = 1$
	
	\[
	f(1) = \begin{array}{c|l}
		1 & \tau_1 + f(2) = 2 + 12 = \circled{14} \\
		2 & \tau_2 + f(3) = 1 + 12 = 13 \\
	\end{array}
	\]
	
	Запишем результат в таблицу:
	
	\begin{table}[H]
		\centering
		\begin{tabular}{ | c | c | c | } 
			\hline
			$i$ & $f(i)$ & $j(i)$ \\ \hline
			$1$ & $14$ & $1$ \\ \hline
			$2$ & $12$ & $4$ \\ \hline
			$3$ & $12$ & $6$ \\ \hline
			$5$ & $6$ & $8$ \\ \hline
			$5$ & $6$ & $7$ \\ \hline
			$6$ & $4$ & $9$ \\ \hline
			$7$ & $0$ & $-$ \\ \hline
		\end{tabular}
	\end{table}
	
\end{enumerate}

\bigskip

\textbf{Определенние пути}

Мы заполнили таблицу, длина длиннейшего пути оказалась равна $f(1) = 14$. Теперь по таблице вычислим этот путь.

Для этого рассматриваем столбец $j(i)$ --- как помним это дуги, с которых начинается максимальный путь для $f(i)$.

Путь начинается в $j(1)$. По таблице видно, что $j(1) = 1$, а дуга с номером 2 ведет в вершину 2, смотрим для неё $j(2)$ и идём так по вершинам, пока не достигнем вершины с номером 7. Графически этот путь выглядит так:

\begin{figure}[H]
	\centering	
	\begin{tikzpicture}[
		vertex/.style={circle, draw, minimum size=1cm, font=\large},
		scale=2.2
		]
		\node[vertex] (1) at (0, 0) {1};
		\node[vertex] (2) at (1, 0) {2};
		\node[vertex] (3) at (2, 0) {5};
		\node[vertex] (4) at (3, 0) {6};
		\node[vertex] (5) at (4, 0) {7};
		
		\draw[->, edge] (1) -- node[above] {\squared{1}} node[below] {2} (2);
		\draw[->, edge] (2) -- node[above] {\squared{4}} node[below] {6} (3);
		\draw[->, edge] (3) -- node[above] {\squared{7}} node[below] {2} (4);
		\draw[->, edge] (4) -- node[above] {\squared{9}} node[below] {4} (5);
		
	\end{tikzpicture}
	\caption{Путь максимальной длины из вершины 1 в вершину 7. Сверху дуг --- номер дуг в сети, снизу --- вес}
\end{figure}

Как видно этот путь не содержит вершины $3, 4$.

\bigskip

\textbf{Вычисление резерва времени выполнения работы}

Вычислим резерв времени выполнения работы 3 (на сколько максимально можно увеличить время этой работы, чтобы общее время не увеличилось).

\[
\Delta^{max}_3 = T - f(k(3)) - g(i(3)) - \tau_3 = 14 - f(5) - g(3) - 2
\]

$g(3)$ --- это максимальная длина пути от вершины 1 до вершины 3, по графу сразу видно что $g(3) = 1$.

$T$ --- это критическое время.

\bigskip

Тогда:

\[
\Delta^{max}_3 = 14 - f(5) - g(3) -2 = 12 - 6 - 1 = 5
\]

\bigskip

\subsection{Проект постройки дома}

Добавим в \hyperref[proj:house_building_project]{проект постройки дома} данные о единицах времени, необходимых на выполнение работ. Сделаем это сразу в табличный вид сети, добавив столбец с весами дуг:

\begin{table}[H]
	\centering
	\begin{tabular}{ | c | c | c | c |} 
		\hline
		$j$ & $i(j)$ & $k(j)$ & $\tau(j)$ \\ \hline
		
		$1$ & $1$ & $2$ & $0.5$ \\ \hline
		$2$ & $2$ & $4$ & $0.5$ \\ \hline
		$3$ & $2$ & $3$ & $1$ \\ \hline
		$4$ & $4$ & $5$ & $4$ \\ \hline
		$5$ & $5$ & $8$ & $1$ \\ \hline
		$6$ & $6$ & $7$ & $2$ \\ \hline
		$7$ & $7$ & $8$ & $2$ \\ \hline
		$8$ & $3$ & $8$ & $3$ \\ \hline
		$9$ & $5$ & $6$ & $0$ \\ \hline
		$10$ & $3$ & $6$ & $0$ \\ \hline
	\end{tabular}
\end{table}

Используя данную таблицу так же можно вычислить критическое время проекта.

Решение здесь фактически ничем не отличается от решения предыдущей задачи, поэтому сразу приступим к заполнению таблицы.

\begin{enumerate}[nosep]
	\item[\fbox{Шаг 1}] На первом шаге $n = 8$
	
	$f(8) = 0$, $j(8)$ не определено.
	
	Запишем результат в таблицу:
	
	\begin{table}[H]
		\centering
		\begin{tabular}{ | c | c | c | } 
			\hline
			$i$ & $f(i)$ & $j(i)$ \\ \hline
			$1$ & & \\ \hline
			$2$ & & \\ \hline
			$3$ & & \\ \hline
			$4$ & & \\ \hline
			$5$ & & \\ \hline
			$6$ & & \\ \hline
			$7$ & & \\ \hline
			$8$ & $0$ & $-$ \\ \hline
		\end{tabular}
	\end{table}
	
	\item[\fbox{Шаг 2}] $n = 7$
	
	\[
	f(7) = \begin{array}{c|l}
		7 & \tau_7 + f(8) = 2 + 0 = \circled{2} \\
	\end{array}
	\]
	
	Запишем результат в таблицу:
	
	\begin{table}[H]
		\centering
		\begin{tabular}{ | c | c | c | } 
			\hline
			$i$ & $f(i)$ & $j(i)$ \\ \hline
			$1$ & & \\ \hline
			$2$ & & \\ \hline
			$3$ & & \\ \hline
			$4$ & & \\ \hline
			$5$ & & \\ \hline
			$6$ & & \\ \hline
			$7$ & $2$ & $7$ \\ \hline
			$8$ & $0$ & $-$ \\ \hline
		\end{tabular}
	\end{table}
	
	\item[\fbox{Шаг 3}] $n = 6$
	
	\[
	f(6) = \begin{array}{c|l}
		6 & \tau_6 + f(7) = 2 + 2 = \circled{4} \\
	\end{array}
	\]
	
	Запишем результат в таблицу:
	
	\begin{table}[H]
		\centering
		\begin{tabular}{ | c | c | c | } 
			\hline
			$i$ & $f(i)$ & $j(i)$ \\ \hline
			$1$ & & \\ \hline
			$2$ & & \\ \hline
			$3$ & & \\ \hline
			$4$ & & \\ \hline
			$5$ & & \\ \hline
			$6$ & $4$ & $6$ \\ \hline
			$7$ & $2$ & $7$ \\ \hline
			$8$ & $0$ & $-$ \\ \hline
		\end{tabular}
	\end{table}
	
	\item[\fbox{Шаг 4}] $n = 5$
	
	\[
	f(5) = \begin{array}{c|l}
		5 & \tau_5 + f(8) = 1 + 0 = 1 \\
		9 & \tau_9 + f(6) = 0 + 4 = \circled{4} \\
	\end{array}
	\]
	
	Запишем результат в таблицу:
	
	\begin{table}[H]
		\centering
		\begin{tabular}{ | c | c | c | } 
			\hline
			$i$ & $f(i)$ & $j(i)$ \\ \hline
			$1$ & & \\ \hline
			$2$ & & \\ \hline
			$3$ & & \\ \hline
			$4$ & & \\ \hline
			$5$ & $4$ & $9$ \\ \hline
			$6$ & $4$ & $6$ \\ \hline
			$7$ & $2$ & $7$ \\ \hline
			$8$ & $0$ & $-$ \\ \hline
		\end{tabular}
	\end{table}
	
	\item[\fbox{Шаг 5}] $n = 4$
	
	\[
	f(4) = \begin{array}{c|l}
		4 & \tau_4 + f(5) = 4 + 4 = \circled{8} \\
	\end{array}
	\]
	
	Запишем результат в таблицу:
	
	\begin{table}[H]
		\centering
		\begin{tabular}{ | c | c | c | } 
			\hline
			$i$ & $f(i)$ & $j(i)$ \\ \hline
			$1$ & & \\ \hline
			$2$ & & \\ \hline
			$3$ & & \\ \hline
			$4$ & $8$ & $4$ \\ \hline
			$5$ & $4$ & $9$ \\ \hline
			$6$ & $4$ & $6$ \\ \hline
			$7$ & $2$ & $7$ \\ \hline
			$8$ & $0$ & $-$ \\ \hline
		\end{tabular}
	\end{table}
	
	\item[\fbox{Шаг 6}] $n = 3$
	
	\[
	f(3) = \begin{array}{c|l}
		8 & \tau_8 + f(8) = 3 + 0 = 3 \\
		10 & \tau_10 + f(6) = 0 + 4 = \circled{4} \\
	\end{array}
	\]
	
	Запишем результат в таблицу:
	
	\begin{table}[H]
		\centering
		\begin{tabular}{ | c | c | c | } 
			\hline
			$i$ & $f(i)$ & $j(i)$ \\ \hline
			$1$ & & \\ \hline
			$2$ & & \\ \hline
			$3$ & $4$ & $10$ \\ \hline
			$4$ & $8$ & $4$ \\ \hline
			$5$ & $4$ & $9$ \\ \hline
			$6$ & $4$ & $6$ \\ \hline
			$7$ & $2$ & $7$ \\ \hline
			$8$ & $0$ & $-$ \\ \hline
		\end{tabular}
	\end{table}
	
	\item[\fbox{Шаг 7}] $n = 2$
	
	\[
	f(2) = \begin{array}{c|l}
		2 & \tau_2 + f(4) = 0.5 + 8 = \circled{8.5} \\
		3 & \tau_3 + f(3) = 1 + 4 = 5 \\
	\end{array}
	\]
	
	Запишем результат в таблицу:
	
	\begin{table}[H]
		\centering
		\begin{tabular}{ | c | c | c | } 
			\hline
			$i$ & $f(i)$ & $j(i)$ \\ \hline
			$1$ & & \\ \hline
			$2$ & $8.5$ & $2$ \\ \hline
			$3$ & $4$ & $10$ \\ \hline
			$4$ & $8$ & $4$ \\ \hline
			$5$ & $4$ & $9$ \\ \hline
			$6$ & $4$ & $6$ \\ \hline
			$7$ & $2$ & $7$ \\ \hline
			$8$ & $0$ & $-$ \\ \hline
		\end{tabular}
	\end{table}
	
	\item[\fbox{Шаг 8}] $n = 1$
	
	\[
	f(1) = \begin{array}{c|l}
		1 & \tau_1 + f(2) = 0.5 + 8.5 = \circled{9} \\
	\end{array}
	\]
	
	Запишем результат в таблицу:
	
	\begin{table}[H]
		\centering
		\begin{tabular}{ | c | c | c | } 
			\hline
			$i$ & $f(i)$ & $j(i)$ \\ \hline
			$1$ & $9$ & $1$ \\ \hline
			$2$ & $8.5$ & $2$ \\ \hline
			$3$ & $4$ & $10$ \\ \hline
			$4$ & $8$ & $4$ \\ \hline
			$5$ & $4$ & $9$ \\ \hline
			$6$ & $4$ & $6$ \\ \hline
			$7$ & $2$ & $7$ \\ \hline
			$8$ & $0$ & $-$ \\ \hline
		\end{tabular}
	\end{table}
	
\end{enumerate}

\bigskip

\textbf{Определенние пути}

\begin{figure}[H]
	\centering	
	\begin{tikzpicture}[
		vertex/.style={circle, draw, minimum size=1cm, font=\large},
		scale=2.2
		]
		\node[vertex] (1) at (0, 0) {1};
		\node[vertex] (2) at (1, 0) {2};
		\node[vertex] (3) at (2, 0) {4};
		\node[vertex] (4) at (3, 0) {5};
		\node[vertex] (5) at (4, 0) {6};
		\node[vertex] (6) at (5, 0) {7};
		\node[vertex] (7) at (6, 0) {8};
		
		\draw[->, edge] (1) -- node[above] {\squared{1}} node[below] {0.5} (2);
		\draw[->, edge] (2) -- node[above] {\squared{2}} node[below] {0.5} (3);
		\draw[->, edge] (3) -- node[above] {\squared{4}} node[below] {4} (4);
		\draw[->, edge] (4) -- node[above] {\squared{9}} node[below] {0} (5);
		\draw[->, edge] (5) -- node[above] {\squared{6}} node[below] {2} (6);
		\draw[->, edge] (6) -- node[above] {\squared{7}} node[below] {2} (7);
		
	\end{tikzpicture}
\end{figure}

\bigskip

\textbf{Вычисление резерва времени выполнения работы}

Вычислим резерв времени выполнения работы 3

\[
\Delta^{max}_3 = T - f(k(3)) - g(i(3)) - \tau_3 = 9 - f(3) - g(2) - 1 = 8 - 4 - 0.5 = 3.5 
\]