\chapter{Линейное программирование}

В данной главе будут рассматриваться задачи, в которых целевая функция и ограничения представляют собой линейные формы от переменных.

\definition

Задача линейного программирования находится в \definitionfont{канонической форме}, если она записана следующим образом

\begin{alignat*}{2}
	& \sum_{j=1}^{n} c_j x_j \to \max_{(x_j)}, \\
	& \sum_{j=1}^{n}a_{ij} x_j \le b_i, && i = 1\dots m, \\
	& x_j \ge 0, && j = 1\dots n.	
\end{alignat*}

\begin{note}	
	В случае поиска $\min$ ограничения должны иметь вид
	\[\sum_{j=1}^{n}a_{ij} x_j \ge b_i, \quad i = 1\dots m.\]
\end{note}

\begin{note}	
	Везде далее в примерах $x_j \ge 0$ и аналогичные ограничения будут опускаться.
\end{note}

\section{Геометрический метод}

\fact

Уравнение $ax + by = c$ определяет прямую на плоскости $XY$.

\fact

Неравенство $ax + by \stackrel{\ge}{\le} c$ определяет полуплоскость, точки которой лежат не выше (не ниже) прямой $ax + by = c$.

\fact

Прямую $ax + by = c$ можно записать в виде
\[
y(x) = \underbrace{-\frac{a}{b}}_{\tan \alpha}x + \frac{c}{b},
\]

где $\alpha$ --- угол между прямой и осью $OX$.

\example

Решим с помощью геометрического метода следующую задачу
\[
\begin{linear}{crrrrl}
	& 4x_1 & + & 5x_2 & \to & \max\limits_{(x_1, x_2)}, \\
	& 4x_1 & + & 3x_2 & \le & 12, \\
	& 2x_1 & + & 5x_2 & \le & 10. \\
\end{linear}
\]

Изобразим на плоскости прямые, соответствующие ограничениям, точку их пересечения $O$ и множество допустимых решений (рис. 4.1).

\begin{figure}[H]
	\centering
	\begin{tikzpicture}
		\begin{axis}[axis lines=middle, xlabel={$x_1$}, ylabel={$x_2$}, grid=major, width=0.7\textwidth, xmin=-1, xmax=6, ymin=-2, ymax=5]
			\addplot[domain=-1:6, samples=2, blue, thick]{-4/3 * x + 12/3};
			\addlegendentry{$4x_1 + 3x_2 = 12$}
			
			\addplot[domain=-1:6, samples=2, red, thick]{-2/5 * x + 10/5};
			\addlegendentry{$2x_1 + 5x_2 = 10$}
			
			\addplot[pattern=north east lines]
			coordinates {
				(0,0)
				(0,2)
				(15/7,8/7)
				(3,0)
				(0,0)
			};
			
			\addplot[only marks, mark=* ,] coordinates {(15/7, 8/7)};
			\node at (axis cs:15/7,8/7) [anchor=south west]{$O$};
		\end{axis}
	\end{tikzpicture}
	\caption{}
\end{figure}

То что заштрихованная на рисунке область являются областью допустимых значений следует из следующих фактов:
\begin{itemize}[nosep]
	\item $4x_1 + 3x_2 \le 12$ и $2x_1 + 5x_2 \le 10$, значит допустимые решения должны лежать не выше прямых $4x_1 + 3x_2 = 12$ и $2x_1 + 5x_2 = 10$;
	
	\item в задачах линейного программирования $x_j \ge 0, j = 1 \dots n$, поэтому $x_1, x_2 \ge 0$.
\end{itemize}

Таким образом, нас будут интересовать точки плоскости, находящиеся в первой четверти и лежащие не выше прямых, соответствующих ограничениям.

Целевой функции соответствует семейство прямых $4x_1 + 5x_2 = c$. Задача нахождения максимума соответствует нахождению максимального числа $c=c'$, при котором прямая целевой функции пересекает множество допустимых решений.

Для примера нарисуем прямые при $c = 10$ и $c = 20$ (рис. 4.2).

\begin{figure}[H]
	\centering
	\begin{tikzpicture}
		\begin{axis}[axis lines=middle, xlabel={$x_1$}, ylabel={$x_2$}, grid=major, width=0.7\textwidth, xmin=-1, xmax=6, ymin=-2, ymax=5]
			\addplot[domain=-1:6, samples=2, blue, thick]{-4/3 * x + 12/3};
			\addlegendentry{$4x_1 + 3x_2 = 12$}
				
			\addplot[domain=0:6, samples=2, red, thick]{-2/5 * x + 10/5};
			\addlegendentry{$2x_1 + 5x_2 = 10$}
		
			\addplot[domain=-1:6, samples=2, green, thick]{-4/5 * x + 20/5};
			\addlegendentry{$4x_1 + 5x_2 = 20$}
			
			\addplot[domain=-1:6, samples=2, orange, thick]{-4/5 * x + 10/5};
			\addlegendentry{$4x_1 + 5x_2 = 10$}
				
			\addplot[pattern=north east lines]
			coordinates {
				(0,0)
				(0,2)
				(15/7,8/7)
				(3,0)
				(0,0)
			};
			
			\addplot[only marks, mark=* ,] coordinates {(15/7, 8/7)};
			\node at (axis cs:15/7,8/7) [anchor=south west]{$O$};
		\end{axis}
	\end{tikzpicture}
	\caption{}
\end{figure}

Видно, что при $c = 20$ прямая целевой функции не пересекает множество допустимых решений, а при $c = 10$ --- пересекает.

Если брать $c > 20$, то прямая будет отдаляться от области допустимых решений, значит нужно брать $c < 20$. Если уменьшать $c$, то прямая $4x_1 + 5x_2 = c$ будет двигаться вниз, а при уменьшении $c=20$ прямая пересечёт множество допустимых решений в тот момент, когда она будет проходить через точку $O$. Таким образом, целевая функция достигает максимального значения $\Leftrightarrow$ прямая целевой функции пересекает точку $O$ (рис. 4.3).

\begin{figure}[H]
	\centering
	\begin{tikzpicture}
		\begin{axis}[axis lines=middle, xlabel={$x_1$}, ylabel={$x_2$}, grid=major, width=0.7\textwidth, xmin=-1, xmax=6, ymin=-2, ymax=5]
			\addplot[domain=-1:6, samples=100, blue, thick]{-4/3 * x + 12/3};
			\addlegendentry{$4x_1 + 3x_2 = 12$}
			
			\addplot[domain=0:6, samples=100, red, thick]{-2/5 * x + 10/5};
			\addlegendentry{$2x_1 + 5x_2 = 10$}
			
			\addplot[domain=-1:6, samples=100, thick]{-4/5 * x + 100/35};
			\addlegendentry{$4x_1 + 5x_2 = c'$}
			
			\addplot[domain=-1:6, samples=100, green, thick]{-4/5 * x + 20/5};
			\addlegendentry{$4x_1 + 5x_2 = 20$}
			
			\addplot[domain=-1:6, samples=100, orange, thick]{-4/5 * x + 10/5};
			\addlegendentry{$4x_1 + 5x_2 = 10$}
			
			\addplot[pattern=north east lines]
			coordinates {
				(0,0)
				(0,2)
				(15/7,8/7)
				(3,0)
				(0,0)
			};
			
			\addplot[only marks, mark=* ,] coordinates {(15/7, 8/7)};
			\node at (axis cs:15/7,8/7) [anchor=south west]{$O$};
		\end{axis}
	\end{tikzpicture}
	\caption{}
\end{figure}

Найдём координаты точки $O$
\begin{align*}
	&\begin{cases}
		4x_1 + 3x_2 = 12, \\
		2x_1 + 5x_2 = 10; \quad \Big| \quad \cdot 2
	\end{cases} \\
	&\begin{cases}
		4x_1 + 3x_2 = 12, \\
		4x_1 + 10x_2 = 20;
	\end{cases}
\end{align*}

\[
\Downarrow
\]
\[
7x_2 = 8,
\]
\[
x_2 = \frac{8}{7};
\]
\[
x_1 = \frac{10}{2} - \frac{5}{2}x_2 = 5 - \frac{20}{7} = \frac{15}{7}.
\]

Таким образом, точка $O$ имеет координаты $(15/7, 8/7)$. Найдём $c'$ из соображения, что прямая $4x_1 + 5x_2 = c'$ проходит через точку $O$
\[
4x_1 + 5x_2 = c',
\]
\[
4 \cdot \frac{15}{7} + 5 \cdot \frac{8}{7} = c',
\]
\[
c' = \frac{100}{7}.
\]

То есть прямая $4x_1 + 5x_2 = 100/7$ пересекает точку $O$, что соответствует максимальному значению целевой функции $4x_1 + 5x_2$.

Таким образом
\[
\boxed{x_1^* = \frac{15}{7}, \qquad x_2^* = \frac{8}{7}, \qquad \max_{(x_1, x_2)} \big(4x_1 + 5x_2\big) = \frac{100}{7}}.
\]

\section{Теория двойственности}

\definition

Пару задач будет называть \definitionfont{двойственной}, если задачи имеют следующий вид

\begin{itemize}
	\item[$\circled{1}$]
	
	\begin{alignat*}{2}
		& \sum_{j=1}^{n} c_j x_j \to \max_{(x_j)}, \\
		& \sum_{j=1}^{n}a_{ij} x_j \le b_i, && i = 1\dots m, \\
		& x_j \ge 0, && j = 1\dots n;		
	\end{alignat*}
	
	\item[$\circled{2}$]
	
	\begin{alignat*}{2}
		& \sum_{i=1}^{m} b_i y_i \to \min_{(y_i)}, \\
		& \sum_{i=1}^{m} a_{ij} y_i \ge c_j, && j = 1\dots n, \\
		& y_i \ge 0, && i = 1\dots m.		
	\end{alignat*}
\end{itemize}

Тогда задача $\circled{1}$ называется \definitionfont{прямой}, а $\circled{2}$ --- \definitionfont{двойственной}.

\begin{note}
	Для понимания можно привести аналогию. Пусть
	
	\begin{itemize}[nosep]
		\item $j = 1 \dots n$ --- вид производимой продукции,
		
		\item $i = 1 \dots m$ --- вид ресурса,
		
		\item $c_j$ --- доход от продажи продукции $j$-го типа,
		
		\item $a_{ij}$ --- расход $i$-го ресурса на производство единицы продукции $j$-го типа,
		
		\item $b_i$ --- запасы $i$-го ресурса.
	\end{itemize}
	
	Решение прямой задачи состоит в том, чтобы понять, сколько производить продукции каждого вида для максимизации прибыли ($x_j$ --- сколько единиц продукции $j$-го типа производить).
	
	Решение двойственной задачи состоит в том, чтобы понять, как минимизировать суммарную стоимость ресурсов так, чтобы доход от продажи продукции $c_j$ не превышал стоимость требуемых для неё ресурсов (иначе эту продукцию можно было бы производить и продавать до бесконечности). Таким образом, $y_i$ --- стоимость ресурса.
\end{note}

\example

Составить двойственную задачу для

\[
\begin{linear}{crrrrrrrl}
	& 4x_1 & + & 5x_2 & + & 9x_3 & + & 11x_4 & \to \max\limits_{(x)}, \\
	& 1x_1 & + & 1x_2 & + & 1x_3 & + & 1x_4 & \le 15, \\
	& 7x_1 & + & 5x_2 & + & 3x_3 & + & 2x_4 & \le 120, \\
	& 3x_1 & + & 5x_2 & + & 10x_3 & + & 15x_4 & \le 100. \\
\end{linear}
\]

Прямая задача имеет $4$ переменные $x_1, x_2, x_3, x_4$ и $3$ ограничения, значит двойственная задача будет иметь $3$ переменные $y_1, y_2, y_3$ и $4$ ограничения.

Для удобства составления двойственной задачи поставим напротив каждого ограничения соответствующую переменную $y_i$
\[
\begin{linear}{crrrrrrrlc}
	& 1x_1 & + & 1x_2 & + & 1x_3 & + & 1x_4 & \le 15, \quad\quad &\big| \quad y_1 \\
	& 7x_1 & + & 5x_2 & + & 3x_3 & + & 2x_4 & \le 120, \quad &\big| \quad y_2 \\
	& 3x_1 & + & 5x_2 & + & 10x_3 & + & 15x_4 & \le 100. \quad &\big| \quad y_3 \\
\end{linear}
\]

Составим $4$ ограничения на переменные $y_1, y_2, y_3$: коэффициенты левой части ограничений будут соответствовать <<столбцу>> из ограничений на $x$, а правая часть будет соответствовать исходной целевой функции.
\[
\begin{linear}{crrrrrl}
	&1y_1 & + & 7y_2 & + & 3y_3 & \ge 4, \\
	&1y_1 & + & 5y_2 & + & 5y_3 & \ge 5, \\
	&1y_1 & + & 3y_2 & + & 10y_3 & \ge 9, \\
	&1y_1 & + & 2y_2 & + & 15y_3 & \ge 11, \\
\end{linear}
\]

Коэффициенты перед $y_1$ --- коэффициенты первого ограничения на $x$, перед $y_2$ --- второго ограничения на $x$ и так далее. Правая часть новых ограничений --- коэффициенты исходной целевой функции $4x_1 + 5x_2 + 9x_3 + 11x_4 \to \max$.

\implication

Если исходная задача имеет $2$ ограничения, то двойственная задача будет содержать лишь $2$ переменные, а значит её можно решить геометрически.

\fact

Если $x = (x_j)$ --- допустимое решение прямой задачи, а $y = (y_i)$ --- двойственной, то
\[
\sum_{j=1}^{n} c_j x_j \le \sum_{i=1}^{m} b_i y_i.
\]

\prooof

\begin{align*}
	\sum_{j=1}^{n} c_j x_j \; &\stackrel{(1)}{\le} \sum_{j=1}^{n} x_j \sum_{i=1}^{m} a_{ij} y_i \\
	&= \sum_{i=1}^{m} \sum_{j=1}^{n} a_{ij} x_j y_i \\
	&= \sum_{i=1}^{m} y_i \sum_{j=1}^{n} a_{ij} x_j \\
	& \stackrel{(2)}{\le} \sum_{i=1}^{m} y_i b_i,
\end{align*}

$(1)$ и $(2)$ --- определения двойственной и прямой задач соответственно:
\[
	c_j \le \sum_{i=1}^{m} a_{ij} y_i \qquad \sum_{j=1}^{n}a_{ij} x_j \le b_i. 
\]

\implication\label{impl:primal_dual}

\[
\sum_{j=1}^{n} c_j x_j \le \sum_{i=1}^{m} \sum_{j=1}^{n} a_{ij} x_j y_i \le \sum_{i=1}^{m} y_i b_i.
\]

\fact[критерий оптимальности решений]\label{fact:opt_solution_criterium}

Допустимые решения прямой и обратной задач являются оптимальными $\Longleftrightarrow$ выполняются равенства
\[
\sum_{j=1}^{n} c_j x_j = \sum_{i=1}^{m} \sum_{j=1}^{n} a_{ij} x_j y_i = \sum_{i=1}^{m} y_i b_i.
\]

\definition

\definitionfont{Условиями дополняющей нежёсткости} будем называть следующие условия
\[
x_j \Big(\sum_{i=1}^{m} a_{ij} y_i - c_j\Big) = 0, \qquad j = 1 \dots n \tag{1}
\]
\[
y_i \Big(\sum_{j=1}^{n} a_{ij} x_j - b_i\Big) = 0. \qquad i = 1 \dots m \tag{2}
\]

\fact

Выполнение условий дополняющей нежёсткости эквивалентно равенству
\[
\sum_{j=1}^{n} c_j x_j = \sum_{i=1}^{m} b_i y_i.
\]

\prooof

\begin{itemize}[nosep]
	\item[]
	
	\item[\fbox{$\Rightarrow$}] Пусть оба условия выполняются. Просуммируем (1) по всем $j$
	\[
	\sum_{j=1}^{n} x_j \Big(\sum_{i=1}^{m} a_{ij} y_i - c_j\Big) = n \cdot 0 = 0,
	\]
	\[
	\sum_{j=1}^{n} x_j \Big(\sum_{i=1}^{m} a_{ij} y_i\Big) = \sum_{j=1}^{n} x_j c_j,
	\]
	\[
	\sum_{j=1}^{n} \sum_{i=1}^{m} a_{ij} x_j y_i = \sum_{j=1}^{n} x_j c_j,
	\]
	\[
	\sum_{i=1}^{m} \sum_{j=1}^{n} a_{ij} x_j y_i = \sum_{j=1}^{n} x_j c_j,
	\]
	\[
	\sum_{i=1}^{m} y_i \sum_{j=1}^{n} a_{ij} x_j = \sum_{j=1}^{n} x_j c_j,
	\]
	\[
	\sum_{i=1}^{m} y_i \Big(\sum_{j=1}^{n} a_{ij} x_j - b_i + b_i \Big) = \sum_{j=1}^{n} x_j c_j,
	\]
	\[
	\Downarrow \text{(2)}
	\]
	\[
	\sum_{i=1}^{m} y_i b_i = \sum_{j=1}^{n} x_j c_j.
	\]

	\item[\fbox{$\Leftarrow$}] Пусть решения $x = (x_j)$ и $y = (y_i)$ прямой и двойственной задач оптимальны. Тогда по \hyperref[fact:opt_solution_criterium]{Критерию оптимальности решений}
	\[
	\sum_{j=1}^{n} c_j x_j = \sum_{i=1}^{m} \sum_{j=1}^{n} a_{ij} x_j y_i = \sum_{i=1}^{m} y_i b_i.
	\]
	
	Рассмотрим одно равенство из двух
	\[
	\sum_{i=1}^{m} \sum_{j=1}^{n} x_j a_{ij} y_i = \sum_{j=1}^{n} x_j c_j,
	\]
	\[
	\sum_{j=1}^{n} \sum_{i=1}^{m} x_j a_{ij} y_i = \sum_{j=1}^{n} x_j c_j,
	\]
	\[
	\sum_{j=1}^{n} x_j \Big(\sum_{i=1}^{m} a_{ij} y_i - c_j\Big) = 0.
	\]

	По \hyperref[impl:primal_dual]{Следствию} все слагаемые неотрицательны, значит каждое из слагаемых равняется нулю, то есть
	\[
	x_j \Big(\sum_{i=1}^{m} a_{ij} y_i - c_j\Big) = 0, \qquad j = 1 \dots n.
	\]
	
	Действия аналогично со вторым равенством, можно доказать, что
	\[
	y_i \Big(\sum_{j=1}^{n} a_{ij} x_j - b_i\Big) = 0, \qquad i = 1 \dots m.
	\]
\end{itemize}

\implication

Выполнение условий дополняющей нежёсткости эквивалентно оптимальности решений прямой и двойственной задач.

\algorithm[решения задач с использованием теории двойственности]

\textbf{Исходные данные}: $n > 2$, но $m = 2$.

\begin{enumerate}[nosep]
	\item По прямой задаче строим двойственную, в которой будет ровно $2$ переменные.
	
	\item Решаем двойственную задачу геометрически, находя её оптимальное решение $(y_1^*, y_2^*)$.

	\item Подставляем $y_1^*$ и $y_2^*$ в первое условие дополняющей нежёсткости. В двух ограничениях двойственной задачи на оптимальном решении будет выполняться равенство, а во всех остальных равенства не будет; пусть равенство выполнено при $j = j_1$ и $j = j_2$.
	
	Если $j \neq j_1$ и $j \neq j_2$, то для выполнения первого условия дополняющей нежёсткости необходимо, чтобы $x_j^* = 0$, значит
	\[
	x_j^* = \begin{cases}
		0, & j \neq j_1 \land j \neq j_2 \\
		\dots, & \dots.
	\end{cases}
	\]
	
	\item Подставляем во второе условие дополняющей нежёсткости значения $(x_j^*)$, из предыдущего пункта следует, что все значения кроме $x_{j_1}^*$ и $x_{j_2}^*$ равняются нулю. После подстановки будет система линейных уравнений с двумя неизвестными
	\[
	\begin{cases}
		y_1^* (a_{1 j_1} x_{j_1}^* - b_1) = 0, \\
		y_2^* (a_{2 j_2} x_{j_2}^* - b_2) = 0;
	\end{cases}
	\]
	
	решением системы будут значения $x_{j_1}^*$ и $x_{j_2}^*$. 
	
	\item Оптимальное решение прямой задачи имеет вид
	\[
	x^* = (0, \dots 0, x_{j_1}^*, 0, \dots, 0, x_{j_2}^*, 0, \dots, 0).
	\]
\end{enumerate}

\section{Симплекс-метод}

В этом методе мы будем рассматривать ограничения задач в виде равенств
\[
\sum_{j=1}^{n} x_j a_{ij} + \underbrace{y_i}_{\ge 0} = b_i, \quad i = 1 \dots m.
\]

\begin{note}
	Если в задаче ищется $\min$, то нужно добавить $-y_i \ge 0$.
\end{note}

Идея: представляем множество допустимых решений с использованием базисных переменных через эквивалентные преобразования.

Необходимо выбрать $m$ базисных переменных (столько же, сколько ограничений). Для простоты пусть это будут переменные $x_1, x_2, \dots, x_m$ (в реальности это могут быть произвольные переменные $x_{j_1}$, $x_{j_2}$ и так далее).

Базисные переменные нужно представить в виде
\[
x_1 = b_1 + l_1(x_{m+1}, x_{m+2}, \dots, x_n),
\]
\[
x_1 = b_2 + l_2(x_{m+1}, x_{m+2}, \dots, x_n),
\]
\[
\dots
\]
\[
x_m = b_m + l_m(x_{m+1}, x_{m+2}, \dots, x_n),
\]

где $l_i$ --- линейная функция от не базисных переменных.

Представить базисные переменные через не базисные можно из исходных ограничений с помощью метода Гаусса (сначала исключаем $x_1$, затем $x_2$, $x_3$ и так далее).

Алгоритм итеративный.

\begin{enumerate}[nosep]
	\item Выбрать базисные переменные и записать их представление через не базисные функции
	
	\item Записать целевую функцию $x_0$ через не базисные переменные
	\[
	x_0 = b_0 + \sum_{j=m+1}^{n} \alpha_j x_j.
	\]
	
	\item Необходимо определить, перед каким $x_j$ есть коэффициент $\alpha_j > 0$.
	
	Действительно, если $\alpha_j < 0$, то если положить $x_j = 0$, то целевая функция увеличится --- что хорошо, ведь мы максимизируем её. Если $\alpha_j > 0$, то при $x_j = 0$ целевая функция уменьшится --- плохо.
	
	Пусть некоторый коэффициент $\alpha_j > 0$, тогда необходимо добавить в базис переменную $x_j$, исключив из него какую-нибудь переменную. Как исключить? Нужно понять, в выражении какой базисной переменной $x_k$ коэффициент перед $x_j$ ($\beta_{kj}$) меньше нуля.
	
	Если таких кандидатов несколько, то необходимо выбрать такую переменную $x_k$, в которой
	\[
	\abs{\frac{b_k}{\beta_{kj}}} \to \min.
	\]
	
	Если таких кандидатов нет, то есть нет $x_k$, в которых коэффициент перед $x_j$ меньше нуля, то $x_k \to \infty$ и все базисные переменные $\to \infty$, значит задача неограничена не имеет решения.
	
	После добавления базисной переменной все повторяем всё заново, пока не получим, что в целевой функции все коэффициенты меньше нуля.
\end{enumerate}

\fact[критерий оптимальности решения]

Если все коэффициенты $\alpha_j$ в выражении $x_0$ меньше нуля, то решение $x_0 = b_0$ является оптимальным.

\begin{note}
	Аналогично для случаев поиска $\min$: если все коэффициенты $> 0$, то имеем оптимальное решение.
\end{note}

\example

12332

\section{Практика применения симплекс-метода}

\subsection{Решение задачи раскроя}

\problem[раскроя]

Рассмотрим задачу аналогичную \hyperref[pr:cutting_stock]{Задаче} из первой главы, но с небольшими отличиями. Требуется

\begin{itemize}[nosep]
	\item 30 труб по 3.5 м,
	
	\item 21 труба по 2.5 м.
\end{itemize}

\bigskip

а покупать трубы можно

\begin{itemize}[nosep]
	\item длиной 8 м по 8 руб/шт,
	
	\item длиной 6 м по 6 руб/шт.
\end{itemize}

Необходимо закрыть потребность в трубах с минимальными затратами.

\mathmodel

Рассмотрим все варианты раскроя длинных труб на короткие (таблицы 4.1 и 4.2).

\begin{table}[h!]
	\centering
	\begin{tabular}{| c | c | c | } 
		\hline
		№ & 3.5 м & 2.5 м \\ 
		\hline
		1 & 2 & 0 \\\hline
		2 & 1 & 1 \\\hline
		3 & 0 & 3 \\\hline
	\end{tabular}
	\caption{Варианты раскроя трубы длиной 8 м}
\end{table}

\begin{table}[h!]
	\centering
	\begin{tabular}{| c | c | c | } 
		\hline
		№ & 3.5 м & 2.5 м \\ 
		\hline
		4 & 1 & 1 \\\hline
		5 & 0 & 2 \\\hline
	\end{tabular}
	\caption{Варианты раскроя трубы длиной 6 м}
\end{table}

Можно заметить, что варианты №2 и №4 дают один и тот же результат (по одной трубе 3.5 м и 2.5 м), но один из них требует трубу длиной 8 м, а второй --- 6 м. Поскольку необходимо минимизировать затраты, то вариант №4 всегда выгоднее варианта №2, поэтому №2 можно отбросить. Таким образом, есть 4 способа раскроя длинных труб.

Введём переменные, соответствующие вариантам раскроя
\begin{itemize}[nosep]
	\item $x_1$ --- №1,
	
	\item $x_2$ --- №3,
	
	\item $x_3$ --- №4,
	
	\item $x_4$ --- №5,
\end{itemize}

где $x_j$ показывает, сколько длинных труб будет раскроено соответствующим способом. Напишем целевую функцию и ограничения в терминах переменных

\[
8x_1 + 8x_2 + 6x_3 + 6x_4 \to \min_{x},
\]
\[
2x_1 + x_3 \ge 30,
\]
\[
3x_2 + x_3 + 2x_4 \ge 21.
\]

\sout{О том, что $x_1, x_2, x_3, x_4 \ge$ говорить не будем}.

\solution

Решать задачу будем с помощью \hyperref[sec:simplex_algorithm]{Симплекс-метода}. Запишем задачу через равенство с переменными $y_i$
\[
\begin{linear}{crrrrrrrrrrrrrrl}
	& \min &\big(&8x_1 & + & 8x_2 & + & 6x_3  & + & 6x_4&\big), \\
	&&& 2x_1 &   &      & + & 1x_3 &   &      && - & 1y_1 &   && = 30, \\
	&&&      &   & 3x_2 & + & 1x_3 & + & 2x_4 &&   &      & - & 1y_2 & = 21.
\end{linear}
\]

\begin{enumerate}
	\item[\fbox{Шаг 0}]
	
	Пусть в базис будут входить переменные $x_1, x_2$. Выразим их и запишем целевую функцию через не базисные
	\[
	\begin{linear}{rrrrrrrr}
		& 2x_1 & = & 30 & - & 1x_3 & + & 1y_1, \\
		& x_1  & = & 15 & - & \frac{1}{2}x_3 & + & \frac{1}{2}y_1.
	\end{linear}
	\]
	
	\[
	\begin{linear}{rrrrrrrrrr}
		& 3x_2 & = & 21 & - & 1x_3 & - & 2x_4 & + & 1y_2, \\
		& x_2 & = & 7 & - & \frac{1}{3}x_3 & - & \frac{2}{3}x_4 & + & \frac{1}{3}y_2.
	\end{linear}
	\]
	
	\[
	\begin{linear}{rrrrrrrrrrrrrrrrrrr}
		& x_0 = &&&&& \cancel{8x_1} & + & \cancel{8x_2} & + & 6x_3 & + & 6x_4 &&&&&& + \\
		&& 8&(&15 &&&&& - & \frac{1}{2}x_3 &&& + & \frac{1}{2}y_1 &&&) & + \\
		&& 8&(&7 &&&&& - & \frac{1}{3}x_3 & - & \frac{2}{3}x_4 &&& + & \frac{1}{3}y_2&) & = \\
		&&&& 176 &&&&& - & \frac{2}{3}x_3 & + & \frac{2}{3}x_4 & + & 4y_1 & + & \frac{8}{3}y_2 &.&
	\end{linear}
	\]
	
	Текущий базис --- $\{x_1, x_2\}$.
	
	\item[\fbox{Шаг 1}] Видно, что коэффициент перед $x_3$ в выражении $x_0$ отрицателен. Это значит, что текущее решение (занулить все не базисные переменные) не является оптимальным, поскольку значение $x_0$ будет меньше при $x_3 > 0$, чем при $x_3 = 0$.
	
	Нам нужно добавить $x_3$, удалив из базиса одну переменную. Коэффициент перед $x_3$ отрицателен и в $x_1$, и в $x_2$, поэтому на удаление есть 2 кандидата. Посчитаем соотношения свободных членов к коэффициенту перед $x_3$
	\[
	x_1: \quad \frac{15}{\frac{1}{2}} = 30,
	\]
	\[
	x_2: \quad \frac{7}{\frac{1}{3}} = 21.
	\]
	
	Минимальное из этих чисел --- $21$, поэтому удалим из базиса $x_2$ и добавим $x_3$.
	
	Выразим $x_3$ через не базисные переменные
	\[
	\begin{linear}{rrrrrrrrrr}
		& \frac{1}{3}x_3 & = & 7- & 1x_2 & - & \frac{2}{3}x_4 & + & \frac{1}{3}y_2, \\
		& x_3 & = & 21- & 3x_2 & - & 2x_4 & + & y_2.
	\end{linear}
	\]
	
	Исключим $x_3$ из $x_1$
	\[
	\begin{linear}{rrrrrrrrrrrrrrrrr}
		& x_1 = &&& 15 &&& - & \cancel{\frac{1}{2}x_3} &&& + & \frac{1}{2}y_1 &&&& + \\
		&& -\frac{1}{2}&(&21 & - & 3x_2 &&& - & 2x_4 &&& + & 1y_2&) & = \\
		&&&& \frac{9}{2} & + & \frac{3}{2}x_2 &&& + & 1x_4 & + & \frac{1}{2}y_1 & - & \frac{1}{2}y_2 &.&
	\end{linear}
	\]
	
	Исключим $x_3$ из целевой функции
	\[
	\begin{linear}{rrrrrrrrrrrrrrrrr}
		& x_0 = &&& 176 &&& - & \cancel{\frac{2}{3}x_3} & + & \frac{2}{3}x_4 & + & 4y_1 & + & \frac{8}{3}y_2 && + \\
		&& -\frac{2}{3}&(&21 & - & 3x_2 &&& - & 2x_4 &&& + & 1y_2 &) & = \\
		&&&& 162 & + & 2x_2 &&& + & 2x_4 & + & 4y_1 & + & 2y_2 &.&
	\end{linear}
	\]
	
	Текущий базис --- $\{x_1, x_3\}$.
	
	\item[\fbox{Шаг 2}] Видно, что все коэффициенты в $x_0$ положительны, значит текущее решение (занулить все не базисные переменные) является оптимальным. То есть
	\[
	x_2^* = x_4^* = y_1^* = y_2^* = 0,
	\]
	\[
	\Downarrow
	\]
	\[
	x_1^* = \frac{9}{2}, \quad x_3^* = 21.
	\]
	
	\textbf{Ответ}:
	\begin{itemize}[nosep]
		\item нужно взять 5 $\times$ 8 м и 21 $\times$ 6 м;
		
		\item каждую трубу длиной 8 м раскроить на 2 $\times$ 3.5 м;
		
		\item каждую трубу длиной 6 м раскроить на 1 $\times$ 3.5 м и 1 $\times$ 2.5 м;
		
		\item суммарные затраты составят 176 рублей.
	\end{itemize}
\end{enumerate}
