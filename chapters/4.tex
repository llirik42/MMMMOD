\chapter{Линейное программирование}

В данной главе будут рассматриваться задачи, в которых целевая функция и ограничения представляют собой линейные формы от переменных.

\definition

Задача линейного программирования находится в \definitionfont{канонической форме}, если она записана следующим образом

\begin{alignat*}{2}
	& \sum_{j=1}^{n} c_j x_j \to \max_{(x_j)}, \\
	& \sum_{j=1}^{n}a_{ij} x_j \le b_i, && i = 1\dots m, \\
	& x_j \ge 0, && j = 1\dots n.	
\end{alignat*}

\begin{note}	
	В случае поиска $\min$ ограничения должны иметь вид
	\[\sum_{j=1}^{n}a_{ij} x_j \ge b_i, \quad i = 1\dots m.\]
\end{note}

\begin{note}	
	Везде далее в примерах $x_j \ge 0$ и аналогичные ограничения будут опускаться.
\end{note}

\section{Геометрический метод}

\fact

Уравнение $ax + by = c$ определяет прямую на плоскости $XY$.

\fact

Уравнение $ax + by \stackrel{\ge}{\le} c$ определяет полуплоскость, точки которой лежат не выше (не ниже) прямой $ax + by = c$.

\fact

Прямую $ax + by = c$ можно записать в виде
\[
y(x) = \underbrace{-\frac{a}{b}}_{\tan \alpha}x + \frac{c}{b},
\]

где $\alpha$ --- угол между прямой и осью $OX$.

\definition

\definitionfont{Геометрический метод} заключается в представлении задачи на плоскости

\begin{note}
	Метод применяется, когда в задаче 2 переменные. Если переменных меньше, то задачу проще решить без геометрического метода, а если больше, то геометрический метод становится неудобен.
\end{note}

\example

Решим с помощью геометрического метода следующую задачу
\[
\begin{linear}{crrrrl}
	& 4x_1 & + & 5x_2 & \to & \max\limits_{(x_1, x_2)}, \\
	& 4x_1 & + & 3x_2 & \le & 12, \\
	& 2x_1 & + & 5x_2 & \le & 10. \\
\end{linear}
\]

\definition

Пару задач будет называть \definitionfont{двойственной}, если задачи имеют следующий вид

\begin{itemize}
	\item[$\circled{1}$]
	
	\begin{alignat*}{2}
		& \sum_{j=1}^{n} c_j x_j \to \max_{(x_j)}, \\
		& \sum_{j=1}^{n}a_{ij} x_j \le b_i, && i = 1\dots m, \\
		& x_j \ge 0, && j = 1\dots n;		
	\end{alignat*}
	
	\item[$\circled{2}$]
	
	\begin{alignat*}{2}
		& \sum_{i=1}^{m} b_i y_i \to \min_{(y_i)}, \\
		& \sum_{i=1}^{m} a_{ij} y_i \ge c_j, && j = 1\dots n, \\
		& y_i \ge 0, && i = 1\dots m.		
	\end{alignat*}
\end{itemize}

Тогда задача $\circled{1}$ называется \definitionfont{прямой}, а $\circled{2}$ --- \definitionfont{двойственной}.

\example

Составить двойственную задачу для

\[
\begin{linear}{crrrrrrrl}
	& 4x_1 & + & 5x_2 & + & 9x_3 & + & 11x_4 & \to \max\limits_{(x)}, \\
	& 1x_1 & + & 1x_2 & + & 1x_3 & + & 1x_4 & \le 15, \\
	& 7x_1 & + & 5x_2 & + & 3x_3 & + & 2x_4 & \le 120, \\
	& 3x_1 & + & 5x_2 & + & 10x_3 & + & 15x_4 & \le 100. \\
\end{linear}
\]

Прямая задача имеет $4$ переменные $x_1, x_2, x_3, x_4$ и $3$ ограничения, значит двойственная задача будет иметь $3$ переменные $y_1, y_2, y_3$ и $4$ ограничения.

Для удобства составления двойственной задачи поставим напротив каждого ограничения соответствующую переменную $y_i$
\[
\begin{linear}{crrrrrrrlc}
	& 1x_1 & + & 1x_2 & + & 1x_3 & + & 1x_4 & \le 15, \quad\quad &\big| \quad y_1 \\
	& 7x_1 & + & 5x_2 & + & 3x_3 & + & 2x_4 & \le 120, \quad &\big| \quad y_2 \\
	& 3x_1 & + & 5x_2 & + & 10x_3 & + & 15x_4 & \le 100. \quad &\big| \quad y_3 \\
\end{linear}
\]

Составим $4$ ограничения на переменные $y_1, y_2, y_3$: коэффициенты левой части ограничений будут соответствовать <<столбцу>> из ограничений на $x$, а правая часть будет соответствовать исходной целевой функции.
\[
\begin{linear}{crrrrrl}
	&1y_1 & + & 7y_2 & + & 3y_3 & \ge 4, \\
	&1y_1 & + & 5y_2 & + & 5y_3 & \ge 5, \\
	&1y_1 & + & 3y_2 & + & 10y_3 & \ge 9, \\
	&1y_1 & + & 2y_2 & + & 15y_3 & \ge 11, \\
\end{linear}
\]

Коэффициенты перед $y_1$ --- это коэффициенты первого ограничения на $x$, перед $y_2$ --- второго ограничения на $x$ и так далее. Правая часть новых ограничений --- это коэффициенты исходной целевой функции $4x_1 + 5x_2 + 9x_3 + 11x_4 \to \max$.

\implication

Если исходная задача имеет $2$ ограничения, то двойственная задача будет содержать лишь $2$ переменные, а значит её можно решить геометрически.

\fact

Если $x = (x_j)$ --- допустимое решение прямой задачи, а $y = (y_j)$ --- двойственной, то
\[
\sum_{j=1}^{n} c_j x_j \le \sum_{i=1}^{m} b_i y_i.
\]

\prooof

\begin{align*}
	\sum_{j=1}^{n} c_j x_j \; &\stackrel{(1)}{\le} \sum_{j=1}^{n} x_j \sum_{i=1}^{m} a_{ij} y_i \\
	&= \sum_{i=1}^{m} \sum_{j=1}^{n} a_{ij} x_j y_i \\
	&= \sum_{i=1}^{m} y_i \sum_{j=1}^{n} a_{ij} x_j \\
	& \stackrel{(2)}{\le} \sum_{i=1}^{m} y_i b_i,
\end{align*}

$(1)$ и $(2)$ --- определения двойственной и прямой задач соответственно:
\[
	c_j \le \sum_{i=1}^{m} a_{ij} y_i \qquad \sum_{j=1}^{n}a_{ij} x_j \le b_i. 
\]

\implication\label{impl:primal_dual}

\[
\sum_{j=1}^{n} c_j x_j \le \sum_{i=1}^{m} \sum_{j=1}^{n} a_{ij} x_j y_i \le \sum_{i=1}^{m} y_i b_i.
\]

\fact[критерий оптимальности решений]\label{fact:opt_solution_criterium}

Допустимые решения прямой и обратной задач являются оптимальными $\Longleftrightarrow$ выполняются равенства
\[
\sum_{j=1}^{n} c_j x_j = \sum_{i=1}^{m} \sum_{j=1}^{n} a_{ij} x_j y_i = \sum_{i=1}^{m} y_i b_i.
\]

\definition

\definitionfont{Условиями дополняющей нежёсткости} будем называть следующие условия
\[
x_j \Big(\sum_{i=1}^{m} a_{ij} y_i - c_j\Big) = 0, \qquad j = 1 \dots n \tag{1}
\]
\[
y_i \Big(\sum_{j=1}^{n} a_{ij} x_j - b_i\Big) = 0. \qquad i = 1 \dots m \tag{2}
\]

\fact

Выполнение условий дополняющей нежёсткости эквивалентно равенству
\[
\sum_{j=1}^{n} c_j x_j = \sum_{i=1}^{m} b_i y_i.
\]

\prooof

\begin{itemize}[nosep]
	\item[]
	
	\item[\fbox{$\Rightarrow$}] Пусть оба условия выполняются. Просуммируем (1) по всем $j$
	\[
	\sum_{j=1}^{n} x_j \Big(\sum_{i=1}^{m} a_{ij} y_i - c_j\Big) = n \cdot 0 = 0,
	\]
	\[
	\sum_{j=1}^{n} x_j \Big(\sum_{i=1}^{m} a_{ij} y_i\Big) = \sum_{j=1}^{n} x_j c_j,
	\]
	\[
	\sum_{j=1}^{n} \sum_{i=1}^{m} a_{ij} x_j y_i = \sum_{j=1}^{n} x_j c_j,
	\]
	\[
	\sum_{i=1}^{m} \sum_{j=1}^{n} a_{ij} x_j y_i = \sum_{j=1}^{n} x_j c_j,
	\]
	\[
	\sum_{i=1}^{m} y_i \sum_{j=1}^{n} a_{ij} x_j = \sum_{j=1}^{n} x_j c_j,
	\]
	\[
	\sum_{i=1}^{m} y_i \Big(\sum_{j=1}^{n} a_{ij} x_j - b_i + b_i \Big) = \sum_{j=1}^{n} x_j c_j,
	\]
	\[
	\Downarrow \text{(2)}
	\]
	\[
	\sum_{i=1}^{m} y_i b_i = \sum_{j=1}^{n} x_j c_j.
	\]

	\item[\fbox{$\Leftarrow$}] Пусть решения $x = (x_j)$ и $y = (y_i)$ прямой и двойственной задач оптимальны. Тогда по \hyperref[fact:opt_solution_criterium]{Критерию оптимальности решений}
	\[
	\sum_{j=1}^{n} c_j x_j = \sum_{i=1}^{m} \sum_{j=1}^{n} a_{ij} x_j y_i = \sum_{i=1}^{m} y_i b_i.
	\]
	
	Рассмотрим одно равенство из двух
	\[
	\sum_{i=1}^{m} \sum_{j=1}^{n} x_j a_{ij} y_i = \sum_{j=1}^{n} x_j c_j,
	\]
	\[
	\sum_{j=1}^{n} \sum_{i=1}^{m} x_j a_{ij} y_i = \sum_{j=1}^{n} x_j c_j,
	\]
	\[
	\sum_{j=1}^{n} x_j \Big(\sum_{i=1}^{m} a_{ij} y_i - c_j\Big) = 0.
	\]

	По \hyperref[impl:primal_dual]{Следствие} все слагаемые неотрицательны, значит каждое из слагаемых равняется нулю, то есть
	\[
	x_j \Big(\sum_{i=1}^{m} a_{ij} y_i - c_j\Big) = 0, \qquad j = 1 \dots n.
	\]
	
	Действия аналогично со вторым равенством, можно доказать, что
	\[
	y_i \Big(\sum_{j=1}^{n} a_{ij} x_j - b_i\Big) = 0, \qquad i = 1 \dots m.
	\]
\end{itemize}

\implication

Выполнение условий дополняющей нежёсткости эквивалентно оптимальности решений прямой и двойственной задач.

\algorithm[решения задач с использованием теории двойственности]

\textbf{Исходные данные}: $n > 2$, но $m = 2$.

\begin{enumerate}[nosep]
	\item По прямой задаче строим двойственную, в которой будет ровно $2$ переменные.
	
	\item Решаем двойственную задачу геометрически, находя её оптимальное решение $(y_1^*, y_2^*)$.

	\item Подставляем $y_1^*$ и $y_2^*$ в первое условие дополняющей нежёсткости. В двух ограничениях двойственной задачи на оптимальном решении будет выполняться равенство, а во всех остальных равенства не будет; пусть равенство выполнено при $j = j_1$ и $j = j_2$.
	
	Если $j \neq j_1$ и $j \neq j_2$, то для выполнения первого условия дополняющей нежёсткости необходимо, чтобы $x_j^* = 0$, то есть
	\[
	x_j^* = \begin{cases}
		0, & j \neq j_1 \land j \neq j_2 \\
		\dots, & \dots.
	\end{cases}
	\]
	
	\item Подставим во второе условие дополняющей нежёсткости значения $(x_j^*)$. Из предыдущего пункта следует, что все значения кроме $x_{j_1}^*$ и $x_{j_2}^*$ равняются нулю. После подстановки будет система линейных уравнений с двумя неизвестными
	\[
	\begin{cases}
		y_1 (a_{1 j_1} x_{j_1}^* - b_1) = 0, \\
		y_2 (a_{2 j_2} x_{j_2}^* - b_2) = 0.
	\end{cases}
	\]
	
	После решения системы уравнения найдём значения $x_{j_1}^*$ и $x_{j_2}^*$. 
	
	\item Оптимальным решением прямой задачи будет
	\[
	x = (0, \dots 0, x_{j_1}^*, 0, \dots, 0, x_{j_2}^*, 0, \dots, 0).
	\]
\end{enumerate}

\section{Симплекс-метод}

