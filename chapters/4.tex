\chapter{Линейное программирование}

В данной главе будут рассматриваться задачи, в которых целевая функция и ограничения представляют собой линейные формы от переменных.

\definition

Задача линейного программирования находится в \definitionfont{канонической форме}, если она записана следующим образом

\begin{alignat*}{2}
	& \sum_{j=1}^{n} c_j x_j \to \max_{(x_j)}, \\
	& \sum_{j=1}^{n}a_{ij} x_j \le b_i, && i = 1\dots m, \\
	& x_j \ge 0, && j = 1\dots n.	
\end{alignat*}

\begin{note}	
	В случае поиска $\min$ ограничения должны иметь вид
	\[\sum_{j=1}^{n}a_{ij} x_j \ge b_i, \quad i = 1\dots m.\]
\end{note}

\begin{note}	
	Везде далее в примерах $x_j \ge 0$ и аналогичные ограничения будут опускаться.
\end{note}

\section{Геометрический метод}\label{sec:geom_method}

\fact

Уравнение $ax + by = c$ определяет прямую на плоскости $XY$.

\fact

Неравенство $ax + by \stackrel{\ge}{\le} c$ определяет полуплоскость, точки которой лежат не выше (не ниже) прямой $ax + by = c$.

\fact

Прямую $ax + by = c$ можно записать в виде
\[
y(x) = \underbrace{-\frac{a}{b}}_{\tan \alpha}x + \frac{c}{b},
\]

где $\alpha$ --- угол между прямой и осью $OX$.

\example[решения задачи геометрическим методом]

Решим с помощью геометрического метода следующую задачу
\[
\begin{linear}{crrrrl}
	& 4x_1 & + & 5x_2 & \to & \max\limits_{(x_1, x_2)}, \\
	& 4x_1 & + & 3x_2 & \le & 12, \\
	& 2x_1 & + & 5x_2 & \le & 10. \\
\end{linear}
\]

Изобразим на плоскости прямые, соответствующие ограничениям, точку их пересечения $O$ и множество допустимых решений (рис. 4.1).

\begin{figure}[H]
	\centering
	\begin{tikzpicture}
		\begin{axis}[axis lines=middle, xlabel={$x_1$}, ylabel={$x_2$}, grid=major, width=0.7\textwidth, xmin=-1, xmax=6, ymin=-2, ymax=5]
			\addplot[domain=-1:6, samples=2, blue, thick]{-4/3 * x + 12/3};
			\addlegendentry{$4x_1 + 3x_2 = 12$}
			
			\addplot[domain=-1:6, samples=2, red, thick]{-2/5 * x + 10/5};
			\addlegendentry{$2x_1 + 5x_2 = 10$}
			
			\addplot[pattern=north east lines]
			coordinates {
				(0,0)
				(0,2)
				(15/7,8/7)
				(3,0)
				(0,0)
			};
			
			\addplot[only marks, mark=* ,] coordinates {(15/7, 8/7)};
			\node at (axis cs:15/7,8/7) [anchor=south west]{$O$};
		\end{axis}
	\end{tikzpicture}
	\caption{}
\end{figure}

То что заштрихованная на рисунке область являются областью допустимых значений следует из следующих фактов:
\begin{itemize}[nosep]
	\item $4x_1 + 3x_2 \le 12$ и $2x_1 + 5x_2 \le 10$, значит допустимые решения должны лежать не выше прямых $4x_1 + 3x_2 = 12$ и $2x_1 + 5x_2 = 10$;
	
	\item в задачах линейного программирования $x_j \ge 0, j = 1 \dots n$, поэтому $x_1, x_2 \ge 0$.
\end{itemize}

Таким образом, нас будут интересовать точки плоскости, находящиеся в первой четверти и лежащие не выше прямых, соответствующих ограничениям.

Целевой функции соответствует семейство прямых $4x_1 + 5x_2 = c$. Задача нахождения максимума соответствует нахождению максимального числа $c=c'$, при котором прямая целевой функции пересекает множество допустимых решений.

Для примера нарисуем прямые при $c = 10$ и $c = 20$ (рис. 4.2).

\begin{figure}[H]
	\centering
	\begin{tikzpicture}
		\begin{axis}[axis lines=middle, xlabel={$x_1$}, ylabel={$x_2$}, grid=major, width=0.7\textwidth, xmin=-1, xmax=6, ymin=-2, ymax=5]
			\addplot[domain=-1:6, samples=2, blue, thick]{-4/3 * x + 12/3};
			\addlegendentry{$4x_1 + 3x_2 = 12$}
				
			\addplot[domain=0:6, samples=2, red, thick]{-2/5 * x + 10/5};
			\addlegendentry{$2x_1 + 5x_2 = 10$}
		
			\addplot[domain=-1:6, samples=2, green, thick]{-4/5 * x + 20/5};
			\addlegendentry{$4x_1 + 5x_2 = 20$}
			
			\addplot[domain=-1:6, samples=2, orange, thick]{-4/5 * x + 10/5};
			\addlegendentry{$4x_1 + 5x_2 = 10$}
				
			\addplot[pattern=north east lines]
			coordinates {
				(0,0)
				(0,2)
				(15/7,8/7)
				(3,0)
				(0,0)
			};
			
			\addplot[only marks, mark=* ,] coordinates {(15/7, 8/7)};
			\node at (axis cs:15/7,8/7) [anchor=south west]{$O$};
		\end{axis}
	\end{tikzpicture}
	\caption{}
\end{figure}

Видно, что при $c = 20$ прямая целевой функции не пересекает множество допустимых решений, а при $c = 10$ --- пересекает.

Если брать $c > 20$, то прямая будет отдаляться от области допустимых решений, значит нужно брать $c < 20$. Если уменьшать $c$, то прямая $4x_1 + 5x_2 = c$ будет двигаться вниз, а при уменьшении $c=20$ прямая пересечёт множество допустимых решений в тот момент, когда она будет проходить через точку $O$. Таким образом, целевая функция достигает максимального значения $\Leftrightarrow$ прямая целевой функции пересекает точку $O$ (рис. 4.3).

\begin{figure}[H]
	\centering
	\begin{tikzpicture}
		\begin{axis}[axis lines=middle, xlabel={$x_1$}, ylabel={$x_2$}, grid=major, width=0.7\textwidth, xmin=-1, xmax=6, ymin=-2, ymax=5]
			\addplot[domain=-1:6, samples=100, blue, thick]{-4/3 * x + 12/3};
			\addlegendentry{$4x_1 + 3x_2 = 12$}
			
			\addplot[domain=0:6, samples=100, red, thick]{-2/5 * x + 10/5};
			\addlegendentry{$2x_1 + 5x_2 = 10$}
			
			\addplot[domain=-1:6, samples=100, thick]{-4/5 * x + 100/35};
			\addlegendentry{$4x_1 + 5x_2 = c'$}
			
			\addplot[domain=-1:6, samples=100, green, thick]{-4/5 * x + 20/5};
			\addlegendentry{$4x_1 + 5x_2 = 20$}
			
			\addplot[domain=-1:6, samples=100, orange, thick]{-4/5 * x + 10/5};
			\addlegendentry{$4x_1 + 5x_2 = 10$}
			
			\addplot[pattern=north east lines]
			coordinates {
				(0,0)
				(0,2)
				(15/7,8/7)
				(3,0)
				(0,0)
			};
			
			\addplot[only marks, mark=* ,] coordinates {(15/7, 8/7)};
			\node at (axis cs:15/7,8/7) [anchor=south west]{$O$};
		\end{axis}
	\end{tikzpicture}
	\caption{}
\end{figure}

Найдём координаты точки $O$
\begin{align*}
	&\begin{cases}
		4x_1 + 3x_2 = 12, \\
		2x_1 + 5x_2 = 10; \quad \Big| \quad \cdot 2
	\end{cases} \\
	&\begin{cases}
		4x_1 + 3x_2 = 12, \\
		4x_1 + 10x_2 = 20;
	\end{cases}
\end{align*}

\[
\Downarrow
\]
\[
7x_2 = 8,
\]
\[
x_2 = \frac{8}{7};
\]
\[
x_1 = \frac{10}{2} - \frac{5}{2}x_2 = 5 - \frac{20}{7} = \frac{15}{7}.
\]

Таким образом, точка $O$ имеет координаты $(15/7, 8/7)$. Найдём $c'$ из соображения, что прямая $4x_1 + 5x_2 = c'$ проходит через точку $O$
\[
4x_1 + 5x_2 = c',
\]
\[
4 \cdot \frac{15}{7} + 5 \cdot \frac{8}{7} = c',
\]
\[
c' = \frac{100}{7}.
\]

То есть прямая $4x_1 + 5x_2 = 100/7$ пересекает точку $O$, что соответствует максимальному значению целевой функции $4x_1 + 5x_2$.

Таким образом
\[
\boxed{x_1^* = \frac{15}{7}, \qquad x_2^* = \frac{8}{7}, \qquad \max_{(x_1, x_2)} \big(4x_1 + 5x_2\big) = \frac{100}{7}}.
\]

\section{Теория двойственности}\label{sec:duality}

\definition

Пару задач будет называть \definitionfont{двойственной}, если задачи имеют следующий вид

\begin{itemize}
	\item[$\circled{1}$]
	
	\begin{alignat*}{2}
		& \sum_{j=1}^{n} c_j x_j \to \max_{(x_j)}, \\
		& \sum_{j=1}^{n}a_{ij} x_j \le b_i, && i = 1\dots m, \\
		& x_j \ge 0, && j = 1\dots n;		
	\end{alignat*}
	
	\item[$\circled{2}$]
	
	\begin{alignat*}{2}
		& \sum_{i=1}^{m} b_i y_i \to \min_{(y_i)}, \\
		& \sum_{i=1}^{m} a_{ij} y_i \ge c_j, && j = 1\dots n, \\
		& y_i \ge 0, && i = 1\dots m.		
	\end{alignat*}
\end{itemize}

Тогда задача $\circled{1}$ называется \definitionfont{прямой}, а $\circled{2}$ --- \definitionfont{двойственной}.

\begin{note}
	Для понимания можно привести аналогию. Пусть
	
	\begin{itemize}[nosep]
		\item $j = 1 \dots n$ --- вид производимой продукции,
		
		\item $i = 1 \dots m$ --- вид ресурса,
		
		\item $c_j$ --- доход от продажи продукции $j$-го типа,
		
		\item $a_{ij}$ --- расход $i$-го ресурса на производство единицы продукции $j$-го типа,
		
		\item $b_i$ --- запасы $i$-го ресурса.
	\end{itemize}
	
	Решение прямой задачи состоит в том, чтобы понять, сколько производить продукции каждого вида для максимизации прибыли ($x_j$ --- сколько единиц продукции $j$-го типа производить).
	
	Решение двойственной задачи состоит в том, чтобы понять, как минимизировать суммарную стоимость ресурсов так, чтобы доход от продажи продукции $c_j$ не превышал стоимость требуемых для неё ресурсов (иначе эту продукцию можно было бы производить и продавать до бесконечности). Таким образом, $y_i$ --- стоимость ресурса.
\end{note}

\example[составления двойственной задачи]

Составить двойственную задачу для

\[
\begin{linear}{crrrrrrrl}
	& 4x_1 & + & 5x_2 & + & 9x_3 & + & 11x_4 & \to \max\limits_{x}, \\
	& 1x_1 & + & 1x_2 & + & 1x_3 & + & 1x_4 & \le 15, \\
	& 7x_1 & + & 5x_2 & + & 3x_3 & + & 2x_4 & \le 120, \\
	& 3x_1 & + & 5x_2 & + & 10x_3 & + & 15x_4 & \le 100. \\
\end{linear}
\]

Прямая задача имеет $4$ переменные $x_1, x_2, x_3, x_4$ и $3$ ограничения, значит двойственная задача будет иметь $3$ переменные $y_1, y_2, y_3$ и $4$ ограничения.

Для удобства составления двойственной задачи поставим напротив каждого ограничения соответствующую переменную $y_i$
\[
\begin{linear}{crrrrrrrlc}
	& 1x_1 & + & 1x_2 & + & 1x_3 & + & 1x_4 & \le 15, \quad\quad &\big| \quad y_1 \\
	& 7x_1 & + & 5x_2 & + & 3x_3 & + & 2x_4 & \le 120, \quad &\big| \quad y_2 \\
	& 3x_1 & + & 5x_2 & + & 10x_3 & + & 15x_4 & \le 100. \quad &\big| \quad y_3 \\
\end{linear}
\]

Составим $4$ ограничения на переменные $y_1, y_2, y_3$: коэффициенты левой части ограничений будут соответствовать <<столбцу>> из ограничений на $x$, а правая часть будет соответствовать исходной целевой функции.
\[
\begin{linear}{crrrrrl}
	&1y_1 & + & 7y_2 & + & 3y_3 & \ge 4, \\
	&1y_1 & + & 5y_2 & + & 5y_3 & \ge 5, \\
	&1y_1 & + & 3y_2 & + & 10y_3 & \ge 9, \\
	&1y_1 & + & 2y_2 & + & 15y_3 & \ge 11, \\
\end{linear}
\]

Коэффициенты перед $y_1$ --- коэффициенты первого ограничения на $x$, перед $y_2$ --- второго ограничения на $x$ и так далее. Правая часть новых ограничений --- коэффициенты исходной целевой функции $4x_1 + 5x_2 + 9x_3 + 11x_4 \to \max$.

\implication

Если исходная задача имеет $2$ ограничения, то двойственная задача будет содержать лишь $2$ переменные, а значит её можно решить геометрически.

\fact

Если $x = (x_j)$ --- допустимое решение прямой задачи, а $y = (y_i)$ --- двойственной, то
\[
\sum_{j=1}^{n} c_j x_j \le \sum_{i=1}^{m} b_i y_i.
\]

\prooof

\begin{align*}
	\sum_{j=1}^{n} c_j x_j \; &\stackrel{(1)}{\le} \sum_{j=1}^{n} x_j \sum_{i=1}^{m} a_{ij} y_i \\
	&= \sum_{i=1}^{m} \sum_{j=1}^{n} a_{ij} x_j y_i \\
	&= \sum_{i=1}^{m} y_i \sum_{j=1}^{n} a_{ij} x_j \\
	& \stackrel{(2)}{\le} \sum_{i=1}^{m} y_i b_i,
\end{align*}

$(1)$ и $(2)$ --- определения двойственной и прямой задач соответственно:
\[
	c_j \le \sum_{i=1}^{m} a_{ij} y_i \qquad \sum_{j=1}^{n}a_{ij} x_j \le b_i. 
\]

\implication\label{impl:primal_dual}

\[
\sum_{j=1}^{n} c_j x_j \le \sum_{i=1}^{m} \sum_{j=1}^{n} a_{ij} x_j y_i \le \sum_{i=1}^{m} y_i b_i.
\]

\fact[критерий оптимальности решений]\label{fact:opt_solution_criterium}

Допустимые решения прямой и обратной задач являются оптимальными $\Longleftrightarrow$ выполняются равенства
\[
\sum_{j=1}^{n} c_j x_j = \sum_{i=1}^{m} \sum_{j=1}^{n} a_{ij} x_j y_i = \sum_{i=1}^{m} y_i b_i.
\]

\definition\label{def:complementary_slackness_conditions}

\definitionfont{Условиями дополняющей нежёсткости} будем называть следующие условия
\[
x_j \Big(\sum_{i=1}^{m} a_{ij} y_i - c_j\Big) = 0, \qquad j = 1 \dots n \tag{1}
\]
\[
y_i \Big(\sum_{j=1}^{n} a_{ij} x_j - b_i\Big) = 0. \qquad i = 1 \dots m \tag{2}
\]

\fact

Выполнение условий дополняющей нежёсткости эквивалентно равенству
\[
\sum_{j=1}^{n} c_j x_j = \sum_{i=1}^{m} b_i y_i.
\]

\prooof

\begin{itemize}[nosep]
	\item[]
	
	\item[\fbox{$\Rightarrow$}] Пусть оба условия выполняются. Просуммируем (1) по всем $j$
	\[
	\sum_{j=1}^{n} x_j \Big(\sum_{i=1}^{m} a_{ij} y_i - c_j\Big) = n \cdot 0 = 0,
	\]
	\[
	\sum_{j=1}^{n} x_j \Big(\sum_{i=1}^{m} a_{ij} y_i\Big) = \sum_{j=1}^{n} x_j c_j,
	\]
	\[
	\sum_{j=1}^{n} \sum_{i=1}^{m} a_{ij} x_j y_i = \sum_{j=1}^{n} x_j c_j,
	\]
	\[
	\sum_{i=1}^{m} \sum_{j=1}^{n} a_{ij} x_j y_i = \sum_{j=1}^{n} x_j c_j,
	\]
	\[
	\sum_{i=1}^{m} y_i \sum_{j=1}^{n} a_{ij} x_j = \sum_{j=1}^{n} x_j c_j,
	\]
	\[
	\sum_{i=1}^{m} y_i \Big(\sum_{j=1}^{n} a_{ij} x_j - b_i + b_i \Big) = \sum_{j=1}^{n} x_j c_j,
	\]
	\[
	\Downarrow \text{(2)}
	\]
	\[
	\sum_{i=1}^{m} y_i b_i = \sum_{j=1}^{n} x_j c_j.
	\]

	\item[\fbox{$\Leftarrow$}] Пусть решения $x = (x_j)$ и $y = (y_i)$ прямой и двойственной задач оптимальны. Тогда по \hyperref[fact:opt_solution_criterium]{Критерию оптимальности решений}
	\[
	\sum_{j=1}^{n} c_j x_j = \sum_{i=1}^{m} \sum_{j=1}^{n} a_{ij} x_j y_i = \sum_{i=1}^{m} y_i b_i.
	\]
	
	Рассмотрим одно равенство из двух
	\[
	\sum_{i=1}^{m} \sum_{j=1}^{n} x_j a_{ij} y_i = \sum_{j=1}^{n} x_j c_j,
	\]
	\[
	\sum_{j=1}^{n} \sum_{i=1}^{m} x_j a_{ij} y_i = \sum_{j=1}^{n} x_j c_j,
	\]
	\[
	\sum_{j=1}^{n} x_j \Big(\sum_{i=1}^{m} a_{ij} y_i - c_j\Big) = 0.
	\]

	По \hyperref[impl:primal_dual]{Следствию} все слагаемые неотрицательны, значит каждое из слагаемых равняется нулю, то есть
	\[
	x_j \Big(\sum_{i=1}^{m} a_{ij} y_i - c_j\Big) = 0, \qquad j = 1 \dots n.
	\]
	
	Действия аналогично со вторым равенством, можно доказать, что
	\[
	y_i \Big(\sum_{j=1}^{n} a_{ij} x_j - b_i\Big) = 0, \qquad i = 1 \dots m.
	\]
\end{itemize}

\implication

Выполнение условий дополняющей нежёсткости эквивалентно оптимальности решений прямой и двойственной задач.

\algorithm[решения задач с использованием теории двойственности]

\textbf{Исходные данные}: $n > 2$, но $m = 2$.

\begin{enumerate}[nosep]
	\item По прямой задаче строим двойственную, в которой будет ровно $2$ переменные.
	
	\item Решаем двойственную задачу геометрически, находя её оптимальное решение $(y_1^*, y_2^*)$.

	\item Подставляем $y_1^*$ и $y_2^*$ в первое условие дополняющей нежёсткости. В двух ограничениях двойственной задачи на оптимальном решении будет выполняться равенство, а во всех остальных равенства не будет; пусть равенство выполнено при $j = j_1$ и $j = j_2$.
	
	Если $j \neq j_1$ и $j \neq j_2$, то для выполнения первого условия дополняющей нежёсткости необходимо, чтобы $x_j^* = 0$, значит
	\[
	x_j^* = \begin{cases}
		0, & j \neq j_1 \land j \neq j_2 \\
		\dots, & \dots.
	\end{cases}
	\]
	
	\item Подставляем во второе условие дополняющей нежёсткости значения $(x_j^*)$, из предыдущего пункта следует, что все значения кроме $x_{j_1}^*$ и $x_{j_2}^*$ равняются нулю. После подстановки будет система линейных уравнений с двумя неизвестными
	\[
	\begin{cases}
		y_1^* (a_{1 j_1} x_{j_1}^* - b_1) = 0, \\
		y_2^* (a_{2 j_2} x_{j_2}^* - b_2) = 0;
	\end{cases}
	\]
	
	решением системы будут значения $x_{j_1}^*$ и $x_{j_2}^*$. 
	
	\item Оптимальное решение прямой задачи имеет вид
	\[
	x^* = (0, \dots 0, x_{j_1}^*, 0, \dots, 0, x_{j_2}^*, 0, \dots, 0).
	\]
\end{enumerate}

\section{Практика применения геометрического метода и теории двойственности}

\problem

Найти $x_1$ и $x_2$, удовлетворяющие следующим условиям:
\[
\begin{linear}{crrrrl}
	& 4x_1 & + & 5x_2 & \to & \max\limits_{(x_1, x_2)}, \\
	& 4x_1 & + & 3x_2 & \ge & 12, \\
	& 2x_1 & + & 5x_2 & \le & 10,
\end{linear}
\]

\solution

Решим данную задачу с помощью \hyperref[sec:geom_method]{Геометрического метода}.

Изобразим на плоскости прямые, соответствующие ограничениям, точку их пересечения $O$ и множество допустимых решений (рис 4.4).

\begin{figure}[H]
	\centering
	\begin{tikzpicture}
		\begin{axis}[axis lines=middle, xlabel={$x_1$}, ylabel={$x_2$}, grid=major, width=0.7\textwidth, xmin=-1, xmax=6, ymin=-2, ymax=5]
			\addplot[domain=-1:6, samples=2, blue, thick]{-4/3 * x + 12/3};
			\addlegendentry{$4x_1 + 3x_2 = 12$}
			
			\addplot[domain=-1:6, samples=2, red, thick]{-2/5 * x + 10/5};
			\addlegendentry{$2x_1 + 5x_2 = 10$}
			
			\addplot[pattern=north east lines]
			coordinates {
				(15/7,8/7)
				(3,0)
				(5,0)
			};
			
			\addplot[only marks, mark=* ,] coordinates {(15/7, 8/7)};
			\node at (axis cs:15/7,8/7) [anchor=south west]{$O$};
		\end{axis}
	\end{tikzpicture}
	\caption{}
\end{figure}

Нарисуем прямую $4x_1+ 5x_2 = 10$ $(c=10)$ и посмотрим, как будет располагаться относительно уже нарисованного (рис. 4.5).

\begin{figure}[H]
	\centering
	\begin{tikzpicture}
		\begin{axis}[axis lines=middle, xlabel={$x_1$}, ylabel={$x_2$}, grid=major, width=0.7\textwidth, xmin=-1, xmax=6, ymin=-2, ymax=5]
			\addplot[domain=-1:6, samples=2, blue, thick]{-4/3 * x + 12/3};
			\addlegendentry{$4x_1 + 3x_2 = 12$}
			
			\addplot[domain=-1:6, samples=2, red, thick]{-2/5 * x + 10/5};
			\addlegendentry{$2x_1 + 5x_2 = 10$}
			
			\addplot[domain=-1:6, samples=2, green, thick]{-4/5 * x + 2};
			\addlegendentry{$4x_1 + 5x_2 = 10$}
			
			\addplot[pattern=north east lines]
			coordinates {
				(15/7,8/7)
				(3,0)
				(5,0)
			};
			
			\addplot[only marks, mark=* ,] coordinates {(15/7, 8/7)};
			\node at (axis cs:15/7,8/7) [anchor=south west]{$O$};
		\end{axis}
	\end{tikzpicture}
	\caption{}
\end{figure}

Параллельно передвигая эту прямую, можно понять, что максимальное $c$, при котором прямая пересекает область допустимых решений, это $c = 20$, при котором прямая проходит через точку $(5, 0)$ (рис. 4.6).

\begin{figure}[H]
	\centering
	\begin{tikzpicture}
		\begin{axis}[axis lines=middle, xlabel={$x_1$}, ylabel={$x_2$}, grid=major, width=0.7\textwidth, xmin=-1, xmax=6, ymin=-2, ymax=5]
			\addplot[domain=-1:6, samples=2, blue, thick]{-4/3 * x + 12/3};
			\addlegendentry{$4x_1 + 3x_2 = 12$}
			
			\addplot[domain=-1:6, samples=2, red, thick]{-2/5 * x + 10/5};
			\addlegendentry{$2x_1 + 5x_2 = 10$}
			
			\addplot[domain=-1:6, samples=2, green, thick]{-4/5 * x + 4};
			\addlegendentry{$4x_1 + 5x_2 = 20$}
			
			\addplot[pattern=north east lines]
			coordinates {
				(15/7,8/7)
				(3,0)
				(5,0)
			};
			
			\addplot[only marks, mark=* ,] coordinates {(15/7, 8/7)};
			\node at (axis cs:15/7,8/7) [anchor=south west]{$O$};
		\end{axis}
	\end{tikzpicture}
	\caption{}
\end{figure}

\textbf{Ответ}: $x^*=(5, 0)$, $\max = 20$.

\problem

Найти $x_1, \cdots, x_5$, удовлетворяющие следующим условиям:

\[
\begin{linear}{crrrrrrrrrrl}
	& -10x_1 & + & 24x_2 & + & 20x_3 & + & 20x_4 & + & 25x_5 & \to & \max\limits_{(x_1, \cdots, x_5)}, \\
	& -x_1 & + & x_2 & + & 2x_3 & + & 3x_4 & + & 5x_5 & \le & 19, \\
	& -x_1 & + & 4x_2 & + & 3x_3 & + & 2x_4 & + & x_5 & \le & 57. \\
\end{linear}
\]

\solution

Для решения данной задачи составим \hyperref[sec:duality]{двойственную} ей задачу.
\[
\begin{linear}{crrrrrrrrrrlc}
	& -x_1 & + & x_2 & + & 2x_3 & + & 3x_4 & + & 5x_5 & \le & 19 \quad\quad & \big| \quad y_1, \\
	& -x_1 & + & 4x_2 & + & 3x_3 & + & 2x_4 & + & x_5 & \le & 57 \quad\quad & \big| \quad y_2. \\
\end{linear}
\]
 
Составим ограничения на переменные $y_1, y_2$
\[
\begin{linear}{crrrrl}
	& 19y_1 & + & 57y_2 & \to & \max\limits_{(y_1, y_2)}, \\
	& -y_1 & + & -y_2 & \ge & -10, \\
	& y_1 & + & 4y_2 & \ge & 24, \\
	& 2y_1 & + & 3y_2 & \ge & 20, \\
	& 3y_1 & + & 2y_2 & \ge & 20, \\
	& 5y_1 & + & y_2 & \ge & 25. \\
\end{linear}
\]

Решим двойственную задачу с помощью \hyperref[sec:geom_method]{Геометрического метода}.

Изобразим на плоскости прямые, соответствующие ограничениям и множество допустимых решений (рис. 4.7).

\begin{figure}[H]
	\centering
	\begin{tikzpicture}
		\begin{axis}[axis lines=middle, xlabel={$y_1$}, ylabel={$y_2$}, grid=major, width=0.7\textwidth, xmin=-1, xmax=11, ymin=-2, ymax=11]
			\addplot[domain=-1:11, samples=2, blue, thick]{-x + 10};
			\addlegendentry{$-y_1 - y_2 = 10$}
			
			\addplot[domain=-1:11, samples=2, red, thick]{-1/4 * x + 6};
			\addlegendentry{$y_1 + 4y_2 = 24$}
			
			\addplot[domain=-1:11, samples=2, green, thick]{-2/3 * x + 20/3};
			\addlegendentry{$2y_1 + 3y_2 = 20$}
			
			\addplot[domain=-1:11, samples=2, cyan, thick]{-3/2 * x + 10};
			\addlegendentry{$3y_1 + 2y_2 = 20$}
			
			\addplot[domain=-1:11, samples=2, black, thick]{-5 * x + 25};
			\addlegendentry{$5y_1 + 1y_2 = 25$}
			
			\addplot[pattern=north east lines]
			coordinates {
				(4,5)
				(3.75,6.25)
				(16/3,14/3)
			};
			
			\addplot[only marks, mark=* ,] coordinates {(4, 5)};
			\node at (axis cs:4,5) [anchor=north west]{$A$};
			\addplot[only marks, mark=* ,] coordinates {(3.75, 6.25)};
			\node at (axis cs:3.75,6.25) [anchor=south west]{$B$};
			\addplot[only marks, mark=* ,] coordinates {(16/3, 14/3)};
			\node at (axis cs:16/3,14/3) [anchor=south west]{$C$};
		\end{axis}
	\end{tikzpicture}
	\caption{}
\end{figure}

Областью допустимых решений оказался треугольник $ABC$.

Нарисуем прямую $19y_1 + 57y_2 = 150 \; (c=150)$ (рис. 4.8).

\begin{figure}[H]
	\centering
	\begin{tikzpicture}
		\begin{axis}[axis lines=middle, xlabel={$y_1$}, ylabel={$y_2$}, grid=major, width=0.7\textwidth, xmin=-1, xmax=11, ymin=-2, ymax=11]
			
			\addplot[domain=-1:11, samples=2, red, thick]{-1/3 * x + 150/57};
			\addlegendentry{$19y_1 + 57y_2 = 150$}
			
			\addplot[pattern=north east lines]
			coordinates {
				(4,5)
				(3.75,6.25)
				(16/3,14/3)
				(4,5)
			};
			
			\addplot[only marks, mark=* ,] coordinates {(4, 5)};
			\node at (axis cs:4,5) [anchor=north west]{$A$};
			\addplot[only marks, mark=* ,] coordinates {(3.75, 6.25)};
			\node at (axis cs:3.75,6.25) [anchor=south west]{$B$};
			\addplot[only marks, mark=* ,] coordinates {(16/3, 14/3)};
			\node at (axis cs:16/3,14/3) [anchor=south west]{$C$};
		\end{axis}
	\end{tikzpicture}
	\caption{}
\end{figure}

Параллельно передвигая эту прямую, можно понять, что минимальное $c$, при котором прямая пересекает область допустимых решений, это $c$, при котором прямая проходит через точку $A$. Найдем её координаты, пользуясь тем, что эта точка --- пересечение прямых $y_1 + 4y_2 = 24$ и $5y_1 + y_2 = 25$.
\begin{align*}
	&\begin{cases}
		y_1 + 4y_2 = 24, \\
		5y_1 + y_2 = 25; \quad \Big| \quad \cdot 4
	\end{cases} \\
	&\begin{cases}
		y_1 + 4y_2 = 24, \\
		20y_1 + 4y_2 = 100;
	\end{cases}
\end{align*}
\[
\Downarrow
\]
\[
-19y_1 = -76,
\]
\[
y_1 = 4;
\]
\[
y_2 = 4 \cdot 5 + y_2 - 25 = 5.
\]

Таким образом, точка $A$ имеет координаты $(4, 5)$. Найдём $c'$ из соображения, что прямая $19y_1 + 57y_2 = c'$ проходит через точку $A$:
\[
19y_1 + 57y_2 = c',
\]
\[
19 \cdot 4 + 57 \cdot 5 = c',
\]
\[
c' = 361.
\]

Данный результат еще не дает ответ, так как необходимо найти $x_1 \cdots x_5$. Для этого воспользуемся \hyperref[def:complementary_slackness_conditions]{Соотношениями дополняющей нежёсткости}.

\bigskip

\textbf{Первое соотношение}
\[
\begin{linear}{crrl}
	& x_1(y_1 - y_2 - 10) & = & 0,  \\
	& x_2(y_1 + 4y_2 - 24) & = & 0,  \\
	& x_3(y_1 + 3y_2 - 20) & = & 0,  \\
	& x_4(y_1 + 2y_2 - 20) & = & 0,  \\
	& x_5(y_1 + y_2 - 25) & = & 0.  \\
\end{linear}
\]

Подставим найденные $y_1, y_2$ и получим

\[
\begin{linear}{crrl}
	& x_1 \cdot (-1) & = & 0,  \\
	& x_2 \cdot (0) & = & 0,  \\
	& x_3 \cdot (-2) & = & 0,  \\
	& x_4 \cdot (2) & = & 0,  \\
	& x_5 \cdot (0) & = & 0.  \\
\end{linear}
\]

Из этого следует

\[
\begin{linear}{crrl}
	& x_1 & = & 0,  \\
	& x_2 & \ge & 0,  \\
	& x_3 & = & 0,  \\
	& x_4 & = & 0,  \\
	& x_5 & \ge & 0.  \\
\end{linear}
\]

\begin{note}
	Все $x_j = 0$, кроме тех $j$, номера прямых которых в пересечении дали точку $A$, это были прямые $2, 5$ (по соответствующим позициям в системе неравенств на $y_1, y_2$).  
\end{note}

\bigskip

\textbf{Второе соотношение}
\[
\begin{linear}{crrl}
	& y_1(-x_1 + x_2 + 2x_3 + 3x_4 + 5x_5 - 19) & = & 0,  \\
	& y_2(-x_1 + 4x_2 + 3x_3 + 2x_4 + x_5 - 57) & = & 0.  \\
\end{linear}
\]

Применим результат, который получился при применении первого соотношения (удалим все $x_j = 0$).

\[
\begin{linear}{crrl}
	& y_1(x_2 + 5x_5 - 19) & = & 0,  \\
	& y_2(4x_2 + x_5 - 57) & = & 0.  \\
\end{linear}
\]

Найдём $x_2, x_5$:

\begin{align*}
	&\begin{cases}
		x_2 + 5x_5 = 19, \quad \Big| \quad \cdot 4 \\
		4x_2 + x_5 = 57;
	\end{cases} \\
	&\begin{cases}
		4x_2 + 20x_5 = 76, \\
		4x_2 + x_5 = 57;
	\end{cases}
\end{align*}

\[
\Downarrow
\]
\[
19x_5 = 19,
\]
\[
x_5 = 1;
\]
\[
x_2 = 19 - 4 \cdot 1 = 14.
\]

Найдем значение целевой функции при $x = (x_1, \dots, x_5) = (0, 14, 0, 0, 1)$:
\[
-10 \cdot 0 + 24 \cdot 14 + 20 \cdot 0 + 20 \cdot 0 + 25 \cdot 1 = 361 = c',
\]

всё совпало.

\textbf{Ответ}: $x^* = (0, 14, 0, 0, 1)$, $\max = 361$.

\bigskip

\problem

Найти $x_1, \cdots, x_4$, удовлетворяющие следующим условиям:
\[
\begin{linear}{crrrrrrrrl}
	& -10x_1 & + & 24x_2 & + & 20x_3 & + & 20x_4 & \to & \max\limits_{(x_1, \cdots, x_5)}, \\
	& -x_1 & + & x_2 & + & 2x_3 & + & 3x_4 & \le & 19, \\
	& -x_1 & + & 4x_2 & + & 3x_3 & + & 2x_4 & \le & 57. \\
\end{linear}
\]

\solution

Для решения данной задачи составим \hyperref[sec:duality]{двойственную} ей задачу.
\[
\begin{linear}{crrrrrrrrrrlc}
	& -x_1 & + & x_2 & + & 2x_3 & + & 3x_4 & \le & 19 \quad\quad & \big| \quad y_1, \\
	& -x_1 & + & 4x_2 & + & 3x_3 & + & 2x_4 & \le & 57 \quad\quad & \big| \quad y_2. \\
\end{linear}
\]

Составим ограничения на переменные $y_1, y_2$
\[
\begin{linear}{crrrrl}
	& 19y_1 & + & 57y_2 & \to & \max\limits_{(y_1, y_2)}, \\
	& -y_1 & + & -y_2 & \ge & -10, \\
	& y_1 & + & 4y_2 & \ge & 24, \\
	& 2y_1 & + & 3y_2 & \ge & 20, \\
	& 3y_1 & + & 2y_2 & \ge & 20, \\
\end{linear}
\]

Решим данную задачу с помощью \hyperref[sec:geom_method]{Геометрического метода}.

Изобразим на плоскости прямые, соответствующие ограничениям и множество допустимых решений (рис. 4.9).

\begin{figure}[H]
	\centering
	\begin{tikzpicture}
		\begin{axis}[axis lines=middle, xlabel={$y_1$}, ylabel={$y_2$}, grid=major, width=0.7\textwidth, xmin=-1, xmax=11, ymin=-2, ymax=11]
			\addplot[domain=-1:11, samples=2, blue, thick]{-x + 10};
			\addlegendentry{$-y_1 - y_2 = 10$}
			
			\addplot[domain=-1:11, samples=2, red, thick]{-1/4 * x + 6};
			\addlegendentry{$y_1 + 4y_2 = 24$}
			
			\addplot[domain=-1:11, samples=2, green, thick]{-2/3 * x + 20/3};
			\addlegendentry{$2y_1 + 3y_2 = 20$}
			
			\addplot[domain=-1:11, samples=2, cyan, thick]{-3/2 * x + 10};
			\addlegendentry{$3y_1 + 2y_2 = 20$}
			
			\addplot[pattern=north east lines]
			coordinates {
				(0,10)
				(16/3,14/3)
				(3.2,5.2)
				(0,10)
			};
			
			\addplot[only marks, mark=* ,] coordinates {(0, 10)};
			\node at (axis cs:0,10) [anchor=north east]{$A$};
			\addplot[only marks, mark=* ,] coordinates {(16/3,14/3)};
			\node at (axis cs:16/3,14/3) [anchor=south west]{$B$};
			\addplot[only marks, mark=* ,] coordinates {(3.2,5.2)};
			\node at (axis cs:3.2,5.2) [anchor=south west]{$C$};
		\end{axis}
	\end{tikzpicture}
	\caption{}
\end{figure}

Областью допустимых решений оказался треугольник $ABC$.

Нарисуем прямую $19y_1 + 57y_2 = 150 \; (c=150)$ (рис. 4.10).

\begin{figure}[H]
	\centering
	\begin{tikzpicture}
		\begin{axis}[axis lines=middle, xlabel={$y_1$}, ylabel={$y_2$}, grid=major, width=0.7\textwidth, xmin=-1, xmax=11, ymin=-2, ymax=11]
			
			\addplot[domain=-1:11, samples=2, red, thick]{-1/3 * x + 150/57};
			\addlegendentry{$19y_1 + 57y_2 = 150$}
			
		\addplot[pattern=north east lines]
		coordinates {
			(0,10)
			(16/3,14/3)
			(3.2,5.2)
			(0,10)
		};
		
		\addplot[only marks, mark=* ,] coordinates {(0, 10)};
		\node at (axis cs:0,10) [anchor=north east]{$A$};
		\addplot[only marks, mark=* ,] coordinates {(16/3,14/3)};
		\node at (axis cs:16/3,14/3) [anchor=south west]{$B$};
		\addplot[only marks, mark=* ,] coordinates {(3.2,5.2)};
		\node at (axis cs:3.2,5.2) [anchor=south west]{$C$};
		\end{axis}
	\end{tikzpicture}
	\caption{}
\end{figure}

Параллельно передвигая эту прямую, можно понять, что минимальное $c$, при котором прямая пересекает область допустимых решений, это $c$, при котором прямая проходит через точку $C$. Найдем её координаты, пользуясь тем, что эта точка --- пересечение прямых $y_1 + 4y_2 = 24$ и $3y_1 + 2y_2 = 20$.
\begin{align*}
	&\begin{cases}
		y_1 + 4y_2 = 24, \\
		3y_1 + 2y_2 = 20; \quad \Big| \quad \cdot 2
	\end{cases} \\
	&\begin{cases}
		y_1 + 4y_2 = 24, \\
		6y_1 + 4y_2 = 40;
	\end{cases}
\end{align*}
\[
\Downarrow
\]
\[
5y_1 = 16,
\]
\[
y_1 = 3.2;
\]
\[
y_2 = \frac{24 - 3.2}{4} = 5.2.
\]

Таким образом, точка $C$ имеет координаты $(3.2, 5.2)$. Найдём $c'$ из соображения, что прямая $19y_1 + 57y_2 = c'$ проходит через точку $C$:
\[
19y_1 + 57y_2 = c',
\]
\[
19 \cdot 3.2 + 57 \cdot 5.2 = c',
\]
\[
c' = 357.2.
\]

Воспользуемся \hyperref[def:complementary_slackness_conditions]{Соотношениями дополняющей нежёсткости}.

\bigskip

\textbf{Первое соотношение}

Зная, что точка $C$ это --- пересечение прямых с номерами $2, 4$, сразу получаем
\[
\begin{linear}{crrl}
	& x_1 & = & 0,  \\
	& x_2 & \ge & 0,  \\
	& x_3 & = & 0,  \\
	& x_4 & \ge & 0,  \\
\end{linear}
\]

\textbf{Второе соотношение}

Сразу напишем его с учётом результатов первого соотношения.
\[
\begin{linear}{crrl}
	& y_1(x_2 + 3x_4 - 19) & = & 0,  \\
	& y_2(4x_2 + 2x_4 - 57) & = & 0.  \\
\end{linear}
\]

Найдём $x_2, x_4$:
\begin{align*}
	&\begin{cases}
		x_2 + 3x_4 = 19, \quad \Big| \quad \cdot 4 \\
		4x_2 + 2x_4 = 57;
	\end{cases} \\
	&\begin{cases}
		4x_2 + 12x_4 = 76, \\
		4x_2 + 2x_4 = 57;
	\end{cases}
\end{align*}
\[
\Downarrow
\]
\[
10x_4 = 19,
\]
\[
x_4 = 1.9;
\]
\[
x_2 = 19 - 3 \cdot 1.9 = 13.3.
\]

Найдем значение целевой функции при $x = (x_1, \dots, x_4) = (0, 13.3, 0, 1.9)$
\[
-10 \cdot 0 + 24 \cdot 13.3 + 20 \cdot 0 + 20 \cdot 1.9 = 357.2 = c',
\]

всё совпало.

\textbf{Ответ}: $x^* = (0, 13.3, 0, 1.9)$, $\max = 357.2$.

\section{Симплекс-метод}\label{sec:simplex_algorithm}

Симплекс-метод позволяет решать задачи линейного программирования с произвольным числом переменных и ограничений.

\definition

Ограничения задачи линейного программирования записаны \definitionfont{в виде равенств}, если они имеют вид
\[
\sum_{j=1}^{n} x_j a_{ij} + \underbrace{y_i}_{\ge 0} = b_i, \quad i = 1 \dots m.
\]

\remark

В случае поиска $\min$ в равенствах должно стоять $-y_i$, где $y_i \ge 0$.

\algorithm[симплекс-метод]

\textbf{Идея}

Метод заключается в представлении множества допустимых решений через базисные переменные. Если в задаче $m$ ограничений, то необходимо $m$ базисных переменных. Везде далее для простоты будем считать, что базисные переменные --- это $\{x_1, x_2,\dots, x_m\}$, хотя на самом деле это могут быть произвольные переменные
$\{x_{j_1}, x_{j_2}, \dots, x_{j_m}\}$ и даже $\{y_1, y_2, \dots, y_m\}$.

Базисные переменные нужно представить в виде
\[
x_1 = b_1 + l_1(x_{m+1}, x_{m+2}, \dots, x_n),
\]
\[
x_1 = b_2 + l_2(x_{m+1}, x_{m+2}, \dots, x_n),
\]
\[
\dots
\]
\[
x_m = b_m + l_m(x_{m+1}, x_{m+2}, \dots, x_n),
\]

где $l_i$ --- линейная функция, то есть все базисные переменные должны выражаться через линейные комбинации не базисных и свободный член. Найти такое представление базисных переменных можно из исходных ограничений с помощью метода Гаусса (сначала исключаем $x_1$, затем $x_2$, $x_3$ и так далее).

\bigskip

\textbf{Алгоритм}

\begin{itemize}[nosep]
	\renewcommand{\labelitemii}{\textbullet}
	  
	\item[\fbox{Шаг 0}]
	
	\begin{enumerate}[nosep]
		\item[]
		
		\item Выбираем базисные переменные и записываем их представление через не базисные;
		
		\item Записываем целевую функцию $x_0$ через не базисные переменные
		\[
		x_0 = b_0 + l_0(x_{m+1}, x_{m+2}, \dots, x_{m}),
		\]
		
		где $l_0$ --- линейная функция.
		
		\item Переходим к шагу 1.
	\end{enumerate}
		
	\item[\fbox{Шаг $i$}]
	
	\begin{enumerate}[nosep]
		\item[] 
		
		\item Решение на текущем шаге --- занулить все не базисные переменные, то есть
		\[
		x_i = \begin{cases}
			b_i, & i \le m, \\
			0, & i > m.
		\end{cases}
		\]
		
		\item Необходимо определить, есть ли в представлении $x_0$ положительные коэффициенты перед не базисными переменными.
		
		\begin{note}
			\begin{itemize}
				\item Если перед некоторым $x_j$ коэффициент $>0$, то значение целевой функции можно увеличить, если $x_j > 0$, поэтому текущее решение не будет оптимальным.
				
				\item Если в $x_0$ все коэффициенты $\le0$, то текущее решение является оптимальным (\underline{критерий оптимальности решения}).
			\end{itemize}
		\end{note}
		
		\item Если решение оптимально, то завершаем алгоритм.
		
		\item Если решение не оптимально, то есть перед некоторым $x_j$ коэффициент $>0$, то нужно добавить переменную $x_j$ в базис, удалив из него другую. Переменную для удаления $x_k$ нужно выбрать так, чтобы в её представлении коэффициент перед $x_j$ был $< 0$.
		
		\begin{note}
			\begin{itemize}[nosep]
				\item Если таких кандидатов несколько, то необходимо выбрать такую переменную $x_k$, в которой
				\[
				\abs{\frac{b_k}{\beta_{kj}}} \to \min,
				\]
				
				где $\beta_{kj}$ --- коэффициент перед $x_j$ в функции $l_k$.
				
				\item Если таких кандидатов нет, то при $x_j \to \infty$, все базисные переменные $\to \infty$, а значит целевая функция $\to \infty$, то есть задача не ограничена и не имеет решений.
			\end{itemize}
		\end{note}
		
		\item После изменения базиса записываем базисные переменные и целевую функцию через не базисные и переходим к следующему шагу $i+1$.
	\end{enumerate} 
\end{itemize}	

\remark

Если задача на $\min$, то решение оптимально $\Longleftrightarrow$ в $x_0$ все коэффициенты $\ge 0$.

\remark

В начальный базис лучше не брать переменные $y_i$, потому что на первом шаге решение не будет оптимальным и базис придётся менять.

\remark

Если на очередном шаге выбрать кандидата на удаление из базиса не по правила алгоритма, то в выражении базисных переменных появятся отрицательные свободные члены.

\example[решения задачи с помощью симплекс-метода]

Решить задачу линейного программирования
\[
\begin{linear}{crrrrrrrl}
	& 4x_1 & + & 5x_2 & + & 9x_3 & + & 11x_4 & \to \max\limits_{x}, \\
	& 1x_1 & + & 1x_2 & + & 1x_3 & + & 1x_4 & \le 15, \\
	& 7x_1 & + & 5x_2 & + & 3x_3 & + & 2x_4 & \le 120, \\
	& 3x_1 & + & 5x_2 & + & 10x_3 & + & 15x_4 & \le 100. \\
\end{linear}
\]

Запишем задачу через равенство с переменными $y_i$
\[
\begin{linear}{crrrrrrrrrrrrrrrl}
	& \max &\big(&4x_1 & + & 5x_2 & + & 9x_3  & + & 11x_4&\big), \\
	&&& 1x_1 & + & 1x_2 & + & 1x_3 & + & 1x_4 & + & 1y_1 &&&&& = 15, \\
	&&&  7x_1 & + & 5x_2 & + & 3x_3 & + & 2x_4 &&& + & 1y_2 &&& = 120, \\
	&&&  3x_1 & + & 5x_2 & + & 10x_3 & + & 15x_4 &&&&& + & 1y_3 & = 100. \\
\end{linear}
\]

\begin{enumerate}
	\item[\fbox{Шаг 0}] Пусть начальный базис --- $\{x_4, y_1, y_2\}$. Запишем выражение для базисных переменных
	\[
	\begin{linear}{rrrrrrrrrrrrrr}
		& y_1 & = & 15 & - & 1x_1 & - & 1x_2 & - & 1x_3 & - & 1x_4, && \\
		& y_2 & = & 120 & - & 7x_1 & - & 5x_2 & - & 3x_3 & - & 2x_4, && \\
		& x_4 & = & \frac{20}{3} & - & \frac{1}{5}x_1 & - & \frac{1}{3}x_2 & - & \frac{2}{3}x_3 &&& - & \frac{1}{15}y_3. \\
	\end{linear}
	\]
	
	Видно, что в выражениях для $y_1$ и $y_2$ есть базисная переменная $x_4$, от которой в них нужно избавиться.
	
	
	Исключим $x_4$ из $y_1$
	\[
	\begin{linear}{rrrrrrrrrrrrrrrrrrr}
		& y_1 = &&& 15 & - & 1x_1 & - & 1x_2 & - & 1x_3 & - & \cancel{1x_4} &&&&+ \\
		&& -&(&\frac{20}{3} & - & \frac{1}{5}x_1 & - & \frac{1}{3}x_2 & - & \frac{2}{3}x_3 &&& - & \frac{1}{15}y_3 &) &= \\
		&&&& \frac{25}{3} & - & \frac{4}{5}x_1 & - & \frac{2}{3}x_2 & - & \frac{1}{3}x_3 &&& + & \frac{1}{15}y_3 &.& \\
	\end{linear}
	\]
	
	Исключим $x_4$ из $y_2$
	\[
	\begin{linear}{rrrrrrrrrrrrrrrrrrr}
		& y_2 = &&& 120 & - & 7x_1 & - & 5x_2 & - & 3x_3 & - & \cancel{2x_4} &&&&+ \\
		&& -2&(&\frac{20}{3} & - & \frac{1}{5}x_1 & - & \frac{1}{3}x_2 & - & \frac{2}{3}x_3 &&& - & \frac{1}{15}y_3 &) &= \\
		&&&& \frac{320}{3} & - & \frac{33}{5}x_1 & - & \frac{13}{3}x_2 & - & \frac{5}{3}x_3 &&& + & \frac{2}{15}y_3 &.& \\
	\end{linear}
	\]
	
	Теперь все базисные переменные $y_1, y_2, x_4$ выражаются через не базисные $x_1, x_2, x_4$.
	
	Запишем целевую функцию $x_0$ через не базисные переменные
	\[
	\begin{linear}{rrrrrrrrrrrrrrrrrrr}
		& x_0 = &&&&& 4x_1 & + & 5x_2 & + & 9x_3 & + & \cancel{11x_4} &&&&+ \\
		&& 11&(&\frac{20}{3} & - & \frac{1}{5}x_1 & - & \frac{1}{3}x_2 & - & \frac{2}{3}x_3 &&& - & \frac{1}{15}y_3 &) &= \\
		&&&& \frac{220}{3} & + & \frac{9}{5}x_1 & + & \frac{4}{3}x_2 & + & \frac{5}{3}x_3 &&& - & \frac{11}{15}y_3 &.& \\
	\end{linear}
	\]
	
	Перейдём к следующей итерации.
	
	\item[\fbox{Шаг 1}] Текущее решение --- занулить все не базисные переменные, то есть
	\[
	x_1 = x_2 = x_3 = 0, \quad x_4 = \frac{20}{3}, \quad y_1 = \frac{25}{3}, \quad y_2 = \frac{320}{3}, \quad x_0 = \frac{220}{3}.
	\]
	
	Значение целевой функции на базисном решение --- $\frac{220}{3}$. Текущее решение не является оптимальным, потому что в выражении целевой функции не все коэффициенты черед не базисными переменными меньше нуля ($\frac{9}{5}, \frac{4}{3}, \frac{5}{3} \nless 0$). Из этого следует, например, что если взять $x_1 > 0$, то значение целевой функции станет больше $\frac{220}{3}$, то есть оно увеличится несмотря на то что в задаче целевая функция максимизируется. Значит целевую функцию можно сделать ещё больше, поэтому текущее решение действительно не оптимально.
	
	Для увеличение целевой функции нужно поменять базис. Коэффициент перед $x_1$ в выражении $x_0$ не меньше нуля, поэтому добавим $x_1$ в базис.
	
	\begin{note}
		Вместо $x_1$ в базис можно было добавить $x_2$ и $x_3$, поскольку коэффициенты перед ними в целевой функции так же больше нуля.
	\end{note}
	
	Для добавления $x_1$ в базис нужно удалить из него переменную. Чтобы выбрать переменную на удаление из базиса, нужно посмотреть, в каких базисных переменных коэффициент перед $x_1$ меньше нуля. Таким образом, имеем трёх кандидатов на удаление из базиса: $y_1, y_2, x_4$. Чтобы выбрать одного из них, посчитаем соотношение свободных членов к коэффициенту перед $x_1$
	\[
	y_1: \quad \frac{\frac{25}{3}}{\frac{4}{5}} = \frac{125}{10},
	\]
	\[
	y_2: \quad \frac{\frac{320}{3}}{\frac{33}{5}} = \frac{1600}{99},
	\]
	\[
	x_4: \quad \frac{\frac{20}{3}}{\frac{1}{5}} = \frac{100}{3}.
	\]
	
	Минимальное из этих чисел --- $\frac{125}{10}$, поэтому удалим из базиса $y_1$, добавив в него $x_1$. Таким образом, текущий базис --- $\{x_1, x_4, y_2\}$. Напишем выражение для $x_1$ через не базисные переменные
	\[
	\begin{linear}{rrrrrrrrrrrr}
		& \frac{4}{5}x_1 & = & \frac{25}{3} & - & \frac{2}{3}x_2 & - & \frac{1}{3}x_3 & - & 1y_1 & + & \frac{1}{15}y_3, \\
		&x_1 & = & \frac{125}{12} & - & \frac{5}{6}x_2 & - & \frac{5}{12}x_3 & - & \frac{5}{4}y_1 & + & \frac{1}{12}y_3.
	\end{linear}
	\]
	
	Заметим, что в выражениях для $y_2$ и $x_4$ есть $x_1$, однако в них должны быть лишь не базисные переменные. Значит в этих выражениях нужно избавиться от $x_1$.
	
	Исключим $x_1$ из $y_2$
	\[
	\begin{linear}{rrrrrrrrrrrrrrrrr}
		& y_2 = &&& \frac{320}{3} & - & \cancel{\frac{33}{5}x_1} & - & \frac{13}{3}x_2 & - & \frac{5}{3}x_3 &&& + & \frac{25}{15}y_3 && + \\
		&& -\frac{33}{5}&(&\frac{125}{12} &&& - & \frac{5}{6}x_2 & - & \frac{5}{12}x_3 & - & \frac{5}{4}y_1 & + & \frac{1}{12}y_3 &) &= \\
		&&&& \frac{455}{12} &&& + & \frac{7}{6}x_2 & + & \frac{13}{12}x_3 & + & \frac{33}{4}y_1& - & \frac{5}{12}y_2 &.& \\
	\end{linear}
	\]
	
	Аналогично исключим $x_1$ из $x_4$
	\[
	\begin{linear}{rrrrrrrrrrrrrrrrrr}
		& x_4 = &&& \frac{20}{3} & - & \cancel{\frac{1}{5}x_1} & - & \frac{1}{3}x_2 & - & \frac{2}{3}x_3 &&& - & \frac{1}{15}y_3 && + \\
		&& -\frac{1}{5}&(&\frac{125}{12} &&& - & \frac{5}{6}x_2 & - & \frac{5}{12}x_3 & - & \frac{5}{4}y_1 & + & \frac{1}{12}y_3 &) &= \\
		&&&& \frac{55}{12} &&& - & \frac{1}{6}x_2 & - & \frac{7}{12}x_3 & + & \frac{1}{4}y_1 & - & \frac{1}{12}y_3 &.& \\
	\end{linear}
	\]
	
	В выражении целевой функции $x_0$ так же есть $x_1$, избавимся от неё там
	\[
	\begin{linear}{rrrrrrrrrrrrrrrrrr}
		& x_0 = &&& \frac{220}{3} & + & \cancel{\frac{9}{5}x_1} & + & \frac{4}{3}x_2 & + & \frac{5}{3}x_3 &&& - & \frac{11}{15}y_3 && + \\
		&& \frac{9}{5}&(&\frac{125}{12} &&& - & \frac{5}{6}x_2 & - & \frac{5}{12}x_3 & - & \frac{5}{4}y_1 & + & \frac{1}{12}y_3 &) &= \\
		&&&& \frac{1105}{12} &&& - & \frac{1}{6}x_2 & + & \frac{11}{12}x_3 & - & \frac{9}{4}y_1 & - & \frac{7}{12}y_3 &.& \\
	\end{linear}
	\]
	
	Перейдём к следующей итерации.
	
	\item[\fbox{Шаг 2}] Текущее решение --- это
	\[
	x_1 = \frac{125}{12}, \quad x_2 = x_3 = 0, \quad x_4 = \frac{55}{12}, \quad y_1 = 0, \quad y_2 = \frac{455}{12}, \quad x_0 = \frac{1105}{12}.
	\]
	
	Вновь не ве коэффициенты в выражении целевой функции отрицательны, в частности, коэффициент перед $x_3$. Добавим $x_3$ в базис и рассмотрим кандидатов на удаление из базиса
	\[
	x_1: \quad \frac{\frac{125}{12}}{\frac{5}{12}} = 25,
	\]
	\[
	x_4: \quad \frac{\frac{55}{12}}{\frac{7}{12}} = \frac{55}{7}.
	\]
	
	Удалим из базиса $x_4$, тогда текущий базис --- $\{x_1, x_3, y_2\}$. Напишем выражение для $x_3$ через не базисные переменные
	\[
	\begin{linear}{rrrrrrrrrrrrrr}
		& \frac{7}{12}x_3 & = & \frac{55}{12} & - & \frac{1}{6}x_2 & - & 1x_4 & + & \frac{1}{4}y_1 & - & \frac{1}{12}y_3, && \\
		& x_3 & = & \frac{55}{7} & - & \frac{2}{7}x_2 & - & \frac{12}{7}x_4 & + & \frac{3}{7}y_1 & - & \frac{1}{7}y_3. && \\
	\end{linear}
	\]
	
	Если исключить $x_3$ из $x_1$ и $y_2$ и написать выражение для целевой функции, исключив из него $x_3$, то оно будет иметь вид
	\[
	\begin{linear}{rrrrrrrrrrrrrrrr}
		& x_0 & = & \frac{695}{7} & + & \underbrace{\dots}_{< 0} x_2 & + & \underbrace{\dots}_{< 0} & x_4 & + & \underbrace{\dots}_{< 0} & y_1 & + & \underbrace{\dots}_{< 0} & y_3, \\
	\end{linear}
	\]
	
	то есть все коэффициенты перед не базисными переменными в $x_0$ будут отрицательными, значит текущее решение (занулить все не базисные переменными) будет оптимальным, а максимальное значение целевой функции --- $\frac{695}{7}$.
\end{enumerate}

\section{Практика применения симплекс-метода}

\problem

Решить задачу линейного программирования
\[
\begin{linear}{crrrrrrrl}
	& 2x_1 & + & 3x_2 & + & 7x_3 & + & 9x_4 & \to \max\limits_{x}, \\
	& 1x_1 & + & 1x_2 & + & 1x_3 & + & 1x_4 & \le 9, \\
	& 1x_1 & + & 2x_2 & + & 4x_3 & + & 8x_4 & \le 24. \\
\end{linear}
\]

\solution

Решать задачу будем с помощью \hyperref[sec:simplex_algorithm]{Симплекс-метода}. Запишем задачу через равенство с переменными $y_i$
\[
\begin{linear}{crrrrrrrrrrrrrl}
	& \max &\big(&2x_1 & + & 3x_2 & + & 7x_3 & + & 9x_4&\big), \\
	&&& 1x_1 & + & 1x_2 & + & 1x_3 & + & 1x_4 & + & 1y_1 &&& = 9, \\
	&&&  1x_1 & + & 2x_2 & + & 4x_3 & + & 8x_4 &&& + & 1y_2 & = 24.
\end{linear}
\]

\begin{enumerate}
	\item[\fbox{Шаг 0}] Пусть начальный базис --- $\{x_2, x_3\}$. Запишем выражение для $x_2$ с помощью первого ограничения
	\[
	\begin{linear}{rrrrrrrrrrrr}
		& x_2  & = & 9 & - & 1x_1 & - & 1x_3 & - & 1x_4 & - & 1y_1.
	\end{linear}
	\]
	
	Выразим $x_3$ через не базисные переменные с помощью второго ограничения, исключив из выражения $x_2$
	\[
	\begin{linear}{rrrrrrrrrrrrrrrrrrrrr}
		& 4x_3 = &&& 24 & - & 1x_1 & - & 2x_2 &&& - & 8x_4 &&& - &  1y_2 &,&&& \\
		& x_3 = &&& 6 & - & \frac{1}{4}x_1 & - & \cancel{\frac{1}{2}x_2} &&& - & 2x_4 &&& - &  \frac{1}{4}y_2 &&+ \\
		&& -\frac{1}{2}&(&9 & - & 1x_1 &&& - & 1x_3 & - & 1x_4 & - & 1y_1 &&&) &, \\
		& \frac{1}{2}x_3 = &&& \frac{3}{2} & + & \frac{1}{4}x_1 &&&&& - & \frac{3}{2}x_4 & + & \frac{1}{2} y_1& - &  \frac{1}{4}y_2 &,& \\
		& x_3 = &&& 3 & + & \frac{1}{2}x_1 &&&&& - & 3x_4 & + & 1y_1& - &  \frac{1}{2}y_2 &.& \\
	\end{linear}
	\]

	Исключим $x_3$ из $x_2$
	\[
	\begin{linear}{rrrrrrrrrrrrrrrrrrr}
		& x_2 = &&& 9 & - & 1x_1 & - & \cancel{1x_3} & - & 1x_4 & - & 1y_1 &&&&+ \\
		&& -&(&3 & + & \frac{1}{2}x_1 &&& - & 3x_4 & + & 1y_1 & - & \frac{1}{2}y_2 &) &= \\
		&&&& 6 & - & \frac{3}{2}x_1 &&& + & 2x_4 & - & 2y_1& + & \frac{1}{2}y_2 &.& \\
	\end{linear}
	\]
	
	Выразим целевую функцию $x_0$ через не базисные переменные, исключив из выражения $x_2$ и $x_3$
	\[
	\begin{linear}{rrrrrrrrrrrrrrrrrrr}
		& x_0 = &&&&& 2x_1 & + & \cancel{3x_2} & + & \cancel{7x_3} & + & 9x_4 &&&&&& + \\
		&& 3&(&6 & - & \frac{3}{2}x_1 &&&&& + & 2x_4 & - & 2y_1 & + & \frac{1}{2}y_2 &) & + \\
		&& 7&(&3 & + & \frac{1}{2}x_1 &&&& & - & 3x_4 & + & 1y_1 & - & \frac{1}{2}y_2&) & = \\
		&&&& 39 & + & 1x_1 &&&& & - & 6x_4 & + & 1y_1 & - & 2y_2 &.&
	\end{linear}
	\]
	
	\item[\fbox{Шаг 1}] Видно, что все не коэффициенты в выражении для $x_0$ отрицательны (например, коэффициент перед $x_1$). Это значит, что текущее решение (занулить все не базисные переменные)
	\[
	x_1 = 0, \quad x_2 = 7, \quad x_3 = 3, \quad x_4 = y_1 = y_2 = 0, \quad x_0 = 39,
	\]
	
	не является оптимальным, поскольку значение $x_0$ можно увеличить при $x_1 > 0$.
	
	Добавим $x_1$ в базис, удалив из него одну переменную. Рассмотрим выражения базисных переменных. Коэффициент перед $x_1$ отрицателен лишь в выражении для $x_2$, поэтому удалим из базиса $x_2$ и добавим $x_1$. Таким образом, новый базис --- $\{x_1, x_3\}$.
	
	Выразим $x_1$ через не базисные переменные с помощью выражения для $x_2$
	\[
	\begin{linear}{rrrrrrrrrrrr}
		& \frac{3}{2}x_1 = & 6 & - & 1x_2 & + & 2x_4 & - & 2y_1 & + & \frac{1}{2}y_2 &, \\
		& x_1 = & 4 & - & \frac{2}{3}x_2 & + & \frac{4}{3}x_4 & - & \frac{4}{3}y_1 & + & \frac{1}{3}y_2 &. \\
	\end{linear}
	\]
	
	Исключим $x_1$ из $x_3$
	\[
	\begin{linear}{rrrrrrrrrrrrrrrrr}
		& x_3 = &&& 3 & + & \cancel{\frac{1}{2}x_1} &&& - & 3x_4 & + & 1y_1 & - &  \frac{1}{2}y_2 && + \\
		&& \frac{1}{2}&(&4 &&& - & \frac{2}{3}x_2 & + & \frac{4}{3}x_4 & - & \frac{4}{3}y_1 & + & \frac{1}{3}y_2 &) &= \\
		&&&& 5 &&& - & \frac{1}{3}x_2 & - & \frac{7}{3}x_4 & + & \frac{1}{3}y_1& - & \frac{1}{3}y_2 &.& \\
	\end{linear}
	\]
	
	Исключим $x_1$ из $x_0$
	\[
	\begin{linear}{rrrrrrrrrrrrrl}
		& x_0 = & 39 & + & \cancel{1x_1} &&& - & 6x_4 & + & 1y_1 & - & 2y_2 & + \\
		&& 4 &&& - & \frac{2}{3}x_2 & + & \frac{4}{3}x_4 & - & \frac{4}{3}y_1 & + & \frac{1}{3}y_2& = \\
		&& 43 &&& - & \frac{2}{3}x_2 & - & \frac{14}{3}x_4 & - & \frac{1}{3}y_1 & - & \frac{5}{3}y_2 &.
	\end{linear}
	\]
	
	\item[\fbox{Шаг 2}] Видно, что все коэффициенты в $x_0$ отрицательны, значит текущее решение является оптимальным. Таким образом,
	\[
	x_1^* = 4, \quad x_2^* = 0, \quad x_3^* = 5, \quad x_4^* = 0, \quad x_0^* = 43
	\]
	
	\begin{note}
		Также $y_1^* = y_2^* = 0$, однако в исходной задаче этих переменных нет.
	\end{note}
\end{enumerate}

\subsection{Решение задачи раскроя}

\problem[раскроя]

Рассмотрим задачу аналогичную \hyperref[pr:cutting_stock]{Задаче} из первой главы, но с небольшими отличиями. Требуется

\begin{itemize}[nosep]
	\item 30 труб по 3.5 м,
	
	\item 21 труба по 2.5 м.
\end{itemize}

\bigskip

а покупать трубы можно

\begin{itemize}[nosep]
	\item длиной 8 м по 8 руб/шт,
	
	\item длиной 6 м по 6 руб/шт.
\end{itemize}

Необходимо закрыть потребность в трубах с минимальными затратами.

\mathmodel

Рассмотрим все варианты раскроя длинных труб на короткие (таблицы 4.1 и 4.2).

\begin{table}[h!]
	\centering
	\begin{tabular}{| c | c | c | } 
		\hline
		№ & 3.5 м & 2.5 м \\ 
		\hline
		1 & 2 & 0 \\\hline
		2 & 1 & 1 \\\hline
		3 & 0 & 3 \\\hline
	\end{tabular}
	\caption{Варианты раскроя трубы длиной 8 м}
\end{table}

\begin{table}[h!]
	\centering
	\begin{tabular}{| c | c | c | } 
		\hline
		№ & 3.5 м & 2.5 м \\ 
		\hline
		4 & 1 & 1 \\\hline
		5 & 0 & 2 \\\hline
	\end{tabular}
	\caption{Варианты раскроя трубы длиной 6 м}
\end{table}

Можно заметить, что варианты №2 и №4 дают один и тот же результат (по одной трубе 3.5 м и 2.5 м), но один из них требует трубу длиной 8 м, а второй --- 6 м. Поскольку необходимо минимизировать затраты, то вариант №4 всегда выгоднее варианта №2, поэтому №2 можно отбросить. Таким образом, есть 4 способа раскроя длинных труб.

Введём переменные, соответствующие вариантам раскроя
\begin{itemize}[nosep]
	\item $x_1$ --- №1,
	
	\item $x_2$ --- №3,
	
	\item $x_3$ --- №4,
	
	\item $x_4$ --- №5,
\end{itemize}

где $x_j$ показывает, сколько длинных труб будет раскроено соответствующим способом. Напишем целевую функцию и ограничения в терминах переменных

\[
8x_1 + 8x_2 + 6x_3 + 6x_4 \to \min_{x},
\]
\[
2x_1 + x_3 \ge 30,
\]
\[
3x_2 + x_3 + 2x_4 \ge 21.
\]

\sout{О том, что $x_1, x_2, x_3, x_4 \ge 0$ говорить не будем}.

\solution

Решать задачу будем с помощью \hyperref[sec:simplex_algorithm]{Симплекс-метода}. Запишем задачу через равенство с переменными $y_i$
\[
\begin{linear}{crrrrrrrrrrrrrl}
	& \min &\big(&8x_1 & + & 8x_2 & + & 6x_3 & + & 6x_4&\big), \\
	&&& 2x_1 &&& + & 1x_3 &&& - & 1y_1 &&& = 30, \\
	&&&      &   & 3x_2 & + & 1x_3 & + & 2x_4 &&& - & 1y_2 & = 21.
\end{linear}
\]

\begin{enumerate}
	\item[\fbox{Шаг 0}] Пусть начальный базис --- $\{x_1, x_2\}$. Выразим базисные переменные и целевую функцию через не базисные
	\[
	\begin{linear}{rrrrrrrr}
		& 2x_1 & = & 30 & - & 1x_3 & + & 1y_1, \\
		& x_1  & = & 15 & - & \frac{1}{2}x_3 & + & \frac{1}{2}y_1.
	\end{linear}
	\]
	
	\[
	\begin{linear}{rrrrrrrrrr}
		& 3x_2 & = & 21 & - & 1x_3 & - & 2x_4 & + & 1y_2, \\
		& x_2 & = & 7 & - & \frac{1}{3}x_3 & - & \frac{2}{3}x_4 & + & \frac{1}{3}y_2.
	\end{linear}
	\]
	
	\[
	\begin{linear}{rrrrrrrrrrrrrrrrrrr}
		& x_0 = &&&&& \cancel{8x_1} & + & \cancel{8x_2} & + & 6x_3 & + & 6x_4 &&&&&& + \\
		&& 8&(&15 &&&&& - & \frac{1}{2}x_3 &&& + & \frac{1}{2}y_1 &&&) & + \\
		&& 8&(&7 &&&&& - & \frac{1}{3}x_3 & - & \frac{2}{3}x_4 &&& + & \frac{1}{3}y_2&) & = \\
		&&&& 176 &&&&& - & \frac{2}{3}x_3 & + & \frac{2}{3}x_4 & + & 4y_1 & + & \frac{8}{3}y_2 &.&
	\end{linear}
	\]
	
	\item[\fbox{Шаг 1}] Видно, что коэффициент перед $x_3$ в выражении $x_0$ отрицателен. Это значит, что текущее решение (занулить все не базисные переменные)
	\[
	x_1 = 15, \quad x_2 = 7, \quad x_3 = x_4 = y_1 = y_2 = 0, \quad x_0 = 176,
	\]
	
	не является оптимальным, поскольку значение $x_0$ можно уменьшить при $x_3 > 0$.
	
	Добавим $x_3$ в базис, удалив из него одну переменную. Рассмотрим выражения базисных переменных. Коэффициент перед $x_3$ отрицателен и в $x_1$, и в $x_2$, поэтому на удаление есть 2 кандидата. Посчитаем соотношения свободных членов к коэффициенту перед $x_3$
	\[
	x_1: \quad \frac{15}{\frac{1}{2}} = 30,
	\]
	\[
	x_2: \quad \frac{7}{\frac{1}{3}} = 21.
	\]
	
	Минимальное из этих чисел --- $21$, поэтому удалим из базиса $x_2$ и добавим $x_3$. Таким образом, новый базис --- $\{x_1, x_3\}$.
	
	Выразим $x_3$ через не базисные переменные с помощью выражения для $x_2$
	\[
	\begin{linear}{rrrrrrrrrr}
		& \frac{1}{3}x_3 & = & 7- & 1x_2 & - & \frac{2}{3}x_4 & + & \frac{1}{3}y_2, \\
		& x_3 & = & 21- & 3x_2 & - & 2x_4 & + & y_2.
	\end{linear}
	\]
	
	Исключим $x_3$ из $x_1$
	\[
	\begin{linear}{rrrrrrrrrrrrrrrrr}
		& x_1 = &&& 15 &&& - & \cancel{\frac{1}{2}x_3} &&& + & \frac{1}{2}y_1 &&&& + \\
		&& -\frac{1}{2}&(&21 & - & 3x_2 &&& - & 2x_4 &&& + & 1y_2&) & = \\
		&&&& \frac{9}{2} & + & \frac{3}{2}x_2 &&& + & 1x_4 & + & \frac{1}{2}y_1 & - & \frac{1}{2}y_2 &.&
	\end{linear}
	\]
	
	Исключим $x_3$ из $x_0$
	\[
	\begin{linear}{rrrrrrrrrrrrrrrrr}
		& x_0 = &&& 176 &&& - & \cancel{\frac{2}{3}x_3} & + & \frac{2}{3}x_4 & + & 4y_1 & + & \frac{8}{3}y_2 && + \\
		&& -\frac{2}{3}&(&21 & - & 3x_2 &&& - & 2x_4 &&& + & 1y_2 &) & = \\
		&&&& 162 & + & 2x_2 &&& + & 2x_4 & + & 4y_1 & + & 2y_2 &.&
	\end{linear}
	\]
	
	\item[\fbox{Шаг 2}] Видно, что все коэффициенты в $x_0$ положительны, значит текущее решение является оптимальным. Таким образом,
	\[
	x_1^* = \frac{9}{2}, \quad x_2^* = 0, \quad x_3^* = 21, \quad x_4^* = 0.
	\]
	
	\begin{note}
		Также $y_1^* = y_2^* = 0$, однако в исходной задаче этих переменных нет.
	\end{note}
	
	\textbf{Ответ}:
	\begin{itemize}[nosep]
		\item нужно взять 5 $\times$ 8 м и 21 $\times$ 6 м;
		
		\item каждую трубу длиной 8 м раскроить на 2 $\times$ 3.5 м;
		
		\item каждую трубу длиной 6 м раскроить на 1 $\times$ 3.5 м и 1 $\times$ 2.5 м;
		
		\item суммарные затраты составят 176 рублей.
	\end{itemize}
\end{enumerate}
